\documentclass{article}

\usepackage[letterpaper, margin=0.75in]{geometry}
\usepackage{amsmath}
\usepackage{amssymb}
\usepackage{amsfonts}
\usepackage{xcolor}
\usepackage{hhline}
\usepackage[shortlabels]{enumitem}
\usepackage{amsthm}
\usepackage{bm}   
\usepackage{color}   %May be necessary if you want to color links
\usepackage[bookmarks,hypertexnames=false,debug,linktocpage=true,hidelinks]{hyperref}  

\hypersetup{
    colorlinks,
    linktoc=all,
    linkcolor={blue},
    citecolor={blue}, 
    urlcolor={blue}  
} 
%\setcounter{section}{-1}

\newcommand{\ints}{\mathbb{Z}} 
\newcommand{\reals}{\mathbb{R}} 
\newcommand{\rats}{\mathbb{Q}} 
\newcommand{\comps}{\mathbb{C}}
\newcommand{\set}[1]{ \{ #1 \} }
\newcommand{\mult}{\star}
\newcommand{\abs}[1]{| #1 |}
\newcommand{\inv}[1]{ {#1}^{-1} }
\newcommand{\id}{ \bm{1} }
\newcommand{\iso}{ \cong }
\newcommand{\comp}{ \circ }
\newcommand{\lcm}{ \textrm{lcm}, }
\newcommand{\Aut}{ \textrm{Aut}, }
\newcommand{\Stab}{ \textrm{Stab}, }
\newcommand{\sheaf}{ \mathcal{O} }
\newcommand{\md}{\,\text{mod}\,}
\newcommand{\floor}[1]{\lfloor #1 \rfloor}
\newcommand{\ceil}[1]{\lceil #1 \rceil}
\newcommand{\cent}{Z}
\newcommand{\central}{C}
\newcommand{\nor}{N}
\renewcommand{\bar}{\overline}
\newcommand{\norm}[1]{|#1|}
\newcommand{\divides}{\big\vert}
\newcommand{\cyclic}[1]{<#1>}
\DeclareMathOperator*{\argmin}{\arg\!\min}

\newtheorem{theorem}{Theorem}[section]
\newtheorem{corollary}{Corollary}[theorem]
\newtheorem{lemma}[theorem]{Lemma} 

\setlength\parindent{0pt}
%\allowdisplaybreaks

\title{TITLE}
\begin{document}
\setcounter{tocdepth}{2}
\tableofcontents 
\setcounter{section}{1} 
\section{Subgroups}  

\subsection{Definition and Examples}
Throughout these exercises, I will denote subgroups by $H$ unless otherwise specified.
\subsubsection{}\label{ex1p1}
\begin{enumerate}[(a)]
\item Let $a+ai,b+bi \in H$. Then $a+ai-(b+bi) = (a-b) + (a-b)i \in H$
\item Let $z,y \in H$. Then $|z\inv{y}| = |z||\inv{y}| = |z|\inv{|y|} = 1\cdot \inv{1} = 1$, so $z\inv{y} \in H$
\item Let $\frac{a}{b}, \frac{c}{d} \in H$. So $b,d|n$, so $xb=n,yd=n$ for ints $x,y$. Let $g = (b,d)$, and $l=(b,d)$.\\
(Recall from Chapter 0 that $gl = bd$, and also $g=rb+sd$ for some ints $r,s$).\\
Then $\frac{a}{b}-\frac{c}{d} = \frac{ad-bc}{bd} = \frac{\frac{1}{g}(ad-bc)}{\frac{1}{g}bd} = \frac{\frac{1}{g}(ad-bc)}{l}$. (Note that the numerator is an integer, since $g|ad-bc$).\\
We need to show that $l|n$.\\
Note that $m=\frac{gy}{b}$ is an integer, because
\begin{align*}
\frac{gy}{b} &= \frac{rby+sdy}{b}\\
&= ry + \frac{sdy}{b}\\
&= ry + \frac{sn}{b}\\
&= ry + sx
\end{align*}
and note that $l=\frac{bd}{g}$, so
\begin{align*}
lm &= \frac{bd}{g}\cdot\frac{gy}{b}\\
&= yd\\
&= n
\end{align*}
Hence $l|n$ as needed.
\item Let $\frac{a}{b}, \frac{c}{d} \in H$, so $(b,n),(d,n)=1$.\\
Suppose $(bd,n) > 1$, so there is an integer $s>1$ such that $s|bd$ and $s|n$. Let $p$ be a prime factor of $s$, so we have $p|n$ and $p|bd$. The latter implies that $p|b$ or $p|d$. Assume without loss of generality the former. Then $p|n$ and $p|d$, but $(b,n)=1$, a contradiction. Hence $(bd,n)=1$, and $\frac{a}{b}-\frac{c}{d} = \frac{ad-bc}{bd} \in H$.
\item Let $x,y \in H$. Then $x^2=\frac{a}{b}, y^2=\frac{c}{d}$, for $a,b,c,d \in \ints$. Then $(x\inv{y})^2 = x^2y^{-2}=\frac{ad}{bc}$. So $x\inv{y}\in H$.
\end{enumerate}
\subsubsection{}\label{ex1p2}
We will show that closure is not satisfied in all these exercises
\begin{enumerate}[(a)]
\item $(3\ 1)\comp(1\ 2) = (1\ 2\ 3) \notin H$ \label{ex1p2pa}
\newcommand{\lereflect}{r^{\ceil{\frac{n}{2}}}s}
\item $\lereflect$ is a reflection (across the line passing through the second vertex and origin), but $(\lereflect)s = r^{\ceil{\frac{n}{2}}}$ which is a rotation \label{ex1p2pb}
\item Since $n$ is composite, $n=ab$ for $0<a,b<n$.\\
Let $x\in G$ with $|x|=n$. If $H$ is a subgroup, note that $x$ times itself $a$ times must be in $H$ by closure, i.e. $x^a \in H$. But $(x^a)^b = x^n = 1$, so $|x^a| \leq b < n$, so $x^a \notin H$ by definition, a contradiction.
\item $1+1=2$
\item Note that $\sqrt{2},\sqrt{3}\in H$. But $(\sqrt{2}+\sqrt{3})^2  = 5 + 2\sqrt{6}$ which is irrational, so $\sqrt{2}+\sqrt{3} \notin H$.
\end{enumerate}
\subsubsection{}\label{ex1p3}
\begin{enumerate}[(a)]
\item Each element is its own inverse, so we verify the operation is closed on the rest of the elements:
\begin{align*}
(r^2)(s) &= sr^{-2} = sr^2 \in H\\
(r^2)(sr^2) &= sr^{-2}r^2 = s \in H\\
(s)(r^2) &= sr^2 \in H\\
(s)(sr^2) &= s^2r^2 = r^2 \in H\\
(sr^2)(r^2) &= sr^4 = s \in H\\
(sr^2)(s) &= ssr^{-2} =s^2r^2 = r^2 \in H
\end{align*}
\item Again, each element is its own inverse, so we verify closure:
\begin{align*}
(r^2)(sr) &= sr^{-2}r = s\inv{r} = sr^3 \in H\\
(r^2)(sr^3) &= sr^{-2}r^3 = sr \in H
(sr)(sr^3) &= srsr^3 = s^2\inv{r}r^3 = s^2r^2 = r^2 \in H\\
(sr)(r^2) &= sr^3 \in H\\
(sr^3)(r^2) &= sr^5 = sr \in H\\
(sr^3)(sr) &= sr^3sr = s^2r^{-3}r = r^{-2} = r^2 \in H
\end{align*}
\end{enumerate}

\subsubsection{}\label{ex1p4}
Let $G=\ints$, and $H$ be the positive even integers.

\subsubsection{}\label{ex1p5}
Let $G=\set{a_1,\ldots,a_2,\ldots,a_n=1}$ (the elements listed are distinct), and assume without loss of generality that\\
$H = \set{a_2,\ldots,a_n=1}$.\\
Note that we can't have $a_1a_2=a_1$ or $a_1a_2=a_2$, or else by cancelation we will have $a_2=1$ or $a_1=1$ respectively, both contradictions.\\
So assume without loss of generality that $a_1a_2=a_3$ 
(it's okay if $n=3$ and $a_3=1$), yielding
\begin{equation}
a_1 = a_3\inv{a_2}
\end{equation}
\begin{itemize}
\item Case: $\inv{a_2} = a_1$.\\
Then $a_2\in H$, but $\inv{a_2} \notin H$, so inverses are not in $H$
\item Case: $\inv{a_2} \neq a_1$\\
Then $\inv{a_2} \in H$. But $a_3\inv{a_2} = a_1 \notin H$, so closure is not satisfied.
\end{itemize}
\subsubsection{}\label{ex1p6}
Let $G$ be abelian and $g,h \in H$. Then $|g|=n,|h|=m$, with $n,m<\infty$.\\
Then $(g\inv{h})^{nm} = (g^n)^m(h^m)^{-n} = 1$. (The first equality follows from the fact that $G$ is abelian). Hence, $|g\inv{h}| \leq nm \leq \infty$.\\
Now suppose $H=S_{\infty}$ a non-abelian group. Consider the permutations
\begin{equation}
\sigma = (1\ 2)(3\ 4)(5\ 6)\cdots\\
\tau = (2\ 3)(4\ 5)(6\ 7)\cdots
\end{equation}
Individually, they just swap elements, so $|\sigma|,|\tau| = 2$, but $|\tau\comp\sigma| = \infty$. 
\subsubsection{}\label{ex1p7}
The torsion subgroup is clearly $H= 0\times \ints/n\ints$\\
Let $I = (G-H)\cup\set{0}$\\
Then $(2,1),(2,0)\in I$. But $(2,1)-(2,0) = (0,1)$, is not in $I$ (It is a nonzer element in $H$)
\subsubsection{}\label{ex1p8}
\begin{itemize}
\item Only if: Suppose $H\cup K$ is a subgroup\\
Suppose $K\not\subset H$, so there exists $k\in K$ such that $k\notin H$\\
Then let $h \in H$.\\
Then $h,k\in H\cup K$, so $hk\in H\cup K$ (since $H\cup K$ is a subgroup).\\
If $hk \in H$, then $\inv{h}(hk) \in H$, so $k\in H$, a contradiction.\\
So we must have $hk \in K$. But then $(hk)\inv{k} \in K$, so $h\in K$.\\
Since $h\in H$ was arbitrary, $H \subset K$
\item If: Assume without loss of generality that $H \subset K$\\
Let $x,y \in H\cup K$. Then $x,y\in K$ (since $H\subset K$), so $x\inv{y} \in K = H\cup K$
\end{itemize}
\subsubsection{}\label{ex1p9}
Let $A,B\in \textrm{SL}_nF$, so $\det A, \det B =1$. Then $\det A\inv{B} = \det A \det \inv{B} = 1$
\subsubsection{}\label{ex1p10}
\begin{enumerate}[(a)]
\item See next part
\item
Let $\set{H_i}_{i\in I}$ be a collection of subgroups of $G$
and let $H = \bigcap_{i\in I}H_i$.\\
Then let $x,y \in H$. So for arbitrary $i$. $x \in H_i$ If $y\in H_i$ for all $i$, then $\inv{y}\in H_i$. Then $x\inv{y} \in H_i$. Since $i$ was arbitrary $x\inv{y} \in H=\bigcap H_i$
\end{enumerate}
\subsubsection{}\label{ex1p11}
\begin{enumerate}[(a)]
\item Let $(a,1),(k,1)\in H$. Then $(a,1)\inv{(k,1)} = (a,1)(\inv{k},1) = (a\inv{k},1) \in H$
\item Analogous to above
\item Let $(a,a),(b,b) \in H$. Then $(a,a)\inv{(b,b)} = (a,a)(\inv{b},\inv{b}) = (a\inv{b},a\inv{b}) \in H$
\end{enumerate}
\subsubsection{}\label{ex1p12}

\begin{enumerate}[(a)]
\item Let $x,y\in H$, with $x=a^n, y=b^n$. Then $x\inv{y} = a^nb^{-n} = (a\inv{b})^n \in H$ (note we used that $A$ is abelian here)
\item Let $x,y\in H$, so $x^n,y^n=1$. Then $(x\inv{y})^n = x^n\inv{(y^n)} = 1$, so $x\inv{y} \in H$.
\end{enumerate}
\subsubsection{}\label{ex1p13}
\subsubsection{}\label{ex1p14}
\newcommand{\lereflect}{r^{\ceil{\frac{n}{2}}}s}
See \ref{ex1p2pb}. $\lereflect$ is a reflection, but $(\lereflect)s = r^{\ceil{\frac{n}{2}}}$ which does not have order $2$ in general. For the case $n=4$ consider the reflection $sr$ across the line $y=0$, and note that $s(sr) = r$ which has order $4$
\subsubsection{}\label{ex1p15}
Let $H = \bigcup_{i=1}^{\infty}H_i$, and let $x,y\in H$. So $x \in H_i, y\in H_j$ for some $i,j$. Assume without loss of generality that $i \geq j$. Then $y \in H_i$, since $H_j \subset H_i$. So $x\inv{y} \in H_i \subset H$
\subsubsection{}\label{ex1p16} 
The inverse of an upper triangular matrix is upper triangular (because the adjoint is upper triangular), and they are closed under multiplication
\subsubsection{}\label{ex1p17}
The same logic as above, with the additional note that diagonals of $1$s are preserved after matrix multiplication. Diagonals of $1$s are preserved by taking inverses too (the adjoint is the transpose of the cofactor matrix, and the minors along the diagonal are all clearly just $1$)
\subsection{Centralizers and Normalizers. Stabilizers and Kernels}
\subsubsection{}\label{ex2p1}
Follows from 
\begin{align*}
ga\inv{g} &= a\\
\iff ga &= ag\\
iff a &= \inv{g}ag
\end{align*}
we will take these equivalences for granted moving forward
\subsubsection{}\label{ex2p2}
Let $x \in G$. And note that given $g \in Z(G)$, we have $gx=xg$. This is because by definition of $Z(G)$, $x$ commutes with all elements of $G$, including $g$. But this also says that $g$ commutes with all elements of $x$ of $Z(G)$, so $g \in C_G(Z(G))$.\\
Since $g\in G$ was arbitrary, $G \subset C_G(Z(G))$ and since the reverse inclusion trivially holds, $C_G(Z(G)) = G$.\\
We already know from the section that the centralizer is contained in the normalizer, i.e. $G = C_G(Z(G)) \subset N_G(Z(G))$, so again the trivial reverse inclusion yields $G = N_G(Z(G))$.
\subsubsection{}\label{ex2p3}
If $g \in C_G(B)$, then given $a\in A$, we have $a \in B$, so $ga=ag$ holds. Hence $g \in C_G(A)$ (since $a\in A$ was arbitrary). And since $g \in C_G(B)$ was arbitrary, $C_G(B) \leq C_G(A)$.
\subsubsection{}\label{ex2p4}
\begin{itemize}
\item $S_3$\\
Recall that
\begin{equation}
S_3 = \set{1,(1\ 2),(2\ 3),(1\ 3),(1\ 2\ 3),(1\ 3\ 2)}
\end{equation}
of course $|S_3| = 6$, so all the centralizers must divide $6$ (by Lagrange's theorem). \\
Trivially, $C(1) = S_3$\\
Now we note from the example in the text that, where $A=\set{1, (1\ 2)}$,  $C_{S_3}(A) = A$. For now, denote $C=C(1,2)$ for convenience\\
By the previous exercises,
\begin{align*}
\set{(1,2)} &\subset A\\
\implies C(A) &\leq C((1,2))\\
\implies \norm{C(A)} &\vert \norm{C}\\
\implies 2 &\vert \norm{C}
\end{align*}
So we have $2\vert\norm{C}\vert 6$, which means we must have $\norm{C}=2$ or $\norm{C}=6$. But $(1\ 2)$ does not commute with $(1\ 2\ 3)$, so $\norm{C}=2$, and $C(A) \leq C$ implies that $C = C(A)$. I.e. we conclude that $C( (1\ 2) ) = \set{1, (1\ 2)}$. With entirely analogous arguments for the other "swapping" permutations, we obtain
\begin{align*}
C( (1\ 2) ) &= \set{1, (1\ 2)}\\
C( (1\ 3) ) &= \set{1, (1\ 3)}\\
C( (2\ 3) ) &= \set{1, (2\ 3)}\\
\end{align*}
Now denote $C = C( (1\ 2\ 3) )$. We have $1, (1\ 2\ 3) \in C$ automatically. And since $C$ is a subgroup, we must have $\inv{(1\ 2\ 3)} = (1\ 3\ 2) \in C$. Now $\norm{C} \leq 3$, but by $\norm{C} \divides 6$, this narrows it down to $\norm{C} = 3$ or $\norm{C} = 6$. But since $(1\ 2)$ doesn't commute with $(1\ 2\ 3)$, we must have $\norm{C} = 3$. Hence, $C( (1\ 2\ 3) ) = \set{1, (1\ 2\ 3), (1\ 3\ 2)}$. An analagous argument applies for $C( (1\ 3\ 2) )$, yielding the same set. So, in total, we have
\begin{align*}
C(1) &= S_3\\
C( (1\ 2) ) &= \set{1, (1\ 2)}\\
C( (1\ 3) ) &= \set{1, (1\ 3)}\\
C( (2\ 3) ) &= \set{1, (2\ 3)}\\
C( (1\ 2\ 3) ) &= \set{1, (1\ 2\ 3), (1\ 3\ 2)}\\
C( (1\ 3\ 2) ) &= \set{1, (1\ 2\ 3), (1\ 3\ 2)}
\end{align*}
Finally, since all the centralizers are missing elements except for $1$, we have $Z(S_3) = 1$
\item $D_8$\\
We have
\begin{equation}
D_8 = \set{1, r, r^2, r^3, s, sr, sr^2, sr^3}
\end{equation}
So all centralizers must divide $8$.\\
The examples give us the following info. Where $A = \set{1,r,r^2,r^3}$,
\begin{align*}
Z(D_8) &= \set{1, r^2}\\
C(A) &= A
\end{align*}
So the center has already been computed, and of course $C(1) = D_8$.\\
Next, we have
\begin{align*}
\set{r} &\subset A\\
\implies C(A) &\leq C(r)\\
\implies 4 &\divides \norm{C(r)}
\end{align*}
Combined with the fact $\norm{C(r)}\divides 8$, we must have $\norm{C(r)}=4$ or $\norm{C(r)}=8$. However, $s$ and $r$ don't commute ($rs = s\inv{r} \neq sr$), so $s \notin C(r)$ and we must hae $\norm{C(r)} = 4$. Since $C(A) \leq C(r)$, this means that $C(r) = C(A)$.\\
Analagously, $\set{r^3} \subset A$ implies $\norm{C(r^3)}=4$ or $\norm{C(r^3)}=8$, but $s$ and $r^3$ don't commute, yielding $C(r^3) = C(A)$.\\
Next, since $r^2 \in Z(D_8)$, we have $C(r^2) = D_8$. Thus, we have so far concluded that
\begin{align*}
C(1) &= D_8\\
C(r) &= \set{1,r,r^2,r^3}\\
C(r^2) &= D_8\\
C(r^3) &= \set{1,r,r^2,r^3}
\end{align*}
Now consider the element $sr^k$ ($k=0,1,2,3$). Automatically, we have $1,r^2,sr^k\in C(sr^k)$. So $\norm{C(sr^k)} \geq 3$. But $\norm{C(sr^k)} \divides 8$, so we must have $\norm{C(sr^k)} = 4$ or $\norm{C(sr^k)} = 8$.\\
Now note that 
\begin{align*}
(sr^k)r &= sr^{k+1}\\
r(sr^k) &= sr^{k-1}
\end{align*}
Which are equal if and only if
\begin{align*}
sr^{k+1} &= sr^{k-1}\\
\iff r^kr &= r^k\inv{r}\\
\iff r &= \inv{r}
\end{align*}
which is impossible in $D_8$. Hence $r \notin C(sr^k)$, and we must conclude $\norm{C(sr^k)} = 4$. Hence, we can conclude that
\begin{equation}
C(sr^k) = \set{1,r^2,sr^k,x_k}
\end{equation}
where $x_k$ is another element in $D_8$ that commutes with $sr^k$.\\
We note that $sr$ and $sr^3$ are inverses, and hence commute, so that covers the case $k=1,3$. Also, $sr^2$ commutes with $s$, covering the cases $k=0,2$. Hence, we can now list the centralizers for every element in $D_8$:
\begin{align*}
C(1) &= D_8\\
C(r) &= \set{1,r,r^2,r^3}\\
C(r^2) &= D_8\\
C(r^3) &= \set{1,r,r^2,r^3}\\
C(s) &= \set{1,r^2,s,sr^2}\\
C(sr) &= \set{1,r^2,sr,sr^3}\\
C(sr^2) &= \set{1,r^2,s,sr^2}\\
C(sr^3) &= \set{1,r^2,sr,sr^3}
\end{align*}
\item $Q_8$\\
man suckmndick
\end{itemize}
\subsubsection{}\label{ex2p5}
\begin{enumerate}[(a)]
\item $G=S_3, A=\set{\id, (1\ 2\ 3), (1\ 3\ 2)}$.\\
In the previous exercise, we computed that $C( (1\ 2\ 3) ) = A$
\begin{align*}
\set{(1\ 2\ 3)} & A\\
\implies C(A) &\leq C( (1\ 2\ 3) )\\
\implies C(A) &\leq A\\
\implies \norm{C(A)} &\divides 3
\end{align*}
\begin{itemize}
\item $C(A)=A$\\
So we must have $\norm{C(A)}=1$ or $\norm{C(A)} = 3$. The latter corresponds to $C(A) = A$, since $C(A) \leq A$. And since $C(A) \leq A$, we're checking if the lements of $A$ commute with each other. But each element in $A$ commutes with $1$ and each element in $A$ commutes with itself. So all that's left to check is if $(1\ 2\ 3)$ and $(1\ 3\ 2)$ commute, which they do, since $(1\ 3\ 2) \in A = C( (1\ 2\ 3) )$. Hence $C(A) = A$.
\item $N(A)=G$\\
Note that $A=C(A) \leq N(A)$, so $3 \divides \norm{N(A)}$. since $\norm{S_3} = 6$, we must have $N(A)=A$ or $N(A)=G$. But
\begin{align*}
(1\ 2)\id\inv{(1\ 2)} &= \id\\
&\in A
\end{align*}
and
\begin{align*}
(1\ 2)(1\ 2\ 3)\inv{(1\ 2)} &= (1\ 2)(1\ 2\ 3)(1\ 2)\\
&= (1\ 2)(1\ 3)\\
&= (1\ 3\ 2)\\
&\in A
\end{align*}
and
\begin{align*}
(1\ 2)(1\ 3\ 2)\inv{(1\ 2)} &= (1\ 2)(1\ 3\ 2)(1\ 2)\\
&= (1\ 2)(2\ 3)\\
&= (1\ 2\ 3)\\
&\in A
\end{align*}
Hence $(1\ 2) \in N(A)$, so $N(A) \neq A$ and we must have $N(A) = G$
\end{itemize}
\item $G=D_8, A=\set{1,s,r^2,sr^2}$\\
\begin{itemize}
\item $C(A) = A$\\
From the previous exercise, we know that $C(s) = A$. Then
\begin{align*}
\set{s} &\subset A\\
\implies C(A) &\leq C(s)\\
\implies C(A) &\leq A\\
\implies \norm{C(A)} &\divides 4
\end{align*}
So we must have $\norm{C(A)}=1,2,$ or $4$\\
Now $1\in C(A)$ trivially. And
\begin{align*}
sr^2 &= r^{-2}s =r^2s\\
s(sr^2) &= s^2r^2 = sr^{-2}s = (sr^2)s
\end{align*}
so $s \in C(A)$. And $r^2(sr^2) = sr^{-2}r^2 = s = sr^4 = (sr^2)r^2$,\\
so $r^2 \in C(A)$\\
Hence, $\norm{C(A)} \geq 3$, so we must have $\norm{C(A)} = 4$, i.e. $C(A)=A$.
\item $N(A)=G$\\
We have that $C(A) \leq N(A) \leq D_8$, so $4 \divides \norm{N(A)} \divides 8$, hence we must have $N(A)=A$ or $N(A)=D_8$\\
But
\begin{align*}
rs\inv{r} &= sr^{-2} = sr^2 \in A\\
rr^r\inv{r} &= r^2 \in A\\
r(sr^2)\inv{r} &= rsr = s \in A
\end{align*}
so $r \in N(A)$, and we must have $N(A)=G$
\end{itemize}
\item $G=D_{10},A=\set{1,r,r^2,r^3,r^4}$\\
\begin{itemize}
\item $C(A)=A$\\
$C(A) \leq D_{10}$, so $\norm{C(A)} \divides 10$. So we must have,\\
$\norm{C(A)} = 1,2,5,$ or $10$.\\
The elements of $A$ clearly commute with each other, so $A \leq C(A)$, leaving us with\\
$\norm{C(A)}=5$ or $10$\\
But $rs=s\inv{r} \neq sr$, so $s \notin C(A)$ and we must have $\norm{C(A)} = 5$, i.e. $C(A) = A$.
\item $N(A) = G$\\
As usual $C(A) \leq N(A) \leq D_{10}$, so\\
$5 \divides \norm{N(A)} \divides 10$,\\
so we must have $\norm{N(A)} = 5$ or $10$\\
Bute for $k=0,\ldots,4$, we have $sr^k\inv{s} = sr^ks=s^2r^{-k} = r^{-k} = r^{5-k} \in A$\\
so $s\in A$, which means we must have $\norm{N(A)} = 10$, i.e. $N(A) = G$
\end{itemize}
\end{enumerate}
\subsubsection{}\label{ex2p6}
\begin{enumerate}[(a)]
\item $H \leq N_G(H)$:\\
If $h \in H$, then take arbitrary $g \in H$. Then $hg\inv{g} \in H$ (since $H$ is a subgroup). Since $g$ was arbitrary, $hH\inv{h} \subset H$, and the reverse inclusion is trivial (set $h=1$, which we can do because $H$ is a subgroup). Hence $hH\inv{h} = H$, and $h \in N(H)$. Since $h$ was arbitrary, $H \subset N(H)$ i.e. $H \leq N(H)$\\
For a counterexample when $H$ is not a subgroups, let $G=S_3$ and $H = \set{(1\ 2), (1\ 2\ 3)}$. But $(1\ 2)(1\ 2\ 3)(1\ 2) = (1\ 3\ 2) \notin H$. So $(1\ 2) \notin N(H)$
\item $H \leq C_G(H)$ iff $H$ abelian\\ 
If $H$ is abelian, let $h \in H$ and $g \in H$. Then $g,h$ commute (since $H$ abelian). Since $g \in H$ was arbitrary, $h \in C(H)$, and since $h\in H$ was arbitrary, $H \leq C(H)$\\
If $H \leq C(H)$, let $h,g\in H$. Since $h\in H$ and $g\in C(H)$, $g,h$ commute. Since $g,h \in H$ were arbitrary, $H$ is abelian
\end{enumerate}
\subsubsection{}\label{ex2p7}
We will do both parts in one\\
Suppose $r^i \in Z(D_{2n})$, ($i=0,\ldots,n-1$).\\
Suppose $r^i\in Z(D_{2n})$. Then
\begin{align*}
sr^i &= r^is\\
\implies sr^i &= sr^{-i}\\
\implies r^i &= r^{-i}\\
\implies r^{2i} &= 1
\end{align*}
\textbf{This is only possible when $i=0$ (the case $r^i = 1$), or when $n$ is even and $i = n/2$}\\
Suppose $sr^i \in Z(D_{2n})$, ($i=0,\ldots,n-1$). Then
\begin{align*}
sr^ir &= rsr^i\\
\implies sr^{i+1} &= sr^{i-1}\\
\implies r^{i+1} &= r^{i-1}\\
\implies r&= \inv{r}
\end{align*}
which never occurs for $n \geq 3$
\subsubsection{}\label{ex2p8}
The text already stated that the stabilizer is a subgroup\\
$G_i$ can be easily identified with $S_{n-1}$, so $\norm{G_i} = \norm{S_{n-1}} = (n-1)!$
\subsubsection{}\label{ex2p9}
$N_H(A) \subset N_G(A)$ and $N_H(A) \subset H$ are trivial, so $N_H(A) \subset N_G(A) \cap H$.\\
Now let $h \in N_G(A) \cap H$. Then by definition of $N_G(A)$, $hA\inv{h} = A$, so $h \in N_H(A)$. 
\subsubsection{}\label{ex2p10}
$\norm{H} = 2$\\
\begin{itemize}
\item $N_G(H) = C_G(H)$:\\
If $\norm{H} = 2$, then $H=\set{1,h}$, for some $h \neq 1$ ($h = \inv{h}$).\\
$C(H) \leq N(H)$ as always. So we prove the reverse inclusion. Let $g \in N(H)$. It trivially commutes with $1$, so we have to show that it commutes with $h$ as well.\\
We do know that $gh\inv{g} \in H$ (since $g \in N(H)$). In the case $gh\inv{g} = 1$, we'd have $gh = g \implies h=1$, a contradiction. So we must have $gh\inv{g} = h \implies gh=hg$ as needed.
\item $N_G(H) = G \implies H \leq Z(G)$:
If $N(H) = G$, then $C(H) = G$, as we have just shown. $C_G(H) = G$ means that the elements of $H$ commute with all the elements of $G$. i.e. $H \subset Z(G)$.
\end{itemize}
\subsubsection{}\label{ex2p11}
Let $g \in Z(G)$. Then let $a\in A$. Then since $g$ is in the center, $ga=ag\implies ga\inv{g} = a \implies ga\inv{g} \in A$. Since $a \in A$ was arbitrary, $gA\inv{g} \subset A$, and the reverse inclusion is trivial by $g=1$, so $g\in N_G(A)$. Since $g\in G$ was arbitrary, $Z(G) \subset N_G(A)$
\subsubsection{}\label{ex2p12}
\begin{enumerate}[(a)]
\item
We have 
\begin{align*}
p&=12x_1^5x_2^7x_4 - 18x_2^3x_3 + 11x_1^6x_2x_3^3x_4^{23}\\
\sigma &= (1\ 2\ 3\ 4)\\
\tau &= (1\ 2\ 3)
\end{align*}
And thus
\begin{align*}
\sigma \cdot p &= 12x_2^5x_3^7x_1 - 18x_3^3x_4 + 11x_2^6x_3x_4^3x_1^{23}\\
\tau\cdot(\sigma \cdot p) &= 12x_3^5x_1^7x_2 - 18x_1^3x_4 + 11x_3^6x_1x_4^3x_2^{23}\\
(\tau\comp\sigma)\cdot p &= \tau\cdot(\sigma \cdot p) & \mbox{(since this is an action, by the next part)}\\
(\sigma\comp\tau)\cdot p &= (1\ 3\ 2\ 4)p\\
&= 12x_3^5x_4^7x_1 - 18x_4^3x_2 + 11x_3^6x_4x_2^3x_1^{23}
\end{align*}
\item We verify the axioms \label{ex2p12b} 
\begin{itemize}
\item Identity: $\id\cdot p(x_1,x_2,x_3,x_4) = p(x_{\id{1}},x_{\id{2}},x_{\id{3}},x_{\id{4}}) = p(x_1,x_2,x_3,x_4)$ as needed
\item Associativity: $\tau\cdot(\sigma\cdot p(x_1,x_2,x_3,x_4)) = \tau\cdot p(x_{\sigma(1)},x_{\sigma(2)},x_{\sigma(3)},x_{\sigma(4)}) = p(x_{\tau(\sigma(1))},x_{\tau(\sigma(2))},x_{\tau(\sigma(3))},x_{\tau(\sigma(4))}) = p(x_{\tau\comp\sigma(1)},x_{\tau\comp\sigma(2)},x_{\tau\comp\sigma(3)},x_{\tau\comp\sigma(4)}) = (\tau\comp\sigma)\cdot p(x_1,x_2,x_3,x_4)$
\end{itemize}
\item $\Stab x_4 = \set{\id,(1\ 2),(2\ 3),(1\ 3),(1\ 2\ 3),(1\ 3\ 2)}$, which is clearly isomorphic to $S_3$
\item $\Stab(x_1+x_2) = \set{\id,(1\ 2),(3\ 4),(1\ 2)(3\ 4)}$
\item $\Stab(x_1x_2 + x_3x_4) = \set{\id,(1\ 2),(3\ 4),(1\ 2)(3\ 4),(1\ 3)(2\ 4),(1\ 4)(2\ 3),(1\ 3\ 2\ 4),(1\ 4\ 2\ 3)}$. This is clearly isomorphic to $D_8$, if you label the vertices clockwise as $1, 3, 2, 4$.
\item Trivial
\end{enumerate}
\subsubsection{}\label{ex2p13}
Identical proof to \ref{ex2p12b}
\subsubsection{}\label{ex2p14}
Let $A = \begin{pmatrix}1 & a & b\\ 0 & 1 & c\\ 0 & 0 & 1\end{pmatrix}$ and suppose $A \in Z(H(F))$ Then given $B = \begin{pmatrix}1 & e & f\\
0 & 1 & g\\
0 & 0 & 1\end{pmatrix}$, we must have
\begin{align*}
AB &= BA\\
\implies \begin{pmatrix}1 & a & b\\0 & 1 & c\\0 & 0 & 1\end{pmatrix}\begin{pmatrix}1 & e & f\\0 & 1 & g\\0 & 0 & 1\end{pmatrix} &= \begin{pmatrix}1 & e & f\\0 & 1 & g\\0 & 0 & 1\end{pmatrix}\begin{pmatrix}1 & a & b\\0 & 1 & c\\0 & 0 & 1\end{pmatrix}
\end{align*}
which yields the equation
\begin{align*}
f+ag+b&=b+ce+f\\
\implies ag&=ce
\end{align*}
Setting $g,e=1$ gives $a=c$. And setting $g=1,e=0$ gives $a=0$, so also $c=0$\\
Hence
\begin{equation}
Z(H(F)) = \set{ \begin{pmatrix}1 & 0 & b\\0 & 1 & 0\\0 & 0 & 1\end{pmatrix} | b \in F }
\end{equation}
which can clearly be identified with $F$
\subsection{Cyclic Groups and Cyclic Subgroups}\label{sec2p3}
\subsubsection{}\label{ex3p1}
\subsubsection{}\label{ex3p2}
\subsubsection{}\label{ex3p3}
\subsubsection{}\label{ex3p4}
\subsubsection{}\label{ex3p5}
\subsubsection{}\label{ex3p6}
\subsubsection{}\label{ex3p7}
\subsubsection{}\label{ex3p8}
\subsubsection{}\label{ex3p9}
\subsubsection{}\label{ex3p10}
\subsubsection{}\label{ex3p11}
\subsubsection{}\label{ex3p12}
\begin{enumerate}[(a)]
\item $Z_2\times Z_2 = \set{(0,0), (0,1), (1,0), (1,1)}$,
And all cyclic subgroups take the form $\cyclic{x}=\set{(0,0), x}$, so $\norm{\cyclic{x}} < \norm{Z_2\times Z_2}$ in all cases.\\
\item Given $a,b \in \ints$
\begin{align*}
(1,0) &\notin \cyclic{(0,a)}\\
(0,1) &\notin \cyclic{(1,b)}\\
\end{align*}
So no cyclic subgroup generates all of $Z_2 \times \ints$
\item The cyclic subgroups $\cyclic{(0,y)}, \cyclic{(x,0)}$ are always $0$ in one coordinate and hence do not generate $\ints\times\ints$. The cyclic subgroup $\cyclic{(x,y)}$ comprises only the line through the origin with a slope of $y/x$, not the entire $\ints\times\ints$ plane.
\end{enumerate}
\subsubsection{}\label{ex3p13}
\begin{enumerate}[(a)]
\item $\ints$ is cyclic but $\ints\times Z_2$ is not
\item Consider a homomorphism $\rats \to \rats\times Z_2$. If $a \mapsto (x,0)$, then there is no way to reach $(0,1)$. If $a \mapsto (y,1)$, there is no way to reach $(1,0)$. 
\end{enumerate}
\subsubsection{}\label{ex3p14}
\subsubsection{}\label{ex3p15}
If $\rats\times\rats$ was cyclic, then we'd have an isomorphism $\ints \to \rats\times\rats$. But if $1 \mapsto (b,c)$, that determines the rest of the map as $k \mapsto (kb,kc)$, which is obviously not surjective.
\subsubsection{}\label{ex3p16}
\begin{itemize}
\item Let $l = \lcm(m,n)$ and write $l=am=bn$. Then
\begin{align*}
(xy)^l &= x^ly^l & \mbox{($x,y$ commute)}\\
&= (x^n)^b(y^m)^a\\
&= 1\cdot 1\\
&= 1
\end{align*}
Hence by proposition 3 in the text, $\norm{xy} \divides l$
\item If $x,y$ do not commute, see my \ref{ex1p6} for a counterexample
\item Note that in $D_{12}$, $\norm{r^2} = 3, \norm{s}=2$, so the lcm of the orders is $6$. But $(sr^2)^2 = sr^2sr^2 = ssr^{-2}r^2 = s^2r^0 = 1\cdot 1 = 1$, so $\norm{sr^2} = 2 < 6$
\end{itemize}
\subsubsection{}\label{ex3p17}
$Z_n = \cyclic{x | x^n=1 }$
\subsubsection{}\label{ex3p18}
Let $\phi: Z_n \to H$ with $x \mapsto h$. Then if $\phi$ is a homomorphism, we must have $x^k \mapsto h^k$, uniquely determining $\phi$ if it exists.\\
If $x^a = x^b$, then $x^{a-b}=1$, so $a-b=cn$ for some $c$. Then $\phi(x^a)=h^a$ and $\phi(x^b)=h^b$. But 
\begin{align*}
h^{a-b} &= (h^n)^c\\
\implies h^{a-b} &= 1\\
\implies h^a &= h^b\\
\implies \phi(x^a) &= \phi(x^b)
\end{align*}
So $\phi$ is well-defined and thus exists.
\subsubsection{}\label{ex3p19}
Let $\phi: \ints \to H$ with $1\mapsto h$. Then if $\phi$ is a homomorphism we must have $\phi(k)=k\phi(1)=h^k$, uniquely determining $\phi$.
\subsubsection{}\label{ex3p20}
Suppose $x^{p^n} = 1$. Let $o=\norm{x}$. We know from prop 3 that $o \divides p^n$, so $ao = p^n$ for some $a \in \ints$ (assume wlg $a$ positive). But the prime factorization is unique, so $ao$ must factor into $p^n$, Hence $o=p^m$ for some $m\leq n$
\subsubsection{}\label{ex3p21}
\subsubsection{}\label{ex3p22}
\subsubsection{}\label{ex3p23}
\subsubsection{}\label{ex3p24}
\begin{enumerate}[(a)]
\item Trivial
\item The proof is traced out. $gx^k\inv{g}=(gx\inv{g})^k = (x^a)^k = x^k \in \cyclic{x}$, so $g\cyclic{x}\inv{g} \leq \cyclic{x}$ indeed. Now let $n=\norm{x}$. Given $0 \leq i \leq k \leq n-1$, suppose $gx^k\inv{g} = gx^i\inv{g}$. Then by right multiplying by $g$ and left multiplying by $\inv{g}$ on both sides,
\begin{align*}
x^k &= x^i\\
\implies x^{k-i} &= 1\\
\implies k-i &= 0 & \mbox{(since $0\leq i\leq k\leq n-1$)}\\
\implies k &= i
\end{align*}
The contrapositive yields that the $gx^k\inv{g}$ are distinct for $k=0,\ldots,n-1$. Hence $\norm{g\cyclic{x}\inv{g}} = n = \norm{\cyclic{x}}$, so $\norm{g\cyclic{x}\inv{g}} = \cyclic{x}$.
\end{enumerate}
\subsubsection{}\label{ex3p25}
\begin{itemize} \label{ex3p25a}
\item Let $|G| = n$ and $G=\cyclic{x}$\\
We have $(k,n)=1$. Select $i,j$ such that $0\leq i \leq j \leq n-1$, and suppose $(x^i)^k = (x^j)^k$. Then
\begin{align*}
x^{ik} &= x^{jk}\\
\implies x^{jk - ik} &= 1\\
\implies x^{k(j - i)} &= 1\\
\implies n &\divides k(j-i) & \mbox{(since $\norm{x}=n$)}\\
\implies n &\divides j-i & \mbox{(since $(k,n)=1$)}\\
\implies j-i &= 0 & \mbox{(since $0\leq i \leq j \leq n-1$)}\\
\implies j &= i
\end{align*}
Hence the powers by the residues are all distinct, so the image has order $n$ and hence comprises all of $G$, so the map is surjective.
\item We prove the following lemma:
\begin{lemma} \label{lemma1}
Let $\norm{G} = n < \infty$, $(k,n)=1$. Then $x^k = 1 \implies x = 1$ (so nonidentity elements taken to the $k$th power are never the identity) 
\end{lemma}
\begin{proof}
Let $m = \norm{x^k}$. Note that $\norm{x^k}=\norm{\cyclic{x^k}}$, and hence by Lagrange's theorem, $m \divides n$. Suppose $m=1$. Then $\norm{x^k}=1$ so $x^k=1$, and $\norm{x} \divides k$. But $\norm{x} = \norm{\cyclic{x}} \divides n$ by Lagrange's theorem, so $(k,n) \geq \norm{x}$. But $(k,n)=1$, so $\norm{x} =1$, i.e. $x=1$
\end{proof}
$1^k =1$, of course. Now take $x \in G, x\neq 1$. Then by the lemma, $x^k = y$ for some $y \neq 1$. Now suppose we also have $z \in G$ with $z^k = y$. Denote $z\inv{x} = a$. Then
\begin{align*}
z^k &= y\\
\implies z^k &= x^k\\
\implies z^k(\inv{x})^k = 1\\
\implies a^k &= 1
\end{align*}
By the lemma, we would have $a=1$, so $z\inv{x}=1$ i.e. $z=x$. Hence the map is injective, and an injective map to itself must be surjective.
\end{itemize}
\subsubsection{}\label{ex3p26}
Let $Z_n = \cyclic{x}$
\begin{enumerate}[(a)]
\item As shown in the first part of the previous exercise \ref{exp3p25a}, $\sigma_a$ is injective and surjective.
\item 
\begin{align*}
\sigma_a &= \sigma_b\\
\iff \sigma_a(x) &= \sigma_b(x) & \mbox{(The value on $x$ determines the whole map)}\\
\iff x^a &= x^b\\
\iff x^{a-b} &= 1\\
\iff n &\divides a-b\\
\iff a &\equiv b \pmod{n}
\end{align*}
\item
Let $\sigma \in \Aut(Z_n)$. Then $\sigma(x) = x^a$ for some $a \in \ints$\\
If $\sigma$ is an automorphism, then
\begin{equation}
x^0, x^a, x^{2a},\ldots, x^{(n-1)a} \label{eq2p26}
\end{equation}
are all distinct. Suppose $(a,n)=k$. Write $a=kb,n=km$. Note that $0<m\leq n$. Then\\
\begin{equation}
x^{am} = x^{kbm} = x^{kmb} = x^{nb} = 1
\end{equation}
In order to not contradict the distinctness of the \ref{eq2p26}, we must have $m=n$. But then $n=km \implies k=1$. I.e. $(a,n)=1$, as needed.
\item
$\sigma_a\comp\sigma_b(x) = \sigma_a(x^b) = (x^b)^a = x^{ab} = \sigma_{ab}(x)$
\end{enumerate}
\end{document}












