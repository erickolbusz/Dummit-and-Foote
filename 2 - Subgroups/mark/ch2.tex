\documentclass{article}

\usepackage[letterpaper, margin=0.75in]{geometry}
\usepackage{amsmath}
\usepackage{amssymb}
\usepackage{amsfonts}
\usepackage{xcolor}
\usepackage{hhline}
\usepackage[shortlabels]{enumitem}
\usepackage{amsthm}
\usepackage{bm}   
\usepackage{color}   %May be necessary if you want to color links
\usepackage[bookmarks,hypertexnames=false,debug,linktocpage=true,hidelinks]{hyperref} 

\hypersetup{
    colorlinks,
    linktoc=all,
    linkcolor={blue},
    citecolor={blue},
    urlcolor={blue}
}
%\setcounter{section}{-1}

\newcommand{\ints}{\mathbb{Z}} 
\newcommand{\reals}{\mathbb{R}} 
\newcommand{\rats}{\mathbb{Q}} 
\newcommand{\comps}{\mathbb{C}}
\newcommand{\set}[1]{ \{ #1 \} }
\newcommand{\mult}{\star}
\newcommand{\abs}[1]{| #1 |}
\newcommand{\inv}[1]{ {#1}^{-1} }
\newcommand{\id}{ \bm{1} }
\newcommand{\iso}{ \equiv }
\newcommand{\comp}{ \circ }
\newcommand{\Aut}{ \textrm{Aut}, }
\newcommand{\Stab}{ \textrm{Stab}, }
\newcommand{\sheaf}{ \mathcal{O} }
\newcommand{\md}{\,\text{mod}\,}
\newcommand{\floor}[1]{\lfloor #1 \rfloor}
\newcommand{\ceil}[1]{\lceil #1 \rceil}
\newcommand{\cent}{Z}
\newcommand{\central}{C}
\newcommand{\nor}{N}
\renewcommand{\bar}{\overline}
\newcommand{\norm}[1]{|#1|}
\newcommand{\divides}{\vert}
\DeclareMathOperator*{\argmin}{\arg\!\min}

\newtheorem{theorem}{Theorem}[section]
\newtheorem{corollary}{Corollary}[theorem]
\newtheorem{lemma}[theorem]{Lemma} 

\setlength\parindent{0pt}
%\allowdisplaybreaks

\title{TITLE}
\begin{document}
\setcounter{tocdepth}{2}
\tableofcontents 
\setcounter{section}{1} 
\section{Subgroups}  

\subsection{Definition and Examples}
Throughout these exercises, I will denote subgroups by $H$ unless otherwise specified.
\subsubsection{}\label{ex1p1}
\begin{enumerate}[(a)]
\item Let $a+ai,b+bi \in H$. Then $a+ai-(b+bi) = (a-b) + (a-b)i \in H$
\item Let $z,y \in H$. Then $|z\inv{y}| = |z||\inv{y}| = |z|\inv{|y|} = 1\cdot \inv{1} = 1$, so $z\inv{y} \in H$
\item Let $\frac{a}{b}, \frac{c}{d} \in H$. So $b,d|n$, so $xb=n,yd=n$ for ints $x,y$. Let $g = (b,d)$, and $l=(b,d)$.\\
(Recall from Chapter 0 that $gl = bd$, and also $g=rb+sd$ for some ints $r,s$).\\
Then $\frac{a}{b}-\frac{c}{d} = \frac{ad-bc}{bd} = \frac{\frac{1}{g}(ad-bc)}{\frac{1}{g}bd} = \frac{\frac{1}{g}(ad-bc)}{l}$. (Note that the numerator is an integer, since $g|ad-bc$).\\
We need to show that $l|n$.\\
Note that $m=\frac{gy}{b}$ is an integer, because
\begin{align*}
\frac{gy}{b} &= \frac{rby+sdy}{b}\\
&= ry + \frac{sdy}{b}\\
&= ry + \frac{sn}{b}\\
&= ry + sx
\end{align*}
and note that $l=\frac{bd}{g}$, so
\begin{align*}
lm &= \frac{bd}{g}\cdot\frac{gy}{b}\\
&= yd\\
&= n
\end{align*}
Hence $l|n$ as needed.
\item Let $\frac{a}{b}, \frac{c}{d} \in H$, so $(b,n),(d,n)=1$.\\
Suppose $(bd,n) > 1$, so there is an integer $s>1$ such that $s|bd$ and $s|n$. Let $p$ be a prime factor of $s$, so we have $p|n$ and $p|bd$. The latter implies that $p|b$ or $p|d$. Assume without loss of generality the former. Then $p|n$ and $p|d$, but $(b,n)=1$, a contradiction. Hence $(bd,n)=1$, and $\frac{a}{b}-\frac{c}{d} = \frac{ad-bc}{bd} \in H$.
\item Let $x,y \in H$. Then $x^2=\frac{a}{b}, y^2=\frac{c}{d}$, for $a,b,c,d \in \ints$. Then $(x\inv{y})^2 = x^2y^{-2}=\frac{ad}{bc}$. So $x\inv{y}\in H$.
\end{enumerate}
\subsubsection{}\label{ex1p2}
We will show that closure is not satisfied in all these exercises
\begin{enumerate}[(a)]
\item $(3\ 1)\comp(1\ 2) = (1\ 2\ 3) \notin H$ \label{ex1p2pa}
\newcommand{\lereflect}{r^{\ceil{\frac{n}{2}}}s}
\item $\lereflect$ is a reflection (across the line passing through the second vertex and origin), but $(\lereflect)s = r^{\ceil{\frac{n}{2}}}$ which is a rotation \label{ex1p2pb}
\item Since $n$ is composite, $n=ab$ for $0<a,b<n$.\\
Let $x\in G$ with $|x|=n$. If $H$ is a subgroup, note that $x$ times itself $a$ times must be in $H$ by closure, i.e. $x^a \in H$. But $(x^a)^b = x^n = 1$, so $|x^a| \leq b < n$, so $x^a \notin H$ by definition, a contradiction.
\item $1+1=2$
\item Note that $\sqrt{2},\sqrt{3}\in H$. But $(\sqrt{2}+\sqrt{3})^2  = 5 + 2\sqrt{6}$ which is irrational, so $\sqrt{2}+\sqrt{3} \notin H$.
\end{enumerate}
\subsubsection{}\label{ex1p3}
\begin{enumerate}[(a)]
\item Each element is its own inverse, so we verify the operation is closed on the rest of the elements:
\begin{align*}
(r^2)(s) &= sr^{-2} = sr^2 \in H\\
(r^2)(sr^2) &= sr^{-2}r^2 = s \in H\\
(s)(r^2) &= sr^2 \in H\\
(s)(sr^2) &= s^2r^2 = r^2 \in H\\
(sr^2)(r^2) &= sr^4 = s \in H\\
(sr^2)(s) &= ssr^{-2} =s^2r^2 = r^2 \in H
\end{align*}
\item Again, each element is its own inverse, so we verify closure:
\begin{align*}
(r^2)(sr) &= sr^{-2}r = s\inv{r} = sr^3 \in H\\
(r^2)(sr^3) &= sr^{-2}r^3 = sr \in H
(sr)(sr^3) &= srsr^3 = s^2\inv{r}r^3 = s^2r^2 = r^2 \in H\\
(sr)(r^2) &= sr^3 \in H\\
(sr^3)(r^2) &= sr^5 = sr \in H\\
(sr^3)(sr) &= sr^3sr = s^2r^{-3}r = r^{-2} = r^2 \in H
\end{align*}
\end{enumerate}

\subsubsection{}\label{ex1p4}
Let $G=\ints$, and $H$ be the positive even integers.

\subsubsection{}\label{ex1p5}
Let $G=\set{a_1,\ldots,a_2,\ldots,a_n=1}$ (the elements listed are distinct), and assume without loss of generality that\\
$H = \set{a_2,\ldots,a_n=1}$.\\
Note that we can't have $a_1a_2=a_1$ or $a_1a_2=a_2$, or else by cancelation we will have $a_2=1$ or $a_1=1$ respectively, both contradictions.\\
So assume without loss of generality that $a_1a_2=a_3$ 
(it's okay if $n=3$ and $a_3=1$), yielding
\begin{equation}
a_1 = a_3\inv{a_2}
\end{equation}
\begin{itemize}
\item Case: $\inv{a_2} = a_1$.\\
Then $a_2\in H$, but $\inv{a_2} \notin H$, so inverses are not in $H$
\item Case: $\inv{a_2} \neq a_1$\\
Then $\inv{a_2} \in H$. But $a_3\inv{a_2} = a_1 \notin H$, so closure is not satisfied.
\end{itemize}
\subsubsection{}\label{ex1p6}
Let $G$ be abelian and $g,h \in H$. Then $|g|=n,|h|=m$, with $n,m<\infty$.\\
Then $(g\inv{h})^{nm} = (g^n)^m(h^m)^{-n} = 1$. (The first equality follows from the fact that $G$ is abelian). Hence, $|g\inv{h}| \leq nm \leq \infty$.\\
Now suppose $H=S_{\infty}$ a non-abelian group. Consider the permutations
\begin{equation}
\sigma = (1\ 2)(3\ 4)(5\ 6)\cdots\\
\tau = (2\ 3)(4\ 5)(6\ 7)\cdots
\end{equation}
Individually, they just swap elements, so $|\sigma|,|\tau| = 2$, but $|\tau\comp\sigma| = \infty$. 
\subsubsection{}\label{ex1p7}
The torsion subgroup is clearly $H= 0\times \ints/n\ints$\\
Let $I = (G-H)\cup\set{0}$\\
Then $(2,1),(2,0)\in I$. But $(2,1)-(2,0) = (0,1)$, is not in $I$ (It is a nonzer element in $H$)
\subsubsection{}\label{ex1p8}
\begin{itemize}
\item Only if: Suppose $H\cup K$ is a subgroup\\
Suppose $K\not\subset H$, so there exists $k\in K$ such that $k\notin H$\\
Then let $h \in H$.\\
Then $h,k\in H\cup K$, so $hk\in H\cup K$ (since $H\cup K$ is a subgroup).\\
If $hk \in H$, then $\inv{h}(hk) \in H$, so $k\in H$, a contradiction.\\
So we must have $hk \in K$. But then $(hk)\inv{k} \in K$, so $h\in K$.\\
Since $h\in H$ was arbitrary, $H \subset K$
\item If: Assume without loss of generality that $H \subset K$\\
Let $x,y \in H\cup K$. Then $x,y\in K$ (since $H\subset K$), so $x\inv{y} \in K = H\cup K$
\end{itemize}
\subsubsection{}\label{ex1p9}
Let $A,B\in \textrm{SL}_nF$, so $\det A, \det B =1$. Then $\det A\inv{B} = \det A \det \inv{B} = 1$
\subsubsection{}\label{ex1p10}
\begin{enumerate}[(a)]
\item See next part
\item
Let $\set{H_i}_{i\in I}$ be a collection of subgroups of $G$
and let $H = \bigcap_{i\in I}H_i$.\\
Then let $x,y \in H$. So for arbitrary $i$. $x \in H_i$ If $y\in H_i$ for all $i$, then $\inv{y}\in H_i$. Then $x\inv{y} \in H_i$. Since $i$ was arbitrary $x\inv{y} \in H=\bigcap H_i$
\end{enumerate}
\subsubsection{}\label{ex1p11}
\begin{enumerate}[(a)]
\item Let $(a,1),(k,1)\in H$. Then $(a,1)\inv{(k,1)} = (a,1)(\inv{k},1) = (a\inv{k},1) \in H$
\item Analogous to above
\item Let $(a,a),(b,b) \in H$. Then $(a,a)\inv{(b,b)} = (a,a)(\inv{b},\inv{b}) = (a\inv{b},a\inv{b}) \in H$
\end{enumerate}
\subsubsection{}\label{ex1p12}

\begin{enumerate}[(a)]
\item Let $x,y\in H$, with $x=a^n, y=b^n$. Then $x\inv{y} = a^nb^{-n} = (a\inv{b})^n \in H$ (note we used that $A$ is abelian here)
\item Let $x,y\in H$, so $x^n,y^n=1$. Then $(x\inv{y})^n = x^n\inv{(y^n)} = 1$, so $x\inv{y} \in H$.
\end{enumerate}
\subsubsection{}\label{ex1p13}
\subsubsection{}\label{ex1p14}
\newcommand{\lereflect}{r^{\ceil{\frac{n}{2}}}s}
See \ref{ex1p2pb}. $\lereflect$ is a reflection, but $(\lereflect)s = r^{\ceil{\frac{n}{2}}}$ which does not have order $2$ in general. For the case $n=4$ consider the reflection $sr$ across the line $y=0$, and note that $s(sr) = r$ which has order $4$
\subsubsection{}\label{ex1p15}
Let $H = \bigcup_{i=1}^{\infty}H_i$, and let $x,y\in H$. So $x \in H_i, y\in H_j$ for some $i,j$. Assume without loss of generality that $i \geq j$. Then $y \in H_i$, since $H_j \subset H_i$. So $x\inv{y} \in H_i \subset H$
\subsubsection{}\label{ex1p16} 
The inverse of an upper triangular matrix is upper triangular (because the adjoint is upper triangular), and they are closed under multiplication
\subsubsection{}\label{ex1p17}
The same logic as above, with the additional note that diagonals of $1$s are preserved after matrix multiplication. Diagonals of $1$s are preserved by taking inverses too (the adjoint is the transpose of the cofactor matrix, and the minors along the diagonal are all clearly just $1$)
\subsection{Centralizers and Normalizers. Stabilizers and Kernels}
\subsubsection{}\label{ex2p1}
Follows from 
\begin{align*}
ga\inv{g} &= a\\
\iff ga &= ag\\
iff a &= \inv{g}ag
\end{align*}
we will take these equivalences for granted moving forward
\subsubsection{}\label{ex2p2}
Let $x \in G$. And note that given $g \in Z(G)$, we have $gx=xg$. This is because by definition of $Z(G)$, $x$ commutes with all elements of $G$, including $g$. But this also says that $g$ commutes with all elements of $x$ of $Z(G)$, so $g \in C_G(Z(G))$.\\
Since $g\in G$ was arbitrary, $G \subset C_G(Z(G))$ and since the reverse inclusion trivially holds, $C_G(Z(G)) = G$.\\
We already know from the section that the centralizer is contained in the normalizer, i.e. $G = C_G(Z(G)) \subset N_G(Z(G))$, so again the trivial reverse inclusion yields $G = N_G(Z(G))$.
\subsubsection{}\label{ex2p3}
If $g \in C_G(B)$, then given $a\in A$, we have $a \in B$, so $ga=ag$ holds. Hence $g \in C_G(A)$ (since $a\in A$ was arbitrary). And since $g \in C_G(B)$ was arbitrary, $C_G(B) \leq C_G(A)$.
\subsubsection{}\label{ex2p4}
\begin{itemize}
\item $S_3$\\
Recall that
\begin{equation}
S_3 = \set{1,(1\ 2),(2\ 3),(1\ 3),(1\ 2\ 3),(1\ 3\ 2)}
\end{equation}
of course $|S_3| = 6$, so all the centralizers must divide $6$ (by Lagrange's theorem). \\
Trivially, $C(1) = S_3$\\
Now we note from the example in the text that, where $A=\set{1, (1\ 2)}$,  $C_{S_3}(A) = A$. For now, denote $C=C(1,2)$ for convenience\\
By the previous exercises,
\begin{align*}
\set{(1,2)} &\subset A\\
\implies C(A) &\leq C((1,2))\\
\implies \norm{C(A)} &\vert \norm{C}\\
\implies 2 &\vert \norm{C}
\end{align*}
So we have $2\vert\norm{C}\vert 6$, which means we must have $\norm{C}=2$ or $\norm{C}=6$. But $(1\ 2)$ does not commute with $(1\ 2\ 3)$, so $\norm{C}=2$, and $C(A) \leq C$ implies that $C = C(A)$. I.e. we conclude that $C( (1\ 2) ) = \set{1, (1\ 2)}$. With entirely analogous arguments for the other "swapping" permutations, we obtain
\begin{align*}
C( (1\ 2) ) &= \set{1, (1\ 2)}\\
C( (1\ 3) ) &= \set{1, (1\ 3)}\\
C( (2\ 3) ) &= \set{1, (2\ 3)}\\
\end{align*}
Now denote $C = C( (1\ 2\ 3) )$. We have $1, (1\ 2\ 3) \in C$ automatically. And since $C$ is a subgroup, we must have $\inv{(1\ 2\ 3)} = (1\ 3\ 2) \in C$. Now $\norm{C} \leq 3$, but by $\norm{C} \divides 6$, this narrows it down to $\norm{C} = 3$ or $\norm{C} = 6$. But since $(1\ 2)$ doesn't commute with $(1\ 2\ 3)$, we must have $\norm{C} = 3$. Hence, $C( (1\ 2\ 3) ) = \set{1, (1\ 2\ 3), (1\ 3\ 2)}$. An analagous argument applies for $C( (1\ 3\ 2) )$, yielding the same set. So, in total, we have
\begin{align*}
C(1) &= S_3\\
C( (1\ 2) ) &= \set{1, (1\ 2)}\\
C( (1\ 3) ) &= \set{1, (1\ 3)}\\
C( (2\ 3) ) &= \set{1, (2\ 3)}\\
C( (1\ 2\ 3) ) &= \set{1, (1\ 2\ 3), (1\ 3\ 2)}\\
C( (1\ 3\ 2) ) &= \set{1, (1\ 2\ 3), (1\ 3\ 2)}
\end{align*}
Finally, since all the centralizers are missing elements except for $1$, we have $Z(S_3) = 1$
\item $D_8$\\
We have
\begin{equation}
D_8 = \set{1, r, r^2, r^3, s, sr, sr^2, sr^3}
\end{equation}
So all centralizers must divide $8$.\\
The examples give us the following info. Where $A = \set{1,r,r^2,r^3}$,
\begin{align*}
Z(D_8) &= \set{1, r^2}\\
C(A) &= A
\end{align*}
So the center has already been computed, and of course $C(1) = D_8$.\\
Next, we have
\begin{align*}
\set{r} &\subset A\\
\implies C(A) &\leq C(r)\\
\implies 4 &\divides \norm{C(r)}
\end{align*}
Combined with the fact $\norm{C(r)}\divides 8$, we must have $\norm{C(r)}=4$ or $\norm{C(r)}=8$. However, $s$ and $r$ don't commute ($rs = s\inv{r} \neq sr$), so $s \notin C(r)$ and we must hae $\norm{C(r)} = 4$. Since $C(A) \leq C(r)$, this means that $C(r) = C(A)$.\\
Analagously, $\set{r^3} \subset A$ implies $\norm{C(r^3)}=4$ or $\norm{C(r^3)}=8$, but $s$ and $r^3$ don't commute, yielding $C(r^3) = C(A)$.\\
Next, since $r^2 \in Z(D_8)$, we have $C(r^2) = D_8$. Thus, we have so far concluded that
\begin{align*}
C(1) &= D_8\\
C(r) &= \set{1,r,r^2,r^3}\\
C(r^2) &= D_8\\
C(r^3) &= \set{1,r,r^2,r^3}
\end{align*}
Now consider the element $sr^k$ ($k=0,1,2,3$). Automatically, we have $1,r^2,sr^k\in C(sr^k)$. So $\norm{C(sr^k)} \geq 3$. But $\norm{C(sr^k)} \divides 8$, so we must have $\norm{C(sr^k)} = 4$ or $\norm{C(sr^k)} = 8$.\\
Now note that 
\begin{align*}
(sr^k)r &= sr^{k+1}\\
r(sr^k) &= sr^{k-1}
\end{align*}
Which are equal if and only if
\begin{align*}
sr^{k+1} &= sr^{k-1}\\
\iff r^kr &= r^k\inv{r}\\
\iff r &= \inv{r}
\end{align*}
which is impossible in $D_8$. Hence $r \notin C(sr^k)$, and we must conclude $\norm{C(sr^k)} = 4$. Hence, we can conclude that
\begin{equation}
C(sr^k) = \set{1,r^2,sr^k,x_k}
\end{equation}
where $x_k$ is another element in $D_8$ that commutes with $sr^k$.\\
We note that $sr$ and $sr^3$ are inverses, and hence commute, so that covers the case $k=1,3$. Also, $sr^2$ commutes with $s$, covering the cases $k=0,2$. Hence, we can now list the centralizers for every element in $D_8$:
\begin{align*}
C(1) &= D_8\\
C(r) &= \set{1,r,r^2,r^3}\\
C(r^2) &= D_8\\
C(r^3) &= \set{1,r,r^2,r^3}\\
C(s) &= \set{1,r^2,s,sr^2}\\
C(sr) &= \set{1,r^2,sr,sr^3}\\
C(sr^2) &= \set{1,r^2,s,sr^2}\\
C(sr^3) &= \set{1,r^2,sr,sr^3}
\end{align*}
\item $Q_8$\\
lol
\end{itemize}
\end{document}












