\documentclass[]{article}

\usepackage[letterpaper, margin=0.75in]{geometry}
\usepackage{amsmath}
\usepackage{amssymb}
\usepackage{amsfonts}
\usepackage{xcolor}
\usepackage{hhline}
\usepackage{hyperref}
\usepackage{tikz}

\newcommand{\abs}[1]{\left\vert #1 \right\vert}
\newcommand{\md}{\,\text{mod}\,}
\newcommand{\bbz}{\mathbb{Z}}
\newcommand{\bbq}{\mathbb{Q}}
\newcommand{\bbr}{\mathbb{R}}
\newcommand{\bbc}{\mathbb{C}}
\newcommand{\bbn}{\mathbb{N}}
\newcommand{\bbf}{\mathbb{F}}
\newcommand{\im}{\text{im\,}}

\usepackage{pgffor}
\newcommand*{\cycle}[1]{( \foreach \entry [count=\i] in {#1} {\ifnum\i>1\ \fi\entry})}

\setlength\parindent{0pt}
%\allowdisplaybreaks

\usepackage{graphicx}
\usepackage{listings}

\graphicspath{ {./py/img/} }

\definecolor{codegreen}{rgb}{0,0.6,0}
\definecolor{codegray}{rgb}{0.5,0.5,0.5}
\definecolor{codepurple}{rgb}{0.58,0,0.82}
\definecolor{backcolour}{rgb}{0.95,0.95,0.92}

\lstdefinestyle{mystyle}{
	backgroundcolor=\color{backcolour},   
	commentstyle=\color{codegreen},
	keywordstyle=\color{orange},
	numberstyle=\tiny\color{codegray},
	stringstyle=\color{codepurple},
	basicstyle=\ttfamily\footnotesize,
	breakatwhitespace=false,         
	breaklines=true,                 
	captionpos=b,
	keepspaces=true,                 
	numbers=left,                    
	numbersep=5pt,                  
	showspaces=false,                
	showstringspaces=false,
	showtabs=false,                  
	tabsize=2
}

\lstset{style=mystyle}



\title{Subgroups}
\author{}
\date{}
\begin{document}

\maketitle
\vspace{-5em}

\tableofcontents

\addcontentsline{toc}{section}{2.1: Definition and Examples}
\section*{\underline{2.1: Definition and Examples}}
\addcontentsline{toc}{subsection}{Exercises}
\begin{enumerate}

\item For the sake of this problem I'll be explicit with group notation. For the proposed subgroup $(H,\cdot) \leq (G,\star)$, we want to show that $H \subset G$ and $\cdot = \star$. Then, if $H$ is finite, we want to show that for all $x,y\in H$, $x\cdot y \in H$. If $H$ is not finite we want to show that $x\cdot y^{-1} \in H$.
\begin{enumerate}
\item Here $H = \left( \{a(1+i)\mid a\in\bbr\},\ + \right)$ which is not finite, and $G = \left(\bbc,\ +\right)$. Clearly every element of $H$ is a complex number so $H \subset G$, and the two have the same operation. The identity is $0$, so the inverse of $a(1+i)$ is $-a(1+i)$. For two arbitrary elements $x,y\in H$,
\begin{equation}
x\cdot y^{-1} = a_x(1+i) + (-a_y)(1+i) = (a_x-a_y)(1+i) \in H
\end{equation}
so $H \leq G$.

\item Here $H = \left( \left\{z\mid z\in\bbc,\, z^*z=1\right\},\ \cdot \right)$ and $G = \left(\bbc,\ \cdot\right)$. Again trivially $H \subset G$ and the operation is the same. The identity is $1$ so $z^{-1} = \frac{1}{z}$. The subgroup $H$ is infinite, so for $z_1,z_2\in H$, $z_1\cdot z_2^{-1} = \frac{z_1}{z_2}$ and
\begin{equation}
\left( \frac{z_1}{z_2} \right)^* \left( \frac{z_1}{z_2} \right) = \left( \frac{z_1^*}{z_2^*} \right)\left( \frac{z_1}{z_2} \right) = \frac{z_1^*z_1}{z_2^*z_2} = \frac{1}{1} = 1
\end{equation}
so $z_1\cdot z_2^{-1} \in H$ and $H \leq G$.

\item For a fixed $n$, if $q = \frac{a}{b} \in \bbq$ with $b\vert n$, then $qn\in \bbz$. This provides an alternate characterization for $H$: $H = ( \{q\mid q\in\bbq ,\,qn\in\bbz \},\ + )$ and $G = (\bbq, +)$. Again $H \subset G$ and the operation being the same are trivial. The identity of $G$ is $0$, so $q^{-1} = -q$. For $q_1, q_2\in H$, $q_1\cdot q_2^{-1} = q_1-q_2$, and clearly this is also an integer when multiplied by $n$, making $H \leq G$.

\item For a fixed $n$, $H = \left( \left\{ \frac{a}{b}\mid \frac{a}{b}\in\bbq,\, (b,n)=1 \right\}, + \right)$ and $G = (\bbq, +)$ again. The subset and operation are trivial, and the identity is $0$ again with more explicit inverse $\left( \frac{a}{b} \right)^{-1} = \frac{-a}{b}$. For $\frac{a_1}{b_1},\frac{a_2}{b_2}\in H$,
\begin{equation}
\frac{a_1}{b_1}\cdot \left( \frac{a_2}{b_2} \right)^{-1} = \frac{a_1}{b_2}+ \frac{-a_2}{b_2} = \frac{a_1b_2-a_2b_1}{b_1b_2}\ .
\end{equation}
If $(b_1,n) = (b_2,n) = 1$, then $(b_1b_2,n) = 1$, and any cancellation of common factors in the numerator and denominator won't change that since none of the factors of $b_1b_2$ are factors of $n$. Therefore $H \leq G$.

\item Here $H = ( \{ x\mid x\in\bbr,\, x^2\in\bbq \}, \cdot)$ with $G = (\bbq,\cdot)$. The subset and operation are trivial and the identity is $1$ so $x^{-1} = \frac{1}{x}$. For $x,y\in H$, $x\cdot y^{-1} = \frac{x}{y}$, so 
\begin{equation}
(x\cdot y^{-1})^2 = \left(\frac{x}{y}\right)^2 = \frac{x^2}{y^2} \in \bbq
\end{equation}
i.e. $H\leq G$.

\end{enumerate}


\item Same as above, but hopefully faster and more interesting.
\begin{enumerate}
\item The identity of $S_n$ is not a 2-cycle.
\item {\color{red} what}
\item 
\item The set isn't closed under the operation since $\text{odd} + \text{odd} = \text{even}$.
\item Again, the set isn't closed under the operation, e.g. $(\sqrt{2} + \sqrt{3})^2 = 5 + 2\sqrt{6}$.
\end{enumerate}


\item Both subsets are obviously finite subsets of $D_8$ with the same group operation, so we just want to check that they are closed under composition. Referring to the solution of exercise 1.5.2 for the multiplication table for $D_8$ confirms this for both subsets.


\item Consider $(\bbn,+)$ as a possible subgroup of $(\bbz,+)$. $\bbn$ is infinite and closed under the group operation since the sum of two positive numbers is positive. However, the additive identity $0$ is not in $\bbn$ (sorry if you disagree), and on top of that $(\bbn,+)$ is not closed under inverses (hopefully you agree with this one).


\item Suppose $\abs{G} = n$ and $H \leq G$ with $\abs{H} = n-1$. Denote the single element of $G-H$ with $g$, and all other elements of $G$ are in $H$. Then $g^{-1} \in H$, but $\left( g^{-1} \right)^{-1} = g$ is not in $H$, so $H$ is not closed under inverses.


\item The torsion subgroup $H$ is potentially infinite so we need to show that for all $g,h,\in H$, $gh^{-1}\in H$. By exercise 1.1.22, if $g,h^{-1} \in G$ and $G$ is abelian, then $(gh^{-1})^n = g^n(h^{-1})^n$ so $\abs{gh^{-1}} = \text{lcm}(\abs{g},\abs{h^{-1}})$. By exercise 1.1.20, $\abs{h^{-1}} = \abs{h}$, so if $g$ and $h$ both have finite order then so does $gh$. {\color{red}counterexample for nonabelian?}


\item For $(x,y) \in \bbz \times (\bbz/n\bbz)$, if $\abs{(x,y)} = k \geq 1$, then $kx = 0$ and $ky = 0$, where the second equation is modulo $n$ but the first one isn't. The first equation requires $x = 0$, while the second always has a solution e.g. $k = \frac{\text{lcm}(y,n)}{n}$. Therefore the torsion subgroup is $\{0\} \times (\bbz/n\bbz)$. \\

The group of elements of infinite order is not a subgroup because it's not closed under addition. An element $(x,y)$ has infinite order if and only if $x\neq 0$, then $(x,y) + (-x,z) = (0,y-z)$ for any $y,z$, so we have added two elements of infinite order together to get one with finite order.


\item We are given $H\leq G$ and $K\leq G$. First assume $H\cup K \leq G$. If either one of $H$ or $K$ is the trivial subgroup then $H \subseteq K$ or $K\subseteq H$ trivially since every subgroup contains the identity. Otherwise, let $h \in H$ and $k \in K$. For $H\cup K$ to be a subgroup, the product $x = hk$ must still be in the subgroup, so it must either be in $H$ or $K$ (or both). If it is in $H$, then $h^{-1}x = k$ must also be in $H$. Since $k$ was arbitrary we see that all elements of $K$ are in $H$, so $K \subseteq H$. Likewise, if $x \in K$ then $xk^{-1} = h \in K$ so $H \subseteq K$. \\

In the other direction, if $H\subseteq K$ or $K\subseteq H$ then $H\cup K$ is just equal to the larger subgroup, so it is trivially a subgroup of $G$.


\item The structure is inherited from $GL_n(F)$ so we just need to show that, for all $X,Y \in SL_n(F)$, $M = XY^{-1} \in SL_n(F)$. This is easily seen with $\det Y^{-1} = \frac{1}{\det Y} = 1$ and $\det M = \det X \det Y^{-1} = 1$.


\item Again the subset is trivial and the group operation is the same.
\begin{enumerate}
\item If $H\leq G$ and $K\leq G$, and $x,y \in H\cap K$, then $xy^{-1}\in H$ and $xy^{-1}\in K$ so $xy^{-1} \in H\cap K$.
\item If you assume the collection is countable you can use the above proof and $(A\cap B)\cap C = A\cap B\cap C$. {\color{red} uncountable??}
\end{enumerate}


\item Again again the subset is trivial and the group operation is the same. Denote each subgroup as $H$. We use 1.1.28.c which states that $(a,b)^{-1} = (a^{-1},b^{-1})$.
\begin{enumerate}
\item For $(a_1,1), (a_2,1) \in H$, 
\begin{equation}
(a_1,1)\cdot (a_2,1)^{-1} = (a_1,1)\cdot (a_2^{-1},1^{-1}) = (a_1\cdot a_2^{-1},1\cdot 1) = (a_1a_2^{-1},1) \in H\ .
\end{equation}
\item For $(1,b_1), (1,b_2) \in H$, 
\begin{equation}
(1,b_1)\cdot (1,b_2)^{-1} = (1,b_1)\cdot (1^{-1},b_2^{-2}) = (1\cdot 1,b_1\cdot b_2^{-1}) = (1,b_1b_2^{-1}) \in H\ .
\end{equation}
\item For $(a_1,a_1), (a_2,a_2) \in H$,
\begin{equation}
(a_1,a_1)\cdot (a_2,a_2)^{-1} = (a_1,a_1)\cdot (a_2^{-1},a_2^{-1}) = (a_1a_2^{-1},a_1a_2^{-1}) \in H\ .
\end{equation}
\end{enumerate}


\item Again again again all elements are trivially in $A$ and the group operation is inherited. We also use exercise 1.1.20 again like we did in exercise 6. Let $H$ denote the subgroup.
\begin{enumerate}
\item For $a^n,b^n \in H$, $a^n(b^n)^{-1} = a^nb^{-n} = (ab^{-1})^n$ where $ab^{-1}\in A$ as required. We use 1.1.24 here. 
\item For $a,b \in H$, $(ab^{-1})^n = a^n(b^{-1})^n$. Since $\abs{a} = n$ and $\abs{b} = \abs{b^{-1}} = n$, we see that $\abs{ab^{-1}} = \text{lcm}(\abs{a},\abs{b}) = n$.
\end{enumerate}


\item {\color{red} hard}


\item From 1.2.3 we know that every element of $D_{2n}$ of the form $sr^k$ (here $k \in [0,n-1]$ but we can just take it as any integer since $r^n = 1$) has order 2. Choose $a,b$ such that $sr^a$ and $sr^b$ are distinct (so $b-a \neq 0 \md n$). Then
\begin{equation}
(sr^a)(sr^b) = (r^{-a}s)(sr^b) = r^{-a}s^2r^b = r^{b-a} \neq 1
\end{equation}
which means that this set of elements is not closed under composition and cannot be a subgroup.


\item {\color{red} the inductive proof is obvious but that's just an arbitrarily large finite union, I don't know what it means to extend it to a countably infinite chain of subgroups without a concrete example that I can't think of}


\item The inverse of an upper triangular matrix is upper triangular; the product of two upper triangular matrices is upper triangular, so for triangular $X,Y \in GL_n(F)$, the product $XY^{-1}$ is upper triangular as well.


\item extreme copout: heisenberg group from 1.4.11


\end{enumerate}



\end{document}