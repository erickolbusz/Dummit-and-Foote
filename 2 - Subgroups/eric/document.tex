\documentclass[]{article}

\usepackage[letterpaper, margin=0.75in]{geometry}
\usepackage{amsmath}
\usepackage{amssymb}
\usepackage{amsfonts}
\usepackage{xcolor}
\usepackage{hhline}
\usepackage{hyperref}
\usepackage{tikz}

\newcommand{\abs}[1]{\left\vert #1 \right\vert}
\newcommand{\gen}[1]{\left\langle\, #1 \,\right\rangle}
\newcommand{\md}{\,\text{mod}\,}
\newcommand{\bbz}{\mathbb{Z}}
\newcommand{\bbq}{\mathbb{Q}}
\newcommand{\bbr}{\mathbb{R}}
\newcommand{\bbc}{\mathbb{C}}
\newcommand{\bbn}{\mathbb{N}}
\newcommand{\bbf}{\mathbb{F}}
\newcommand{\im}{\text{im\,}}

\usepackage{pgffor}
\newcommand*{\cycle}[1]{( \foreach \entry [count=\i] in {#1} {\ifnum\i>1\ \fi\entry})}

\setlength\parindent{0pt}
%\allowdisplaybreaks

\usepackage{graphicx}
\usepackage{listings}

\graphicspath{ {./py/img/} }

\definecolor{codegreen}{rgb}{0,0.6,0}
\definecolor{codegray}{rgb}{0.5,0.5,0.5}
\definecolor{codepurple}{rgb}{0.58,0,0.82}
\definecolor{backcolour}{rgb}{0.95,0.95,0.92}

\lstdefinestyle{mystyle}{
	backgroundcolor=\color{backcolour},   
	commentstyle=\color{codegreen},
	keywordstyle=\color{orange},
	numberstyle=\tiny\color{codegray},
	stringstyle=\color{codepurple},
	basicstyle=\ttfamily\footnotesize,
	breakatwhitespace=false,         
	breaklines=true,                 
	captionpos=b,
	keepspaces=true,                 
	numbers=left,                    
	numbersep=5pt,                  
	showspaces=false,                
	showstringspaces=false,
	showtabs=false,                  
	tabsize=2
}

\lstset{style=mystyle}



\title{Subgroups}
\author{}
\date{}
\begin{document}

\maketitle
\vspace{-5em}

\tableofcontents

\addcontentsline{toc}{section}{2.1: Definition and Examples}
\section*{\underline{2.1: Definition and Examples}}
\addcontentsline{toc}{subsection}{Exercises}
\subsection*{\underline{Exercises}}
\begin{enumerate}

\item For the sake of this problem I'll be explicit with group notation. For the proposed subgroup $(H,\cdot) \leq (G,\star)$, we want to show that $H \subset G$ and $\cdot = \star$. Then, if $H$ is finite, we want to show that for all $x,y\in H$, $x\cdot y \in H$. If $H$ is not finite we want to show that $x\cdot y^{-1} \in H$.
\begin{enumerate}
\item Here $H = \left( \{a(1+i)\mid a\in\bbr\},\ + \right)$ which is not finite, and $G = \left(\bbc,\ +\right)$. Clearly every element of $H$ is a complex number so $H \subset G$, and the two have the same operation. The identity is $0$, so the inverse of $a(1+i)$ is $-a(1+i)$. For two arbitrary elements $x,y\in H$,
\begin{equation}
x\cdot y^{-1} = a_x(1+i) + (-a_y)(1+i) = (a_x-a_y)(1+i) \in H
\end{equation}
so $H \leq G$.

\item Here $H = \left( \left\{z\mid z\in\bbc,\, z^*z=1\right\},\ \cdot \right)$ and $G = \left(\bbc,\ \cdot\right)$. Again trivially $H \subset G$ and the operation is the same. The identity is $1$ so $z^{-1} = \frac{1}{z}$. The subgroup $H$ is infinite, so for $z_1,z_2\in H$, $z_1\cdot z_2^{-1} = \frac{z_1}{z_2}$ and
\begin{equation}
\left( \frac{z_1}{z_2} \right)^* \left( \frac{z_1}{z_2} \right) = \left( \frac{z_1^*}{z_2^*} \right)\left( \frac{z_1}{z_2} \right) = \frac{z_1^*z_1}{z_2^*z_2} = \frac{1}{1} = 1
\end{equation}
so $z_1\cdot z_2^{-1} \in H$ and $H \leq G$.

\item For a fixed $n$, if $q = \frac{a}{b} \in \bbq$ with $b\vert n$, then $qn\in \bbz$. This provides an alternate characterization for $H$: $H = ( \{q\mid q\in\bbq ,\,qn\in\bbz \},\ + )$ and $G = (\bbq, +)$. Again $H \subset G$ and the operation being the same are trivial. The identity of $G$ is $0$, so $q^{-1} = -q$. For $q_1, q_2\in H$, $q_1\cdot q_2^{-1} = q_1-q_2$, and clearly this is also an integer when multiplied by $n$, making $H \leq G$.

\item For a fixed $n$, $H = \left( \left\{ \frac{a}{b}\mid \frac{a}{b}\in\bbq,\, (b,n)=1 \right\}, + \right)$ and $G = (\bbq, +)$ again. The subset and operation are trivial, and the identity is $0$ again with more explicit inverse $\left( \frac{a}{b} \right)^{-1} = \frac{-a}{b}$. For $\frac{a_1}{b_1},\frac{a_2}{b_2}\in H$,
\begin{equation}
\frac{a_1}{b_1}\cdot \left( \frac{a_2}{b_2} \right)^{-1} = \frac{a_1}{b_2}+ \frac{-a_2}{b_2} = \frac{a_1b_2-a_2b_1}{b_1b_2}\ .
\end{equation}
If $(b_1,n) = (b_2,n) = 1$, then $(b_1b_2,n) = 1$, and any cancellation of common factors in the numerator and denominator won't change that since none of the factors of $b_1b_2$ are factors of $n$. Therefore $H \leq G$.

\item Here $H = ( \{ x\mid x\in\bbr,\, x^2\in\bbq \}, \cdot)$ with $G = (\bbq,\cdot)$. The subset and operation are trivial and the identity is $1$ so $x^{-1} = \frac{1}{x}$. For $x,y\in H$, $x\cdot y^{-1} = \frac{x}{y}$, so 
\begin{equation}
(x\cdot y^{-1})^2 = \left(\frac{x}{y}\right)^2 = \frac{x^2}{y^2} \in \bbq
\end{equation}
i.e. $H\leq G$.

\end{enumerate}


\item Same as above, but hopefully faster and more interesting.
\begin{enumerate}
\item The identity of $S_n$ is not a 2-cycle.
\item {\color{red} what}
\item 
\item The set isn't closed under the operation since $\text{odd} + \text{odd} = \text{even}$.
\item Again, the set isn't closed under the operation, e.g. $(\sqrt{2} + \sqrt{3})^2 = 5 + 2\sqrt{6}$.
\end{enumerate}


\item Both subsets are obviously finite subsets of $D_8$ with the same group operation, so we just want to check that they are closed under composition. Referring to the solution of exercise 1.5.2 for the multiplication table for $D_8$ confirms this for both subsets.


\item Consider $(\bbn,+)$ as a possible subgroup of $(\bbz,+)$. $\bbn$ is infinite and closed under the group operation since the sum of two positive numbers is positive. However, the additive identity $0$ is not in $\bbn$ (sorry if you disagree), and on top of that $(\bbn,+)$ is not closed under inverses (hopefully you agree with this one).


\item Suppose $\abs{G} = n$ and $H \leq G$ with $\abs{H} = n-1$. Denote the single element of $G-H$ with $g$, and all other elements of $G$ are in $H$. Then $g^{-1} \in H$, but $\left( g^{-1} \right)^{-1} = g$ is not in $H$, so $H$ is not closed under inverses.


\item The torsion subgroup $H$ is potentially infinite so we need to show that for all $g,h,\in H$, $gh^{-1}\in H$. By exercise 1.1.22, if $g,h^{-1} \in G$ and $G$ is abelian, then $(gh^{-1})^n = g^n(h^{-1})^n$ so $\abs{gh^{-1}} = \text{lcm}(\abs{g},\abs{h^{-1}})$. By exercise 1.1.20, $\abs{h^{-1}} = \abs{h}$, so if $g$ and $h$ both have finite order then so does $gh$. {\color{red}counterexample for nonabelian?}


\item For $(x,y) \in \bbz \times (\bbz/n\bbz)$, if $\abs{(x,y)} = k \geq 1$, then $kx = 0$ and $ky = 0$, where the second equation is modulo $n$ but the first one isn't. The first equation requires $x = 0$, while the second always has a solution e.g. $k = \frac{\text{lcm}(y,n)}{n}$. Therefore the torsion subgroup is $\{0\} \times (\bbz/n\bbz)$. \\

The group of elements of infinite order is not a subgroup because it's not closed under addition. An element $(x,y)$ has infinite order if and only if $x\neq 0$, then $(x,y) + (-x,z) = (0,y-z)$ for any $y,z$, so we have added two elements of infinite order together to get one with finite order.


\item We are given $H\leq G$ and $K\leq G$. First assume $H\cup K \leq G$. If either one of $H$ or $K$ is the trivial subgroup then $H \subseteq K$ or $K\subseteq H$ trivially since every subgroup contains the identity. Otherwise, let $h \in H$ and $k \in K$. For $H\cup K$ to be a subgroup, the product $x = hk$ must still be in the subgroup, so it must either be in $H$ or $K$ (or both). If it is in $H$, then $h^{-1}x = k$ must also be in $H$. Since $k$ was arbitrary we see that all elements of $K$ are in $H$, so $K \subseteq H$. Likewise, if $x \in K$ then $xk^{-1} = h \in K$ so $H \subseteq K$. \\

In the other direction, if $H\subseteq K$ or $K\subseteq H$ then $H\cup K$ is just equal to the larger subgroup, so it is trivially a subgroup of $G$.


\item The structure is inherited from $GL_n(F)$ so we just need to show that, for all $X,Y \in SL_n(F)$, $M = XY^{-1} \in SL_n(F)$. This is easily seen with $\det Y^{-1} = \frac{1}{\det Y} = 1$ and $\det M = \det X \det Y^{-1} = 1$.


\item Again the subset is trivial and the group operation is the same.
\begin{enumerate}
\item If $H\leq G$ and $K\leq G$, and $x,y \in H\cap K$, then $xy^{-1}\in H$ and $xy^{-1}\in K$ so $xy^{-1} \in H\cap K$.
\item If you assume the collection is countable you can use the above proof and $(A\cap B)\cap C = A\cap B\cap C$. {\color{red} uncountable??}
\end{enumerate}


\item Again again the subset is trivial and the group operation is the same. Denote each subgroup as $H$. We use 1.1.28.c which states that $(a,b)^{-1} = (a^{-1},b^{-1})$.
\begin{enumerate}
\item For $(a_1,1), (a_2,1) \in H$, 
\begin{equation}
(a_1,1)\cdot (a_2,1)^{-1} = (a_1,1)\cdot (a_2^{-1},1^{-1}) = (a_1\cdot a_2^{-1},1\cdot 1) = (a_1a_2^{-1},1) \in H\ .
\end{equation}
\item For $(1,b_1), (1,b_2) \in H$, 
\begin{equation}
(1,b_1)\cdot (1,b_2)^{-1} = (1,b_1)\cdot (1^{-1},b_2^{-2}) = (1\cdot 1,b_1\cdot b_2^{-1}) = (1,b_1b_2^{-1}) \in H\ .
\end{equation}
\item For $(a_1,a_1), (a_2,a_2) \in H$,
\begin{equation}
(a_1,a_1)\cdot (a_2,a_2)^{-1} = (a_1,a_1)\cdot (a_2^{-1},a_2^{-1}) = (a_1a_2^{-1},a_1a_2^{-1}) \in H\ .
\end{equation}
\end{enumerate}


\item Again again again all elements are trivially in $A$ and the group operation is inherited. We also use exercise 1.1.20 again like we did in exercise 6. Let $H$ denote the subgroup.
\begin{enumerate}
\item For $a^n,b^n \in H$, $a^n(b^n)^{-1} = a^nb^{-n} = (ab^{-1})^n$ where $ab^{-1}\in A$ as required. We use 1.1.24 here. 
\item For $a,b \in H$, $(ab^{-1})^n = a^n(b^{-1})^n$. Since $\abs{a} = n$ and $\abs{b} = \abs{b^{-1}} = n$, we see that $\abs{ab^{-1}} = \text{lcm}(\abs{a},\abs{b}) = n$.
\end{enumerate}


\item {\color{red} hard}


\item From 1.2.3 we know that every element of $D_{2n}$ of the form $sr^k$ (here $k \in [0,n-1]$ but we can just take it as any integer since $r^n = 1$) has order 2. Choose $a,b$ such that $sr^a$ and $sr^b$ are distinct (so $b-a \neq 0 \md n$). Then
\begin{equation}
(sr^a)(sr^b) = (r^{-a}s)(sr^b) = r^{-a}s^2r^b = r^{b-a} \neq 1
\end{equation}
which means that this set of elements is not closed under composition and cannot be a subgroup.


\item {\color{red} the inductive proof is obvious but that's just an arbitrarily large finite union, I don't know what it means to extend it to a countably infinite chain of subgroups without a concrete example that I can't think of}


\item The inverse of an upper triangular matrix is upper triangular; the product of two upper triangular matrices is upper triangular, so for triangular $X,Y \in GL_n(F)$, the product $XY^{-1}$ is upper triangular as well.


\item extreme copout: heisenberg group from 1.4.11


\end{enumerate}











\addcontentsline{toc}{section}{2.2: Centralizers and Normalizers, Stabilizers and Kernels}
\section*{\underline{2.2: Centralizers and Normalizers, Stabilizers and Kernels}}
\addcontentsline{toc}{subsection}{Exercises}
\subsection*{\underline{Exercises}}
\begin{enumerate}

\item The definition is $C_G(A) = \{ g\in G\mid gag^{-1} = a\text{ for all }a\in A\}$. If $gag^{-1} = a$, then, left-multiplying $g^{-1}$ and right-multiplying $g$, we equivalently have $a = g^{-1}ag$.


\item Choose an arbitrary $x\in G$. By definition, $x$ commutes with every element of $Z(G)$, so $xg = gx$ for all $g \in Z(G)$ and $x \in C_G(Z(G))$. Since $C_G(A) \leq N_G(A)$ for any $A$, in this case we also find that $N_G(Z(G)) = G$.


\item For $A \subseteq B$, every element in $C_G(B)$ must commute with every element of $A$, so it must also be in $C_G(A)$, i.e. $C_G(B) \subseteq C_G(A)$. Both are subgroups of $G$ so $C_G(B) \leq C_G(A)$.


\item I'm just going to be reading off of the multiplication tables in exercise 1.5.2 for this. Lagrange's Theorem doesn't help because I already did all the manual labor for this. For $S_3$, the list of elements that commute are
\begin{enumerate}
\item[$1$]: all elements
\item[$\cycle{1,2}$]: 1, $\cycle{1,2}$
\item[$\cycle{1,3}$]: 1, $\cycle{1,3}$
\item[$\cycle{2,3}$]: 1, $\cycle{2,3}$
\item[$\cycle{1,2,3}$]: 1, $\cycle{1,2,3}$, $\cycle{1,3,2}$
\item[$\cycle{1,3,2}$]: 1, $\cycle{1,2,3}$, $\cycle{1,3,2}$
\end{enumerate}
so $Z(S_3) = \{1\}$. For $D_8$, 
\begin{enumerate}
\item[$1$]: all elements
\item[$r$]: 1, $r$, $r^2$, $r^3$
\item[$r^2$]: all elements
\item[$r^3$]: 1, $r$, $r^2$, $r^3$
\item[$s$]: 1, $r^2$, $s$
\item[$sr$]: 1, $r^2$, $sr$, $sr^3$
\item[$sr^2$]: 1, $r^2$, $s$, $sr^2$
\item[$sr^3$]: 1, $r^2$, $sr$, $sr^3$
\end{enumerate}
so $Z(D_8) = \{1,r^2\}$. For $Q_8$,
\begin{enumerate}
\item[$1$]: all elements
\item[$-1$]: all elements
\item[$i$]: 1, $-1$, $i$, $-i$
\item[$-i$]: 1, $-1$, $i$, $-i$
\item[$j$]: 1, $-1$, $j$, $-j$
\item[$-j$]: 1, $-1$, $j$, $-j$
\item[$k$]: 1, $-1$, $k$, $-k$
\item[$-k$]: 1, $-1$, $k$, $-k$
\end{enumerate}
so $Z(Q_8) = \{1,-1\}$.


\item More reusing previous results. 
\begin{enumerate}
\item The set of elements of $S_3$ that commute with all elements of $A$ is exactly $A$ as seen in the previous exercise. That $N_G(A) = S_3$ can be seen by following the multiplication table in exercise 1.5.2 again.
\item Same as above.
\item All elements $D_{10}$ are either powers of $r$ (and are in $A$) or are of the form $sr^k$. Since $rs \neq sr$, none of those elements commute with every element of $A$, and the only elements of $D_{10}$ that commute with powers of $r$ are other powers or $r$ so $C_G(A) = A$. To see how conjugation acts on $A$ consider conjugation by an arbitrary element not in $A$:
\begin{equation}
(sr^k)r^n(sr^k)^{-1} = (sr^k)r^n(sr^k) = sr^{k+n}sr^k = ssr^{k+n}r^k = r^{2k+n}
\end{equation}
so the exponents in $A$ are just shifted up by $2k$ and $A \mapsto A$. Conjugation by an element of the form $r^k$ trivially preserves $A$, so $N_G(A) = G$.
\end{enumerate}


\item We are given $H \leq G$.
\begin{enumerate}
\item Since $H$ is closed under inverses and multiplication, for any $g,h \in H$, $hgh^{-1} \in H$. Therefore $hHh^{-1} = H$. {\color{red} counterexample}
\item If $H \leq C_G(H)$, then, for all $g,h \in H$, $ghg^{-1} = h$ i.e. $gh = hg$ so $H$ is abelian. The other direction is the argument in reverse.
\end{enumerate}


\item 
\begin{enumerate}
\item All elements of $D_{2n}$ are $r^k$ and $sr^k$ for $k=0,\ldots,n-1$. We just need to show that every nonidentity element has at least one other element it doesn't commute with. Consider the multiplication of $r^k$ and $sr^l$:
\begin{align}
r^k \cdot sr^l &= r^k \cdot r^{-l}s = r^{k-l}s \\
sr^l \cdot r^k &= sr^{k+l} = r^{-k-l}s\ .
\end{align}
These products are only the same if $r^{2k} = 1$. Since $n$ is odd, that means $r^k = 1$, in which case we were working with $r^k = 1$ and $sr^k = s$. We still need to show that $s$ is not in the center; this is obvious because it doesn't commute with $r$. We have now explicitly shown that every nonidentity element fails to commute.
\item If $n$ is even then $r^{2k} = 1$ has another solution: $k = \frac{n}{2}$. Then this $r^k$ commutes with all $sr^l$, and it also trivially commutes with powers of $r$, so it is in the center of the group.
\end{enumerate}


\item The group $G = S_n$ acts on the set $A = \{1,\ldots ,n\}$, and $G_i = \{ \sigma\in G \mid \sigma (i) = i \}$ all keep $i$ fixed. The identity permutation has $\sigma(k) = k$ for all $k$, not just $i$, so $1 \in G_i$. If $\sigma_1, \sigma_2 \in G_i$, then $\sigma_1 \circ \sigma_2$ is as well since $(\sigma_1\circ \sigma_2)(i) = \sigma_1(\sigma_2(i)) = \sigma_1(i) = i$. Also, $\sigma_1^{-1}(i) = i$ as well, so $G_i$ is a group. If we send $i \mapsto i$ then there are only $n-1$ elements of $A$ to permute, so $\abs{G_i} = (n-1)!$.


\item By definition, $N_H(A)$ is the same as $N_G(A)$ but the conjugating elements come from $H$ instead. All of these elements have to be in $H$ (duh), but since $H \leq G$, all of these elements are also in $G$, so $N_H(A)$ is exactly the subset of $N_G(A)$ that is in $H$, i.e. $N_H(A) = N_G(A) \cap H$. {\color{red} subgroup of $H$ follows?}


\item If $H$ is a subgroup of order 2, then $H = \{1, x\}$ for some $x\in G$ where $x^2 = 1$. The difference between $N_G(H)$ and $C_G(H)$ in general is that conjugation by an element in $C_G$ sends every element of $H$ to itself, while conjugation by an element of $N_G$ can permute different elements of $H$. In this case, conjugation always maps the identity to itself ($g1g^{-1} = 1$ for any $g$), so if we want all of $H$ to be in the image of the conjugation we require $x \mapsto x$. We see that the only possible conjugation actions of $G$ sending $H$ to $H$ sends every element to itself, so $C_G(H) = N_G(H)$. If $N_G(H) = G$, then $C_G(H) = G$, meaning the elements of $H$ commute with all elements of $G$ and $H \leq Z(G)$. 


\item We want to show that if $g \in Z(G)$, then $g \in N_G(A)$ for any subset $A$ of $G$. If $g \in Z(G)$ then $g$ commutes with all elements of $G$, so $gag^{-1} = gg^{-1}a = a$ for any element $a \in A$. We then see that $gAg^{-1} = A$, so $g \in N_G(A)$. The shorter version is $Z(G) \leq C_G(A)$ for any subset $A$, and $C_G(A) \leq N_G(A)$ by definition.


\item Oh boy! (in retrospect this problem was interesting and I retract this statement)
\begin{enumerate}
\item Given $\sigma = \cycle{1,2,3,4}$ and $\tau = \cycle{1,2,3}$, $\sigma\circ\tau = \cycle{1,3,2,4}$ and $\tau\circ\sigma = \cycle{1,3,4,2}$. I'm not writing the polynomial out because it's what you find in the dictionary when you look up the word ``arbitrary''. The original is $p(x_1,x_2,x_3,x_4)$.
\begin{enumerate}
\item $\sigma\cdot p = p(x_2,x_3,x_4,x_1)$
\item $\tau\cdot (\sigma\cdot p) = \tau\cdot p(x_2,x_3,x_4,x_1) = p(x_3,x_1,x_4,x_2)$
\item $(\tau\circ\sigma)\cdot p = p(x_3,x_1,x_4,x_2)$ by associativity
\item $(\sigma\circ\tau)\cdot p = p(x_3,x_4,x_2,x_1)$
\end{enumerate}

\item The identity permutation sends each $p\in R$ to itself. The inverse of a permutation is a permutation, and the composition of two permutations is another permutation:
\begin{equation}
(\sigma_2 \circ \sigma_1)\cdot p(x_1,x_2,x_3,x_4) = p(x_{\sigma_2(\sigma_1(1))}, x_{\sigma_2(\sigma_1(2))}, x_{\sigma_2(\sigma_1(3))}, x_{\sigma_2(\sigma_1(4))}) = \sigma_2\cdot (\sigma_1\cdot p(x_1,x_2,x_3,x_4))
\end{equation}

\item The permutations that stabilize $x_4$ are the ones with no ``4'' in the cycle decomposition: $1$, $\cycle{1,2}$, $\cycle{1,3}$, $\cycle{2,3}$, $\cycle{1,2,3}$, and $\cycle{1,3,2}$. These are trivially isomorphic to $S_3$, I think I remember a sentence saying that the whole point of cycle notation being how it is is so that we could see that $S_n \leq S_m$ for $n < m$.

\item To stabilize the polynomial $x_1 + x_2$, we can either do nothing (duh), swap the labels $1 \leftrightarrow 2$, permute the remaining labels 3 and 4, or some mix. The permutations are $1$, $\cycle{1,2}$, $\cycle{3,4}$, and $\cycle{1,2}\cycle{3,4}$. Since $\cycle{1,2}$ and $\cycle{3,4}$ are disjoint cycles that square to the identity, this is an abelian subgroup.

\item To stabilize $x_1 x_2 + x_3 x_4$, we either swap $1 \leftrightarrow 2$, $3 \leftrightarrow 4$, or both as before, or we can also swap the two terms with $(1,2) \leftrightarrow (3,4)$. The explicit permutations are $1$, $\cycle{1,2}$, $\cycle{3,4}$, $\cycle{1,2}\cycle{3,4}$, $\cycle{1,3}\cycle{2,4}$, $\cycle{1,4}\cycle{2,3}$, $\cycle{1,4,2,3}$, and $\cycle{1,3,2,,4}$. Let $r = \cycle{1,4,2,3}$ and $s = \cycle{1,2}$, then $r^4 = s^2 = 1$ and $rs = sr^{-1}$, which is the usual presentation of $D_8$. The explicit representation of each element I listed is, in order, $1$, $s$, $sr^2$, $r^2$, $sr^3$, $sr$, $r$, $r^3$.,

\item To stabilize $(x_1 + x_2)(x_3 + x_4)$ we can perform the same swaps as the previous example. It's the same structure since 1 and 2 are grouped in a way where order doesn't matter, 3 and 4 are grouped in a way order doesn't matter, and the order of the two terms doesn't matter.


\end{enumerate}


\item See part $b$ of the previous exercise.


\item We want to find the conditions for an element of $H(F)$ to commute with all other elements. The matrix bashing is
\begin{align}
\begin{bmatrix}1&a&b\\0&1&c\\0&0&1\end{bmatrix}\begin{bmatrix}1&x&y\\0&1&z\\0&0&1\end{bmatrix} &= \begin{bmatrix}1&x+a&y+az+b\\0&1&z+c\\0&0&1\end{bmatrix} \\
\begin{bmatrix}1&x&y\\0&1&z\\0&0&1\end{bmatrix}\begin{bmatrix}1&a&b\\0&1&c\\0&0&1\end{bmatrix} &= \begin{bmatrix}1&a+x&b+xc+y\\0&1&c+z\\0&0&1\end{bmatrix}\ .
\end{align}
If we want this to be true for all $a,b,c$, then $x=z=0$, so 
\begin{equation}
Z(H(F)) = \left\{ \begin{bmatrix}1&0&y\\0&1&0\\0&0&1\end{bmatrix}\mid y \in F \right\}
\end{equation}
which has an obvious isomorphism to $F$ itself.

\end{enumerate}














\addcontentsline{toc}{section}{2.3: Cyclic Groups and Cyclic Subgroups}
\section*{\underline{2.3: Cyclic Groups and Cyclic Subgroups}}
\addcontentsline{toc}{subsection}{Exercises}
\subsection*{\underline{Exercises}}
\begin{enumerate}

\item Theorem 7 lets us list the subgroups by finding the gcd of $(n,45)$ for all $n<45$:
\begin{enumerate}
\item $Z_{45} = \gen{x} = \gen{x^2} = \gen{x^4} = \gen{x^7} = \gen{x^8}$ and all other $x^n$ for $n$ coprime to $45$, which is (I'm just padding so the set goes on the next line) $n \in \{11,13,14,16,17,19,22,23,26,28,29,31,32,34,37,38,41,43,44 \}$
\item $Z_{15} = \gen{x^3} = \gen{x^6} = \gen{x^{12}} = \gen{x^{21}} = \gen{x^{24}} = \gen{x^{33}} = \gen{x^{39}} = \gen{x^{42}}$
\item $Z_{9} = \gen{x^5} = \gen{x^{10}} = \gen{x^{20}} = \gen{x^{25}} = \gen{x^{35}} = \gen{x^{40}}$
\item $Z_{5} = \gen{x^9} = \gen{x^{18}} = \gen{x^{27}} = \gen{x^{36}}$
\item $Z_{3} = \gen{x^{15}} = \gen{x^{30}}$
\item $1 = \gen{1}$

\end{enumerate}


\item 


\item 


\item 


\item 


\item 


\item 


\item 


\item 


\item 


\item 


\item 


\item 


\item 


\item 


\item 


\item Isn't this just $Z_n = \langle\, x\mid x^n = 1\,\rangle$?


\item Let $Z_n$ be generated by $x$, and $h^n = 1$ in $H$. For a homomorphism $\phi: Z_n \to H$ with $\phi(x) = h$ then
\begin{equation}
1 = \phi(xx^{-1}) = \phi(x)\phi(x^{-1}) = h\phi(x^{-1})
\end{equation}
so $\phi(x^{-1}) = h^{-1}$. By induction,
\begin{align}
\phi(x^n) &= \phi(x)\phi(x^{n-1}) = hh^{n-1} = h^n \\
\phi(x^{-n}) &= \phi(x^{-1})\phi(x^{-(n-1)}) = h^{-1}h^{-(n-1)} = h^{-n}
\end{align}
and $\phi(1) = 1$, so $\phi(x^k) = h^k$ for any $k\in\bbz$. This is well-defined since $x^n = 1$ and $\phi(x^n) = h^n = 1$. By the division algorithm we have a unique map sending $x^k$ to $h^k$ for $k = 0,\ldots,n$.


\item This is very similar to the previous exercise. $(\bbz, +)$ is generated by the number $1$, so for a homomorphism $\phi: \bbz \to H$ with $\phi(1) = h$, 
\begin{equation}
\phi(2) = \phi(1+1) = \phi(1)\phi(1) = hh = h^2
\end{equation}
so $\phi(n) = h^n$ for all positive $n$ by induction. For negative exponents, since $\phi(0) = 1_H$,
\begin{equation}
1_H = \phi(0) = \phi(-1 + 1) = \phi(-1)\phi(1) = \phi(-1)h
\end{equation}
so $\phi(-1) = h^{-1}$. Then $\phi(-2) = h^{-2}$, and $\phi(-n) = h^{-n}$ by induction again. We have completely specified $\phi$. 


\item First we prove that, if for some $x\in G$, $\abs{x} = k$, and $x^m = 1$ for some $m>0$, then $k\mid m$. Assume that this is not true; that $\abs{x} = k$, $x^m = 1$, and $m = pk + r$ for some integers $p,r$ where $1 \leq r < k$ (the remainder is nonzero). Then 
\begin{equation}
1 = x^m = x^{pk + r} = \left(x^k \right)^p x^r = 1^p x^r = x^r
\end{equation}
which is a contradiction since $\abs{x} = k$ means that $x^r$ is not the identity. \\

Now, if $x^{p^n} = 1$, then $\abs{x}\mid p^n$. By the Fundamental Theorem of Arithmetic (is this overkill?) the only divisors of $p^n$ are $p^m$ for $m = 1,\ldots,n$.

\item 


\item 


\item 


\item $G$ is a finite group here.
\begin{enumerate}
\item If $g \in N_g(\gen{x})$ then by definition there exists an $n$ such that $gx^mg^{-1} = x^n$ for all $m$. Just choose $m=1$ here.
\item If $gxg^{-1} = x^a$, then $x^{2a} = gxg^{-1} \cdot gxg^{-1} = gxxg^{-1} = gx^2g^{-1}$ and by induction $x^{ka} = gx^kg^{-1}$. Therefore all powers of $x$ get sent to some (possibly other) power of $x$ under conjugation by $g$. We just need to show that the map is surjective. $G$ being finite means that $x$ has finite order, call it $n$. If $gx^ag^{-1} = gx^bg^{-1}$, then, left-canceling $g$ and right canceling $g^{-1}$, $x^a=x^b$, so conjugation by $g$ sends each power of $x$ to a different element of $G$. Therefore $\abs{g\gen{x}g^{-1}} = n = \abs{\gen{x}}$ so the two sets are equal and $g \in N_g(\gen{x}$.
\end{enumerate}


\item The first part of this is pretty obvious once written out in excruciating detail. Here $G = Z_n$, $(k,n) = 1$, and we want to show that $x\mapsto x^n$ is surjective. That means that for all $a \in \{0, 1, \ldots, n-1\}$, we want to show that there exists an element of $G$, namely $x^m$, that gets mapped to $x^a$, i.e. $x^a = (x^m)^k$ or $a = km$ where equality is modulo $n$. Since $(k,n) = 1$, $k \in (\bbz/n\bbz)^\times$ and there exists a $k^{-1}$ so that $m = k^{-1}a$. This identifies $m$ for a given $a$. {\color{red} lagrange?}


\item Let $Z_n$ be generated by $x$.
\begin{enumerate}
\item If $(a,n) = 1$, then we showed in the previous exercise that $\sigma_a$ is surjective. To show it is injective, suppose $\sigma_a(x^b) = \sigma_a(x^c)$. Then $x^{ab} = x^{ac}$ so $1 = x^{ac}x^{-(ab)} = x^{a(c-b)}$ and we see that either $a=0$, in which case $(a,n)$ couldn't be 1, or $b = c$ (modulo $n$) so the preimages are equal. The homomorphism part is just 
\begin{equation}
\sigma_a(x^b)\cdot \sigma_a(x^c) = x^{ab}x^{ac} = x^{a(b+c)} = \sigma_a(x^{b+c}) = \sigma_a(x^b\cdot x^c)\ .
\end{equation}
{\color{red} opposite direction?}
\item If $\sigma_a = \sigma_b$, then, for all $y \in Z_n$, $y^a = y^b$. Since $y = x^k$ for some $k$, that means $x^{ak} = x^{bk}$ or $x^{k(a-b)} = 1$. Since this is true for all $k$, we see that we must have $a = b$ (modulo $n$). In the other direction, if $a = b$ modulo $n$, let $c$ be the representative of the equivalence class on $[0,n-1]$. Then $a = m_a n + c$ and $b = m_b n + c$ (I am totally not running out of letters). For any $y = x^k$, 
\begin{align}
\sigma_a(y) = x^{ak} = x^{k(m_a n + c)} = x^{km_an}x^{kc} = x^{kc} \\
\sigma_b(y) = x^{bk} = x^{k(m_b n + c)} = x^{km_bn}x^{kc} = x^{kc}
\end{align}
so $\sigma_a = \sigma_b$.
\item 
\item Follow the image of an arbitrary element $y = x^k$:
\begin{align}
\sigma_a(\sigma_b(y)) = \sigma_a(y^b) = (y^b)^a = y^{ab} = \sigma_{ab}(y)\ .
\end{align}
Parts a and c showed that all automorphisms of $Z_n$ are given by some $\sigma_a$ where $(a,n) = 1$. Part b showed that these maps can be defined modulo $n$, so we can say that $a\in (\bbz/n\bbz)^\times$ without loss of generality. Therefore there is a bijection between elements of $(\bbz/n\bbz)^\times$ and automorphisms of $Z_n$, where $\sigma_a$ corresponds to $\bar{a}$. The homomorphism part is shown above, so we have an isomorphism between $(\bbz/n\bbz)^\times$ and $\text{Aut}(Z_n)$.
\end{enumerate}


\end{enumerate}

\end{document}