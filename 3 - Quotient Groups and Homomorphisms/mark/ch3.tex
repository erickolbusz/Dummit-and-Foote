\documentclass{article}
 
\usepackage[letterpaper, margin=0.75in]{geometry}
\usepackage{amsmath}
\usepackage{amssymb}
\usepackage{amsfonts}
\usepackage{xcolor}
\usepackage{hhline}
\usepackage[shortlabels]{enumitem}
\usepackage{amsthm}
\usepackage{bm}   
\usepackage{color}   %May be necessary if you want to color links
\usepackage[bookmarks,hypertexnames=false,debug,linktocpage=true,hidelinks]{hyperref}  

\hypersetup{
    colorlinks,
    linktoc=all,
    linkcolor={blue},
    citecolor={blue}, 
    urlcolor={blue}  
} 
%\setcounter{section}{-1}

\newcommand{\ints}{\mathbb{Z}}  
\newcommand{\reals}{\mathbb{R}} 
\newcommand{\rats}{\mathbb{Q}} 
\newcommand{\comps}{\mathbb{C}}
\newcommand{\set}[1]{ \{ #1 \} }
\newcommand{\mult}{\star}
\newcommand{\abs}[1]{| #1 |} 
\newcommand{\inv}[1]{ {#1}^{-1} }
\newcommand{\id}{ \bm{1} }
\newcommand{\iso}{ \cong }
\newcommand{\comp}{ \circ }
\newcommand{\lcm}{ \textrm{lcm}, }
\newcommand{\Aut}{ \textrm{Aut}, }
\newcommand{\Stab}{ \textrm{Stab}, }
\newcommand{\sheaf}{ \mathcal{O} }
\newcommand{\md}{\,\text{mod}\,}
\newcommand{\floor}[1]{\lfloor #1 \rfloor}
\newcommand{\ceil}[1]{\lceil #1 \rceil}
\newcommand{\cent}{Z}
\newcommand{\central}{C}
\newcommand{\nor}{N}
\renewcommand{\bar}{\overline}
\newcommand{\norm}[1]{|#1|}
\newcommand{\divides}{\big\vert}
\newcommand{\cyclic}[1]{\langle#1\rangle}
\newcommand{\normal}{\unlhd}
\newcommand{\nml}{\normal}
\newcommand{\phiinv}{\phi^{-1}}
\newcommand{\vbar}{\vert}
\DeclareMathOperator*{\argmin}{\arg\!\min}

\newtheorem{theorem}{Theorem}[section]
\newtheorem{corollary}{Corollary}[theorem]
\newtheorem{lemma}[theorem]{Lemma} 

\setlength\parindent{0pt}
%\allowdisplaybreaks

\title{TITLE}
\begin{document}
\setcounter{tocdepth}{2}
\tableofcontents 
\setcounter{section}{2} 
\section{Quotient Groups and Homomorphisms}  

\subsection{Definitions and Examples}
\subsubsection{}\label{ex1p1}
\begin{itemize}
\item $\phiinv(E) \leq G$
We're given $E \leq H$\\
Let $g,h \in \phiinv(E)$, so $\phi(g)=e,\phi(h)=f$, for $e,f\in E$.\\
Note that
\begin{align*}
\phi(g\inv{h}) &= \phi(g)\phi(h)^{-1}\\
&= e\inv{f}\\
&\in E\\
\implies g\inv{h} &\in \phiinv(E)
\end{align*}
Hence, by the subgroup criterion, $\phiinv(E)$ is a subgroup.
\item $\phiinv(E) \nml G$
We're given $E \nml H$\\
Let $g\in G$, $n\in \phiinv(E)$ with $\phi(n) = e$\\
\begin{align*}
\phi(gn\inv{g}) &= \phi(g)e\phi(g)^{-1}\\
&\in E & \mbox{(since $E \nml H$)}\\
\implies gn\inv{g} &\in \phiinv(E)
\end{align*}
Since $n\in \phiinv(E)$ was arbitrary, we have $g\phiinv(E)\inv{g} \subset \phiinv(E)$, making $\phiinv(E)$ normal.\\
Setting $E=1$ makes $\ker\phi$ normal.
\end{itemize}
\subsubsection{}\label{ex1p2}
We're given $w \in Z$, i.e. $w \in XY$, sp $w=rs$ for $r\in X, s\in Y$.\\
So
\begin{align*}
\phi(\inv{u}w) &= \phi(u)^{-1}\phi(r)\phi(s)\\
&= \inv{a}ab\\
&= b
\end{align*}
So $\inv{u}w \in Y$, i.e. $\inv{u}w = v$ for some $v \in Y$, i.e. $w=uv$.
\subsubsection{}\label{ex1p3}
Let $A$ be abelian, let $aB,bB \in A/B$, so
\begin{align*}
(aB)(bB) &= (ab)B\\
&= (ba)B\\
&= (bB)(aB)
\end{align*}
so $A/B$ is abelian.\\
Following the example in the text, $G=D_8$ is not abelian, but $D_8/Z(D_8) = V_4$ is.
\subsubsection{}\label{ex1p4}
$(gN)^0 = N$ (since $N$ is the identity)\\
But $g^0N = 1N =N$.\\
Hence, $(gN)^0 = g^0N$\\
Now suppose that $(gN)^k = g^kN$ for $k=1,\ldots,n$. Then
\begin{align*}
(gN)^{n+1} &= (gN)^n(gN)\\
&= g^nNgN\\
&= (g^ng)N\\
&= g^{n+1}N
\end{align*}
We have proved the statement for all nonnegative integers by induction. We prove it for negative integers by showing that they are appropriate inverses. For $k\in\ints^{+}$,
\begin{align*}
(g^{-k}N)(gN)^k &= g^{-k}Ng^kN\\
&= (g^{-k}g^k)N\\
&= 1N\\
&= N
\end{align*}
Hence,
\begin{align*}
g^{-k}N &= ((gN)^k)^{-1}\\
&= (gN)^{-k}
\end{align*}
\subsubsection{}\label{ex1p5}
Suppose $(gN)^k = N$. Then $g^kN = N$, and since $1\in N$, $g^k\cdot 1 \in N$, i.e. $g^k \in N$. The converse is also true, so the order of $gN$ must be the smallest int $k$ for which this holds.\\
Let $G=D_8$, $N=\set{1,r^2}$. Then $\abs{r} = 8$, but $r^2 \in N$ so $\abs{rN} =2$
\subsubsection{}\label{ex1p6}
$\phiinv(1)$ are the positive reals, $\phiinv(-1)$ are the negative reals.\\
Let $a,b\in \reals^{\times}$. Then $\phi(ab) = \frac{ab}{\abs{ab}} = \frac{a}{\abs{a}}\frac{b}{\abs{b}} = \phi(a)\phi(b)$
\subsubsection{}\label{ex1p7}
Let $(x,y),(a,b)\in\reals^{2}$. then $\pi((x,y)+(a,b))=\pi(x+a,y+b)=x+a+y+b=x+y+a+b = \pi(x,y)+\pi(a,b)$, making $\pi$ into a homomorphism. Also, given $a\in\reals$, $\pi(a,0) = a+0 = a$, so $\pi$ is surjective.\\
Note, $(x,y)\in\ker\pi \iff x+y=0 \iff y=-x$. So the kernel is the line $y=-x$. The fibers are simply translations of the line: The fiber of $b$ is the line $y=-x+b$.
\subsubsection{}\label{ex1p8}
Let $x,y\in\reals^{\times}$.\\
Then $\phi(xy) = \abs{xy} = \abs{x}\abs{y} = \phi(x)\phi(y)$, making it into a homomorphism. The image of $\phi$ is the positive reals.\\
We have 
\begin{align*}
x&\in\ker\phi \\
\iff \abs{x} &= 1\\
\iff x=1 &\mbox{or} x=-1
\end{align*}
So $\ker\phi=\set{-1,1}$.\\
The fibers take the form $x\ker\phi=x\set{-1,1}=\set{-x,x}$
\subsubsection{}\label{ex1p9}
This map just takes the square of the "modulus" or "norm" or "absolute value" of a complex number, so it is definitely a homorphism, and the image is the positive reals.\\
The kernel is the unit circle in the complex plainand the fiber of $x \in \reals^{\times}$ is the circle of radius $\sqrt{x}$
\subsubsection{}\label{ex1p10}
Suppose $\bar{a} = \bar{b}$ in $\ints/8\ints$.\\
Then $8\divides(b-a)$. I.e. $\exists d \in \ints$ such that $8d=b-a$.\\
But then $4(2d) = b-a$, so $4\divides b-a$, so in fact $\bar{a}=\bar{b}$ in $\ints/\ints$, making the map well-defined. The map is clearly a homomorphism and surjective.\\
We have $\bar{a} \in \ker\phi \iff \phi(\bar{a})=0 \iff \bar{a}=0 \iff 4\divides a$, so $\ker\phi=\set{\bar{0},\bar{4}}$. and the fibers take the form $\set{\bar{a},\bar{a+4}}$
\subsubsection{}\label{ex1p11}
\begin{enumerate}[(a)]
\item 
We have $\phi(\begin{pmatrix}a & b\\ 0 & c\end{pmatrix}\begin{pmatrix}e & f\\ 0 & g\end{pmatrix}) = \phi(\begin{pmatrix}ae & af+bg\\ 0 & cg\end{pmatrix}) = ae = \phi(\begin{pmatrix}a & b\\ 0 & c\end{pmatrix})\phi(\begin{pmatrix}e & f\\ 0 & g\end{pmatrix})$, so $\phi$ is a homomorphism. And it is clearly surjective because $\begin{pmatrix}a & 1\\ 0 & 1\end{pmatrix} \mapsto a$ for any $a \in F^{\times}$.\\
The kernel is
\begin{equation}
\ker\phi = \set{\begin{pmatrix}1 & b\\0 & c\end{pmatrix} \in G \vert c \neq 0}
\end{equation}
and fibers take the form
\begin{equation}
\begin{pmatrix}e & f\\0 & g\end{pmatrix}\ker\phi =
\set{\begin{pmatrix}e & f+bg\\0 & cg\end{pmatrix} \in G \vert c,e,g \neq 0}
\end{equation}
\item We have $\phi(\begin{pmatrix}a & b\\ 0 & c\end{pmatrix}\begin{pmatrix}e & f\\ 0 & g\end{pmatrix}) = \phi(\begin{pmatrix}ae & af+bg\\ 0 & cg\end{pmatrix}) = (ae,cg) = (a,c)(e,g) = \phi(\begin{pmatrix}a & b\\ 0 & c\end{pmatrix})\phi(\begin{pmatrix}e & f\\ 0 & g\end{pmatrix})$ and it's obviously surjective.\\
The kernel is
\begin{equation}
\ker\phi = \set{\begin{pmatrix}1 & b\\0 & 1\end{pmatrix} \in G | c \neq 0}
\end{equation}
and fibers take the form
\begin{equation}
\begin{pmatrix}e & f\\0 & g\end{pmatrix}\ker\phi =
\set{\begin{pmatrix}e & f+bg\\0 & g\end{pmatrix} \in G \vert e,g \neq 0}
\end{equation}
\item Let the map be given by $\begin{pmatrix}1 & b\\0 & 1\end{pmatrix} \mapsto b$. Note that in $H$, $\begin{pmatrix}1 & b\\0 & 1\end{pmatrix}\begin{pmatrix}1 & c\\0 & 1\end{pmatrix}\begin{pmatrix}1 & b+c\\0 & 1\end{pmatrix}$, so this map is clearly an isomorphism.
\end{enumerate}
\subsubsection{}\label{ex1p12}
\begin{align*}
\ker\phi &= \ints\\
\ker\phiinv(i) &= \set{\frac{1+4k}{4} \vbar k \in \ints}\\
\ker\phiinv(-1) &= \set{\frac{1+2k}{2} \vbar k \in \ints}\\
\ker\phiinv(e^{4\pi i/3}) &= \set{\frac{2+3k}{3} \vbar k \in \ints}
\end{align*}
We obtained $\ker\phiinv(i)$ by solving for $r$ in $2\pi r = \pi/2 + 2\pi k$, etc.
\subsubsection{}\label{ex1p13}
Divide out the results of the previous exercise by $2$ to account for the extra factor of $2$, and we obtain
\begin{align*}
\ker\phi &= \frac{1}{2}\ints = \set{k/2 \vbar k \in \ints}\\
\ker\phiinv(i) &= \set{\frac{1+4k}{8} \vbar k \in \ints}\\
\ker\phiinv(-1) &= \set{\frac{1+2k}{4} \vbar k \in \ints}\\
\ker\phiinv(e^{4\pi i/3}) &= \set{\frac{2+3k}{6} \vbar k \in \ints}
\end{align*}
\subsubsection{}\label{ex1p14}
\begin{enumerate}[(a)]
\item This is "obvious" but okay. Suppose $0\leq p,q < 1$ and suppose $p+\ints = q+\ints$. Then $p+0 \in q+\ints$, so $p=q+k$ for some $k\in\ints$. But $0\leq p,q < 1$, so we must have $\abs{k} < 1$, forcing $k=0$, so $p=q$.
\item Let $x+\ints \in \rats/\ints$, with $x = \frac{p}{q}$. Note that $qx=\frac{pq}{q} = p \in \ints$, so by exercise \ref{ex1p5}, $\abs{x} \leq q$. Hence all elements have finite order.\\
Euclid's theorem allows us to set the denominator to be an arbitrarily large prime number, so we have elements of arbitrarily large order.\\
\item Let $T$ be the torsion subgroup. We've just shown that $\rats/\ints \leq T$.  Suppose now that $x\in\reals/\rats$ (so $x$ is irrational). Then suppose $nx \in \ints$ for some int $n$. Then $nx = m$ for some int $m$, yielding $x=m/n$, making $x$ rational, a contradiction. Hence, $x+\ints$ has infinite order, and is note in $T$. Hence $\rats/\ints = T$.
\item Identify $e^{2\pi ik/n}$ with $k/n$. 
\end{enumerate}
\subsubsection{}\label{ex1p15}
Let $G$ be abelian and divisible.\\
Let $H \leq G$ be a proper subgroup.\\
Suppose we are given $aH \in G/H$ and $k \in \ints^{+}$.\\
We know that $\exists x \in G$ such that $x^k = a$.\\
But then $(xH)^k = x^kH = aH$.\\
So $\rats/\ints$ is certainly divisible.\\
{\large WHY DO I NEED $G$ TO BE ABELIAN AND WHY DOES $H$ HAVE TO BE PROPER}
\subsubsection{}\label{ex1p16}
Suppose $G=\cyclic{S}$. Let $xN\in\bar{G}$.\\
We know that $x=s_1\cdots s_n$ for some $s_1,\ldots,s_n\in S$.\\
Then $s_1N\cdots s_nN = s_1\cdots s_n N = xN$,\\
so $xN \in \cyclic{\bar{S}}$.\\
Since $x$ was arbitrary, we have $\bar{G}=\cyclic{\bar{S}}$.
\subsubsection{}\label{ex1p17}
Note that $G = D_{16}$.\\
\begin{enumerate}[(a)]
\item
Taking the left cosets of the (finitely many) elements of $D_8$, we obtain exactly 8 distinct cosets:
\begin{align*}
\bar{1} &= \set{1,r^4}\\
\bar{r} &= \set{r,r^5}\\
\bar{r^2} &= \set{r^2,r^6}\\
\bar{r^3} &= \set{r^3,r^7}\\
\bar{s} &= \set{s,sr^4}\\
\bar{sr} &= \set{sr,sr^5}\\
\bar{sr^2} &= \set{sr^2,sr^6}\\
\bar{sr^3} &= \set{sr^3,sr^7}
\end{align*}
NOTE THAT $\bar{G} \iso D_8$.
\item Listed out in part a
\item We note that $\bar{G} \iso D_8$, and we've computed these orders before:
\begin{align*}
\abs{\bar{1}} &= 1\\
\abs{\bar{r}} &= 4\\
\abs{\bar{r^2}} &= 2\\
\abs{\bar{r^3}} &= 4\\
\abs{\bar{s}} &= 2\\
\abs{\bar{sr}} &= 2\\
\abs{\bar{sr^2}} &= 2\\
\abs{\bar{sr^3}} &= 2
\end{align*}
\item Again, using $\bar{G} \iso D_8$, these calculations are trivial.
\begin{align*}
\bar{rs} &= \bar{s\inv{r}}\\
\bar{sr^{-2}s} &= \bar{r^2}\\
\bar{\inv{s}\inv{r}sr} &= \bar{r^2}
\end{align*}
\item Again we use $\bar{G}\iso D_8$, and note that in exercise 2.4 from chapter 2, we computed the following:
\begin{align*}
C(r^2) &= D_8\\
C(s) &= \set{1, r^2, s, sr^2}
\end{align*}
So we have to verify that $r,sr,sr^3$ are in the normalizer of $\cyclic{s,r^2}$ (specifically, when conjugating over $s$, since we already know $C(r^2) = D_8$. But $rs\inv{r} = r^2s, (sr)s(sr)^{-1} = sr^2, (sr^3)s(sr^3)^{-1} = s(r^2)^3$. Hence, $N(\cyclic{s,r^2}) = D_8$, sp $\cyclic{s,r^2}$ is normal.\\
That $\bar{H} \iso V_4$ is obvious with the identification of $s,r^2,sr^2$ with $a,b,c$ respectively.\\
Now consider the map $\pi: G \to \bar{G}$, which corresponds to the natural map $\pi: D_{16} \to D_8$. Specifically, it is given by $s \mapsto s$ and $r^a \mapsto r^{a \mod 4}$. Since $a$ is even $\iff a \mod 4$ is even, we can write
\begin{align*}
\inv{\pi}(H) &= \set{s^ar^b \vbar a\in\ints, b\in2\ints}\\
&= \set{1,s,r^2,r^4,r^6,sr^2,sr^4,sr^6}\\
&\iso D_6
\end{align*}
\item Noting again that $\bar{G} \iso D_8$, this was already computed in the examples of the text to be $V_8$.
\end{enumerate}
\subsubsection{}\label{ex1p18}
I will do all of the parts at once. We list out the elements again by taking left cosets\\
\begin{align*}
\bar{1} &= \set{1,\sigma^4}\\
\bar{\sigma} &= \set{\sigma,\sigma^5}\\
\bar{\sigma^2} &= \set{\sigma^2,\sigma^6}\\
\bar{\sigma^3} &= \set{\sigma^3,\sigma^7}\\
\bar{\tau} &= \set{\tau,\tau\sigma^4}\\
\bar{\tau \sigma} &= \set{\tau \sigma,\tau\sigma^5}\\
\bar{\tau\sigma^2} &= \set{\tau\sigma^2,\tau\sigma^6}\\
\bar{\tau\sigma^3} &= \set{\tau\sigma^3,\tau\sigma^7}
\end{align*}
Now note that $\bar{\sigma^3}\bar{\sigma} = \bar{\sigma^4} = \bar{1}$. Hence $\bar{\sigma^3}=\bar{\sigma}^{-1} = \bar{\inv{\sigma}}$, making the relation $\bar{\sigma\tau} = \bar{\tau\sigma^3}$ into $\bar{\sigma\tau} =\bar{\tau\inv{\sigma}}$. Hence, we are in the same situation as the last problem. We end up with $\bar{G} = D_8$ and the rest of the parts are straightforward.
\subsubsection{}\label{ex1p19}
\begin{enumerate}[(a)]
\item Same stuff as before, leading to...
\item $\bar{1},\bar{v},\bar{v}^2,\bar{v}^3,\bar{u},\bar{u}\bar{v},\bar{u}\bar{v}^2,\bar{u}\bar{v}^3$
\item
\begin{align*}
\abs{\bar{1}} &= 1\\
\abs{\bar{v}} &= 4\\
\abs{\bar{v^2}} &= 2\\
\abs{\bar{v^3}} &= 4\\
\abs{\bar{u}} &= 2\\
\abs{\bar{uv}} &= 4\\
\abs{\bar{uv^2}} &= 2\\
\abs{\bar{uv^3}} &= 4
\end{align*}
\item We note now that $\bar{vu} = \bar{uv^5} = \bar{uv}$, making $G$ abelian and the other computations for this part very straightforward. We obtain
\begin{align*}
\bar{vu} &= \bar{uv}\\
\bar{uv^{-2}u} &= \bar{u^2v^2}\\
\bar{\inv{u}\inv{v}uv} &= \bar{1}
\end{align*}
\item The map $\bar{G} \to \ints_2 \times \ints_4$ given by $u^av^b \mapsto (a,b)$ is clearly an isomorphism.
\end{enumerate}
\subsubsection{}\label{ex1p20}
Obvious.
\subsubsection{}\label{ex1p21}
IDK
\subsubsection{}\label{ex1p22}
We'll do the general case. We've already shown in a previous chapter that $H=\bigcap_{i\in I} H_i$ is a subgroup of $G$. Now let $h\in H$, and let $g\in G$. For each $i\in I$ we know $H_i$ is normal, and since $h\in H_i$,
\begin{align*}
\forall i\in I: gh\inv{g} &\in H_i\\
\implies gh\inv{g} &\in \bigcap_{i\in I} H_i\\
&= H
\end{align*}
Since $h\in H$ was arbitrary, $gH\inv{g} \subset H$, making $H$ normal.
\subsubsection{}\label{ex1p23}
Let $H,K$ be normal. Let $l\in \cyclic{H,K}, g\in G$.\\
So we can write $l=h_1k_1\cdots h_nk_n$.\\
For each $i=1,\ldots,n$,
\begin{align*}
gh_i &= h_i'g & h_i'\in H\\
hk_i &= k_i'h & k_i'\in K
\end{align*}
And thus
\begin{align*}
gl &= gh_1k_1\cdots h_nk_n\\
&= h_1'k_1'\cdots h_n'k_n'g\\
&\in \cyclic{H,K}g
\end{align*}
Since $l \in \cyclic{H,K}$ was arbitrary, $g\cyclic{H,K} \subset \cyclic{H,K}g$. An analogous proof for the reverse inclusion makes $\cyclic{H,K}$ normal.
\subsubsection{}\label{ex1p24}
Let $n\in N\cap H, h\in H$. Since $N$ is normal, $hn=n'h$ for some $n'\in N$. Since $n'=hn\inv{h}$, we have $n'\in H$. So in fact $n'\in N\cap H$. So $hN\cap H \subset N\cap Hh$. The reverse inclusion is analogous and thus $N\cap H$ is normal.
\subsubsection{}\label{ex1p25}
\begin{enumerate}[(a)]
\item \label{ex1p25a}
\begin{align}
N &\mbox{normal}\\
\iff \forall g\in G: gN &= Ng \label{line1}\\ 
\implies \forall g\in G,n\in N\exists l \in N: gn&=lg\\
\implies \forall g\in G,n\in N\exists l \in N: gn\inv{g}&=l\\
\implies \forall g\in G,n\in N: gn\inv{g}&\in N\\
\implies \forall g\in G: gN\inv{g} &\subset N
\end{align}
Following the proof backwards proves the other direction (\eqref{line1} turns into a onesided inclusion, but the reverse inclusion is analagous)
\item \label{ex1p25b}
We have $g=\begin{pmatrix}2 & 0\\0 & 1\end{pmatrix}, \inv{g}=\begin{pmatrix}\frac{1}{2} & 0\\0 & 1\end{pmatrix}$. Let $n\in N$, so $n=\begin{pmatrix}1 & a\\0 & 1\end{pmatrix}$. Then 
\begin{align*}
gn\inv{g} &= 
\begin{pmatrix}2 & 0\\0 & 1\end{pmatrix} \begin{pmatrix}1 & a\\0 & 1\end{pmatrix} \begin{pmatrix}\frac{1}{2} & 0\\0 & 1\end{pmatrix}\\
&=
\begin{pmatrix}2 & 2a\\0 & 1\end{pmatrix}
\begin{pmatrix}\frac{1}{2} & 0\\0 & 1\end{pmatrix}\\
&= 
\begin{pmatrix}1 & 2a\\0 & 1\end{pmatrix} \label{leeven}\\
&\in N
\end{align*}
Since $n\in N$ was arbitrary, $gN\inv{g} \subset N$, making $N$ normal.\\
Note from \eqref{leeven} that the top-right position in an element of $gN\inv{g}$ is in the form $2a$. So, for example $\begin{pmatrix}1 & \frac{3}{2}\\0 & 1\end{pmatrix}$
????????????
\end{enumerate}
\subsubsection{}\label{ex1p26}
\begin{enumerate}[(a)]
\item \label{ex1p26a}
$gab\inv{g}=ga\inv{g}gb\inv{g}$.\\
Let $\norm{a}=n$. Then $(ga\inv{g})^n = ga^n\inv{g} = g\inv{g}=1$, so $\norm{ga\inv{g}} \leq \norm{a}$. Now let $\norm{ga\inv{g}} = m$. Then $(ga\inv{g})^m = 1\implies ga^m\inv{g} = 1\implies a^m = 1$, so $\norm{a} \leq \norm{ga\inv{g}}$.
\item \label{ex1p26b}
$(g\inv{a}\inv{g})(ga\inv{g}) = g\inv{a}\inv{g}ga\inv{g} = g\inv{a}a\inv{g} = g\inv{g} = 1$.
\item \label{ex1p26c}
Let $n\in N$ be normal. Then $n=s_1\cdots s_m$ for $s_i\in S$. Given $g\in G$, we're given $gS\inv{g} \subset N$, so we have for $i=1,\ldots, m$, we have $gs_i\inv{g} = n_i$ for some $n_i \in N$. Then,
\begin{align*}
gn\inv{g} &= gs_1\cdots s_m\inv{g}\\
&= gs_1\inv{g}\cdots gs_m\inv{g}\\
&= gn_1\inv{g}\cdots gn_m\inv{g}\\
&= gn_1\cdots n_m\inv{g}\\
&\in N
\end{align*}
\item \label{ex1p26d}
Follows immediately from \ref{ex1p26c} by setting $S=\set{x}$
\item \label{ex1p26e}
Follows directly from \ref{ex1p26c}
\end{enumerate}
\subsubsection{}\label{ex1p27}
One side is trivial, so suppose $gN\inv{g} \subset N$. Conjugation is injective ($ga\inv{g} = gb\inv{g} \implies a = b$ by left and right multiplication), and an injective map from a finite set to itself is surjective, so $gN\inv{g} = n$. Then it's clear that $N_G(N) = \set{g \in G \vbar gN\inv{g} \subset N}$
\subsubsection{}\label{ex1p28}
From the proof of exercise \ref{ex1p26}, part \ref{ex1p26b}, we have $gS\inv{g} \subset N \iff gN\inv{g} \subset N$, then applying \ref{ex1p27} completes the proof.
\subsubsection{}\label{ex1p29}
One direction is easy: $g$ normalizes $N \implies gN\inv{g} = N \implies gS\inv{g} \subset N$. Now for the converse.\\
Suppose $tS\inv{t} \subset N$ for all $t\in T$. By the previous exercise \ref{ex1p28}, $tN\inv{t} = N$ for any $t\in T$\\
Now $g\in G$ be arbitrary. Since $G=\cyclic{T}, g=t_1\cdots t_n, t_i \in T$. Then
\begin{align*}
gN\inv{g} &=  (t_1\cdots t_n)N(t_1\cdots t_n)^{-1}\\
&=  t_1\cdots t_nN\inv{t_n}\cdots \inv{t_1}\\
&=  t_1\cdots t_{n-1}(t_nN\inv{t_n})\inv{t_{n-1}}\cdots \inv{t_1}\\
&=
t_1\cdots t_{n-1}N\inv{t_{n-1}}\cdots \inv{t_1}\\
& \mbox{etc...}\\
&= N
\end{align*}
\subsubsection{}\label{ex1p30}
\begin{align*}
g &\in N_G(N)\\
\iff gN\inv{G} &= N\\
\implies\forall n\in N \exists m\in N: gn\inv{g} &= m\\
\implies\forall n\in N \exists m\in N: gn &= mg\\
\implies gN &\subset Ng
\end{align*}
Switch $m$ and $n$ in the quantifiers to obtain $gN \supset Ng$, so $gN=Ng$.\\
For the converse, 
\begin{align*}
gN &= Ng\\
\implies\forall n\in N \exists m\in N: gn\inv{g} &= m\\
\implies\forall n\in N \exists m\in N: gn &= mg\\
\implies gN\inv{g} &\subset N
\end{align*}
Again, switch $m$ and $n$ to obtain the reverse inclusion.
\subsubsection{}\label{ex1p31}
Since $N\lhd H$, we have $hN = Nh$, and the previous exercise \ref{ex1p30} says that $h\in N_G(N)$. Since $h\in H$ was arbitrary, $H\leq N_G(N)$. The deduction is trivial.
\subsubsection{}\label{ex1p32}
$1 and Q_8$ are obviously normal, and $\cyclic{-1} = \set{-1,1}$ is obviously normal since the elements commute with all of $Q_8$ ($\cyclic{-1} \subset Z(Q_8)$). Let $N = \cyclic{i}$. From the lattice, we know that $\cyclic{i,j}=Q_8$, so to show $N$ is normal, we use problem \ref{ex1p29} with $T=\set{i,j}, S=\set{i}$. We have $ii\inv{i} = i \in S \subset N$, and $ji\inv{g} =i \in S \subset N$. Hence, $N=\cyclic{S}$ is normal. The proof for $\cyclic{j}$ and $\cyclic{k}$ is analagous.
\subsubsection{}\label{ex1p33}
NO THANKS
\subsubsection{}\label{ex1p34}
\end{document}












