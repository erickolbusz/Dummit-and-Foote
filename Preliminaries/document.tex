\documentclass[]{article}

\usepackage[letterpaper, margin=0.75in]{geometry}
\usepackage{amsmath}
\usepackage{amssymb}
\usepackage{amsfonts}
\usepackage{xcolor}
\usepackage{hhline}
%\usepackage{cancel}
%\usepackage{mathtools}
%\usepackage{empheq}
%\usepackage{centernot}

%\usepackage{tikz}
%\usepackage{tikz-feynman}

%\newcommand*\emphbox[1]{\fbox{\hspace{0.25em}#1\hspace{0.25em}}}
%\newcommand{\normord}[1]{{:}\!\mathrel{#1}\!{:}}

%\newcommand{\pdx}[2]{\frac{\partial #1}{\partial #2}}
%\newcommand{\abs}[1]{\left\lvert #1 \right\rvert}
%\newcommand{\intp}[1]{\int \frac{d^3 #1}{(2\pi)^3}}
%\newcommand{\intpp}[1]{\int \frac{d^4 #1}{(2\pi)^4}}
%\newcommand{\intt}{\int d\alpha d\beta}
%\newcommand{\ints}{\int dx dy}
%\newcommand{\comm}[2]{\left[ #1 , #2 \right]}
%\newcommand{\acomm}[2]{\left\{ #1 , #2 \right\}}
%\newcommand{\bra}[1]{\langle #1 \vert}
%\newcommand{\ket}[1]{\vert #1 \rangle}
%\newcommand{\braz}{\bra{0}}
%\newcommand{\ketz}{\ket{0}}
%\newcommand{\bkz}{\left\langle 0 \vert 0 \right\rangle}
%\newcommand{\fsl}[1]{{\ooalign{\(#1\)\cr\hidewidth\(/\)\hidewidth\cr}}}
%\newcommand{\fsl}[1]{{\centernot{#1}}}

\newcommand{\abs}[1]{\left\vert #1 \right\vert}
\newcommand{\md}{\,\text{mod}\,}
\newcommand{\bbz}{\mathbb{Z}}
\newcommand{\bbq}{\mathbb{Q}}
\newcommand{\bbr}{\mathbb{R}}
\newcommand{\bbc}{\mathbb{C}}

\setlength\parindent{0pt}
%\allowdisplaybreaks

\title{Preliminaries}
\author{}
\date{}
\begin{document}

\maketitle
\vspace{-5em}

\section*{\underline{0.1: Basics}}

\subsection*{\underline{Proposition 1}}
\begin{enumerate}
\item For (not injective) $\rightarrow$ (no left inverse): if $f$ is not injective then there exist $a_1,a_2 \in A$ such that $f(a_1) = f(a_2)$, call this element $b \in B$. If a left inverse $g: B \to A$ existed it would be the case that $g(b) = a_1$ but also $g(b) = a_2$. This is not a function. \\
For (injective) $\rightarrow$ (left inverse): we want to construct a $g:B\to A$ such that $g(f(a)) = a$ holds for all $a \in A$. The definition of injective given is that if $a_1 \neq a_2$, then $f(a_1) \neq f(a_2)$. The contrapositive is that if $f(a_1) = f(a_2)$, then $a_1 = a_2$, meaning that the preimage of any $b \in \text{im}f$ is a one-element set $\{a\}$. Defining $g(b) = a$ means $g(f(a)) = g(b) = a$ as required.
\item For (not surjective) $\rightarrow$ (no right inverse): if $f$ is not surjective then there exists $b \in B$ such that $b \notin \text{im}f$. Then, $f \circ g$ (for some $g: B \to A$) cannot be the identity map because $(f \circ g)(b)$ cannot be $b$. \\ For (surjective) $\rightarrow$ (right inverse): we want to construct $g: B\to A$ such that $f(g(b)) = b$ holds for all $b \in B$. For a given $b \in B$, choose an $a$ such that $f(a) = b$. This is always possible since $\text{im}f = B$. Define $g(b) = a$, then $f(g(b)) = f(a) = b$ as required.
\item For (bijective) $\rightarrow$ (inverse exist): if $f$ is bijective then it is injective and surjective. From the above we see that $f$ has a left inverse $g$ and right inverse $h$, now we need to show $g=h$.
\begin{align}
f &= f \\
I \circ f &= f \circ I \\
(f \circ h) \circ f &= f \circ (g \circ f) \\
f \circ h \circ f &= f \circ g \circ f \\
g \circ f \circ h \circ f &= g \circ f \circ g \circ f \\
I \circ h \circ f &= I \circ g \circ f \\
I \circ h \circ f \circ h &= I \circ g \circ f \circ h \\
I \circ h \circ I &= I \circ g \circ I \\
h &= g
\end{align}
\item It suffices to show (injective) $\leftrightarrow$ (surjective). If $f$ is injective then (from part 1) the preimage of every $b \in \text{im}f$ is a single element set. Since every element of $A$ is some preimage, $\abs{\text{im}f} = \abs{A}$. Then since $\abs{A} = \abs{B}$ we see that $\text{im}f = B$. If $f$ is surjective then $\abs{\text{im}f} = \abs{B}$, but $\abs{A} \geq \abs{\text{im}f}$ (by something like the pigeonhole principle), so $\abs{A} = \abs{\text{im}f}$ and we have a bijection between the two sets.
\end{enumerate}

\subsection*{\underline{Exercises}}
\begin{enumerate}
\item Instead of checking these just work out the general case first. Let $X = \begin{bmatrix}a&b\\c&d\end{bmatrix}$.
\begin{align}
MX &= \begin{bmatrix}1&1\\0&1\end{bmatrix}\begin{bmatrix}a&b\\c&d\end{bmatrix} = \begin{bmatrix}a+c&b+d\\c&d\end{bmatrix} \\
XM &= \begin{bmatrix}a&b\\c&d\end{bmatrix}\begin{bmatrix}1&1\\0&1\end{bmatrix} = \begin{bmatrix}a&a+b\\c&c+d\end{bmatrix}
\end{align}
The equality of the diagonal elements means $c=0$, the equality of the top right entries means $a=d$. The general form of $X$ is
\begin{equation}
X = \begin{bmatrix}a&b\\0&a\end{bmatrix}\ . \label{eq:ex1}
\end{equation}
So in order, the answers are: yes, no, yes, no, yes, no.
\item If $P,Q \in \mathcal{B}$, $(P+Q)M = PM + QM = MP + MQ = M(P+Q)$.
\item $(PQ)M = P(QM) = P(MQ) = (PM)Q = (MP)Q = M(PQ)$.
\item See \eqref{eq:ex1}.
\item \begin{enumerate}
\item No: then $1 = f\left(\frac{1}{2}\right) = f\left(\frac{2}{4}\right) = 2$.
\item Yes: $f\left(\frac{mx}{my}\right) = \frac{m^2x^2}{m^2y^2} = \frac{x^2}{y^2} = f\left(\frac{x}{y}\right)$ (anyway, this is just $x \mapsto x^2$).
\end{enumerate}
\item No: then $0 = f(1.0) = f(0.\bar{9}) = 9$.
\item Reflexivity: $f(a) = f(a)$ so $a \sim a$. Symmetry: $a \sim b \implies f(a) = f(b) \implies f(b) = f(a) \implies b\sim a$. Transitivity: equality is transitive so if $a \sim b$ and $b \sim c$ then $f(a) = f(b) = f(c)$ so $a \sim c$.
\end{enumerate}









\section*{\underline{0.2: Properties of the Integers}}
\subsection*{\underline{Exercises}}
\begin{enumerate}
\item {\color{red} Future Eric promises to do this boring computation: find GCD using the Euclidean Algorithm, then LCM = product/GCD}
\item If $k \vert a$ and $k \vert b$ then $a = mk$ and $b = nk$ for some $m,n\in\bbz$. Then $as+bt = mks + nkt = k(ms+nt)$.
\item If $n$ is composite then it is either a power of a prime $p^n$ or its prime factorization $\prod_i p_i^{n_i}$ contains multiple primes. In the first case, $a = p$ and $b = p^{n-1}$. In the second case, $a = p_1^{n_1}$ and $b = \prod_{i>1}p_i^{n_i}$.
\item $ax + by = a\left(x_0 + \frac{b}{d}t\right) + b\left(y_0 - \frac{a}{d}t\right) = ax_0 + by_0 + a\frac{b}{d}t - b\frac{a}{d}t = N$
\item {\color{red} Future Eric does not promise to do this boring computation}
\item Call the set $A$. If $\abs{A}=1$, trivially that element $m$ is the minimal element. For $\abs{A}=n>1$ elements, choose one element $a$ (finite choice) and make the set $A^* = \{b \vert b\in A, b\neq a \}$. Then $A^*$ has a minimal element, call it $m_{n-1}$. Since $A$ is a set, $a \neq m_{n-1}$. Then one of them is smaller, and is the unique minimal element of $A$.
\item If $a^2 = pb^2$ then $p = \frac{a^2}{b^2} = \left(\frac{a}{b}\right)^2$. Any prime factor on the right-hand side shows up an even number of times; the only prime factor on the left-hand side shows up once (an odd number of times).
\item {\color{red} ?}
\item {\color{red} programming}
\item {\color{red} ?}
\item Write the prime factorization of $n$ as $\prod_i p_i^{n_i}$. Then $\phi(n) = \prod_i (p_i-1)p_i^{n_i-1}$. If $d \vert n$ then all the prime factors of $d$ are prime factors of $n$; the prime factorization of $d$ is $p_{i_1}^{m_1} \ldots p_{i_j}^{m_j}$ where $i_1 \ldots i_j$ are some labels of prime factors of $n$ and all $m_i$ are at most the corresponding $n_i$. Then $\phi(d) = \prod_k (p_{i_k}-1)p_k^{m_k-1}$. Comparing to $\phi(n)$, we see $(p_{i_k}-1)p_k^{m_k-1} \vert (p_k-1)p_k^{n_k-1}$ since $n_k \geq m_k$. This is true for each prime factor of $d$.
\end{enumerate}














\section*{\underline{0.3: $\bbz/n\bbz$ : The Integers Modulo $n$}}

\subsection*{\underline{Exercises}}
\begin{enumerate}
\item \begin{enumerate}
\item[$\bar{0}$]: $\{ \ldots, -36, -18, 0, 18, 36, \ldots \}$ 
\item[$\bar{1}$]: $\{ \ldots, -35, -17, 1, 19, 37, \ldots \}$ 
\item[$\bar{n}$]: $\{ \ldots, -36+n, -18+n, n, 18+n, 36+n, \ldots \}$ 
\end{enumerate}
\item For $a \in \bbz$, there exists unique $q,r\in \bbz$ with $r\in [0,n)$ such that $a = qn+r$. Then $a$ and $r$ are in the same equivalence class modulo $n$.
\item Write $a = \sum_i a_i10^i$. Then 
\begin{align}
a\md 9 &= \sum_i \left(a_i10^i\right)\md 9 \\
&= \sum_i \left(a_i\right)\md 9 \left(10^i\right)\md 9 \\
&\equiv \sum_i \left(a_i\right)\md 9 \left(10\md 9\right)^i \\
&= \sum_i \left(a_i\right)\md 9 \left(1\md 9\right)^i \\
&= \sum_i \left(a_i\right)\md 9 \left(1\md 9\right) \\
&= \sum_i a_i\md 9
\end{align}
\item $37 \equiv 8\mod 29$, so we want $8^{100}\mod 29$. Notice that $8^2 = 64 = 58+8 \equiv 8\mod 29$ so you can keep multiplying by 8 to find $8 \equiv 8^2 \equiv 8^3 \equiv \ldots \equiv 8^{100}$.
\item We want $9^{1500}\mod 100$. Note that (trial and error) $9^10 \equiv 1\mod 100$. Then $9^{1500} = (9^{10})^{150} \equiv 1^{150}\mod 100 = 1\mod 100$.
\item $0^2 = 0$; $1^2 = 1$; $2^2 = 4 \equiv 0$; $3^2 = 9 \equiv 1$
\item Both $a^2$ and $b^2$ are 0 or 1 mod 4, so their sum is 0, 1 or 2 mod 4 (not 3).
\item The left-hand side of $a^2 + b^2 = 3c^2$ is, from above, in one of $\bar{0}, \bar{1}, \bar{2}$. Since $c^2$ is in $\bar{0}$ or $\bar{1}$, the right-hand side is in $\bar{0}$ or $\bar{3}$. For equality the only option is that $c^2$ is in $\bar{0}$ while $a^2$ and $b^2$ are either both in $\bar{0}$ or $\bar{2}$. In either case all are even; being squares, this means that $a^2, b^2, c^2$ are all divisible by 4 and in $\bar{0}$, so we can divide both sides by 4. This can be repeated ad infinitum, even though $a^2, b^2, c^2 > 0$ and there is a minimal positive integer 1.
\item $1^2 = 1$; $3^3 = 9 \equiv 1$; $5^2 = 25 \equiv 1$; $7^2 = 49 \equiv 1$; any higher odd integer is equivalent to one of these.
\item $(\bbz / n \bbz)^x = \{ \bar{a} \in \bbz / n\bbz \vert (a,n) = 1 \}$, and $\phi(n)$ is the number of $a$ on $[1,n]$ (equivalent to $[0,n-1]$ modulo $n$) such that $(a,n) = 1$.
\item If $(a,n) = 1$ and $(b,n) = 1$, then, taking the prime factorizations of all three, we see that $a$ and $n$ have no prime factors in common (likewise for $b$ and $n$). Then $ab$ has no prime factors in common with $n$, so $(ab,n) = 1$. Take everything modulo $n$ to complete the proof.
\item 
\item If $(a,n) = 1$ then there exist $x,y\in\bbz$ such that $ax+ny=1$. Take both sides modulo $n$ to get \\ $(a\mod n)(x\mod n) \equiv 1\mod n$. Then $c = x$, or any equivalent.
\item Exercise 12 shows that if $(a,n) \neq 1$, then there is no $c$ such that $ac \equiv 1\mod n$. Exercise 13 shows that if $(a,n) = 1$, then there is such a $c$. We can take $c$ to be on $[0,n-1)$ without loss of generality. Then $\{ \bar{a} \vert (a,n) = 1 \} = \{ \bar{a} \vert \exists c\text{ such that }\bar{a}\cdot\bar{c} \equiv 1\mod n \}$. For $n=12$, the first set is $\{ 1,5,7,11 \}$. To see which elements aren't invertible we have to make a times table: 
\begin{center}
\begin{tabular}{c||c|c|c|c|c|c|c|c|c|c|c|c|}
& \textbf{0} & \textbf{1} & \textbf{2} & \textbf{3} & \textbf{4} & \textbf{5} & \textbf{6} & \textbf{7} & \textbf{8} & \textbf{9} & \textbf{10} & \textbf{11} \\
\hhline{|=|=|=|=|=|=|=|=|=|=|=|=|=|}
\textbf{0} & 0 & 0 & 0 & 0 & 0 & 0 & 0 & 0 & 0 & 0 & 0 & 0 \\ \hline
\textbf{1} & 0 & {\color{red} 1} & 2 & 3 & 4 & 5 & 6 & 7 & 8 & 9 & 10 & 11 \\ \hline
\textbf{2} & 0 & 2 & 4 & 6 & 8 & 10 & 0 & 2 & 4 & 6 & 8 & 10 \\ \hline
\textbf{3} & 0 & 3 & 6 & 9 & 0 & 3 & 6 & 9 & 0 & 3 & 6 & 9 \\ \hline
\textbf{4} & 0 & 4 & 8 & 0 & 4 & 8 & 0 & 4 & 8 & 0 & 4 & 8 \\ \hline
\textbf{5} & 0 & 5 & 10 & 3 & 8 & {\color{red} 1} & 6 & 11 & 4 & 9 & 2 & 7 \\ \hline
\textbf{6} & 0 & 6 & 0 & 6 & 0 & 6 & 0 & 6 & 0 & 6 & 0 & 6 \\ \hline
\textbf{7} & 0 & 7 & 2 & 9 & 4 & 11 & 6 & {\color{red} 1} & 8 & 3 & 10 & 5 \\ \hline
\textbf{8} & 0 & 8 & 4 & 0 & 8 & 4 & 0 & 8 & 4 & 0 & 8 & 4 \\ \hline
\textbf{9} & 0 & 9 & 6 & 3 & 0 & 9 & 6 & 3 & 0 & 9 & 6 & 3 \\ \hline
\textbf{10} & 0 & 10 & 8 & 6 & 4 & 2 & 0 & 10 & 8 & 6 & 4 & 2 \\ \hline
\textbf{11} & 0 & 11 & 10 & 9 & 8 & 7 & 6 & 5 & 4 & 3 & 2 & {\color{red} 1}\\ \hline
\end{tabular}
\end{center}
\item {\color{red} Future Eric does not promise to do this boring computation}
\item {\color{red} programming}
\end{enumerate}
\end{document}