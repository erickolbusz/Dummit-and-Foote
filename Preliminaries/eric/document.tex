\documentclass[]{article}

\usepackage[letterpaper, margin=0.75in]{geometry}
\usepackage{amsmath}
\usepackage{amssymb}
\usepackage{amsfonts}
\usepackage{xcolor}
\usepackage{hhline}
\usepackage{hyperref}

\newcommand{\abs}[1]{\left\vert #1 \right\vert}
\newcommand{\md}{\,\text{mod}\,}
\newcommand{\bbz}{\mathbb{Z}}
\newcommand{\bbq}{\mathbb{Q}}
\newcommand{\bbr}{\mathbb{R}}
\newcommand{\bbc}{\mathbb{C}}

\setlength\parindent{0pt}
%\allowdisplaybreaks

\usepackage{graphicx}
\usepackage{listings}

\graphicspath{ {./py/img/} }

\definecolor{codegreen}{rgb}{0,0.6,0}
\definecolor{codegray}{rgb}{0.5,0.5,0.5}
\definecolor{codepurple}{rgb}{0.58,0,0.82}
\definecolor{backcolour}{rgb}{0.95,0.95,0.92}

\lstdefinestyle{mystyle}{
	backgroundcolor=\color{backcolour},   
	commentstyle=\color{codegreen},
	keywordstyle=\color{orange},
	numberstyle=\tiny\color{codegray},
	stringstyle=\color{codepurple},
	basicstyle=\ttfamily\footnotesize,
	breakatwhitespace=false,         
	breaklines=true,                 
	captionpos=b,                    
	keepspaces=true,                 
	numbers=left,                    
	numbersep=5pt,                  
	showspaces=false,                
	showstringspaces=false,
	showtabs=false,                  
	tabsize=2
}

\lstset{style=mystyle}



\title{Preliminaries}
\author{}
\date{}
\begin{document}

\maketitle
\vspace{-5em}



\section*{\underline{0.1: Basics}}
\subsection*{\underline{Proposition 1}}
\begin{enumerate}
\item For (not injective) $\rightarrow$ (no left inverse): if $f$ is not injective then there exist $a_1,a_2 \in A$ such that $f(a_1) = f(a_2)$, call this element $b \in B$. If a left inverse $g: B \to A$ existed it would be the case that $g(b) = a_1$ but also $g(b) = a_2$. This is not a function. \\
For (injective) $\rightarrow$ (left inverse): we want to construct a $g:B\to A$ such that $g(f(a)) = a$ holds for all $a \in A$. The definition of injective given is that if $a_1 \neq a_2$, then $f(a_1) \neq f(a_2)$. The contrapositive is that if $f(a_1) = f(a_2)$, then $a_1 = a_2$, meaning that the preimage of any $b \in \text{im}f$ is a one-element set $\{a\}$. Defining $g(b) = a$ means $g(f(a)) = g(b) = a$ as required.
\item For (not surjective) $\rightarrow$ (no right inverse): if $f$ is not surjective then there exists $b \in B$ such that $b \notin \text{im}f$. Then, $f \circ g$ (for some $g: B \to A$) cannot be the identity map because $(f \circ g)(b)$ cannot be $b$. \\ For (surjective) $\rightarrow$ (right inverse): we want to construct $g: B\to A$ such that $f(g(b)) = b$ holds for all $b \in B$. For a given $b \in B$, choose an $a$ such that $f(a) = b$. This is always possible since $\text{im}f = B$. Define $g(b) = a$, then $f(g(b)) = f(a) = b$ as required.
\item For (bijective) $\rightarrow$ (inverse exist): if $f$ is bijective then it is injective and surjective. From the above we see that $f$ has a left inverse $g$ and right inverse $h$, now we need to show $g=h$.
\begin{align}
f &= f \\
I \circ f &= f \circ I \\
(f \circ h) \circ f &= f \circ (g \circ f) \\
f \circ h \circ f &= f \circ g \circ f \\
g \circ f \circ h \circ f &= g \circ f \circ g \circ f \\
I \circ h \circ f &= I \circ g \circ f \\
I \circ h \circ f \circ h &= I \circ g \circ f \circ h \\
I \circ h \circ I &= I \circ g \circ I \\
h &= g
\end{align}
\item It suffices to show (injective) $\leftrightarrow$ (surjective). If $f$ is injective then (from part 1) the preimage of every $b \in \text{im}f$ is a single element set. Since every element of $A$ is some preimage, $\abs{\text{im}f} = \abs{A}$. Then since $\abs{A} = \abs{B}$ we see that $\text{im}f = B$. \\ If $f$ is surjective then $\abs{\text{im}f} = \abs{B}$, but $\abs{A} \geq \abs{\text{im}f}$ (by something like the pigeonhole principle), so $\abs{A} = \abs{\text{im}f}$ and we have a bijection between the two sets.
\end{enumerate}


\subsection*{\underline{Exercises}}
\begin{enumerate}
\item Instead of checking these just work out the general case first. Let $X = \begin{bmatrix}a&b\\c&d\end{bmatrix}$.
\begin{align}
MX &= \begin{bmatrix}1&1\\0&1\end{bmatrix}\begin{bmatrix}a&b\\c&d\end{bmatrix} = \begin{bmatrix}a+c&b+d\\c&d\end{bmatrix} \\
XM &= \begin{bmatrix}a&b\\c&d\end{bmatrix}\begin{bmatrix}1&1\\0&1\end{bmatrix} = \begin{bmatrix}a&a+b\\c&c+d\end{bmatrix}
\end{align}
The equality of the diagonal elements means $c=0$, the equality of the top right entries means $a=d$. The general form of $X$ is
\begin{equation}
X = \begin{bmatrix}a&b\\0&a\end{bmatrix}\ . \label{eq:ex1}
\end{equation}
So in order, the answers are: yes, no, yes, no, yes, no.
\item If $P,Q \in \mathcal{B}$, $(P+Q)M = PM + QM = MP + MQ = M(P+Q)$.
\item $(PQ)M = P(QM) = P(MQ) = (PM)Q = (MP)Q = M(PQ)$.
\item See \eqref{eq:ex1}.
\item \begin{enumerate}
\item No: then $1 = f\left(\frac{1}{2}\right) = f\left(\frac{2}{4}\right) = 2$.
\item Yes: $f\left(\frac{mx}{my}\right) = \frac{m^2x^2}{m^2y^2} = \frac{x^2}{y^2} = f\left(\frac{x}{y}\right)$ (anyway, this is just $x \mapsto x^2$).
\end{enumerate}
\item No: then $0 = f(1.0) = f(0.\bar{9}) = 9$. 
\item Reflexivity: $f(a) = f(a)$ so $a \sim a$. \\
Symmetry: $a \sim b \implies f(a) = f(b) \implies f(b) = f(a) \implies b\sim a$. \\
Transitivity: equality is transitive so if $a \sim b$ and $b \sim c$ then $f(a) = f(b) = f(c)$ so $a \sim c$.
\end{enumerate}





\section*{\underline{0.2: Properties of the Integers}}
\subsection*{\underline{Exercises}}
\begin{enumerate}
\item Find the gcd using the Euclidean Algorithm, then $\text{lcm} = \frac{ab}{(a,b)}$.
\begin{enumerate}
\item Find the gcd first:
\begin{align}
20 &= 1\cdot 13 + 7 \\
13 &= 1\cdot 7 + 6 \\
7 &= 1\cdot 6 + 1\\
6 &= 6\cdot 1
\end{align}
so $(20,13) = 1$. Then the lcm is $\frac{20\cdot 13}{1} = 260$.
\item Find the gcd first:
\begin{align}
372 &= 5\cdot 69 + 27 \\
69 &= 2\cdot 27 + 15 \\
27 &= 1\cdot 15 + 12 \\
15 &= 1\cdot 12 + 3\\
12 &= 4\cdot 3
\end{align}
so $(69,372) = 3$. Then the lcm is $\frac{69\cdot372}{3} = 8556$.
\item Find the gcd first:
\begin{align}
792 &= 2\cdot 275 + 242 \\
275 &= 1\cdot 242 + 33 \\
242 &= 7\cdot 33 + 11 \\
33 &= 3\cdot 11
\end{align}
so $(792,275) = 11$. Then the lcm is $\frac{792\cdot 275}{11} = 19800$.
\item Find the gcd first:
\begin{align}
11391 &= 2\cdot 5673 + 45 \\
5673 &= 126\cdot 45 + 3\\
45 &= 15\cdot 3
\end{align}
so $(11391,5673) = 3$. Then the lcm is $\frac{11391\cdot 5673}{3} = 21540381$.
\item Find the gcd first:
\begin{align}
1761 &= 1\cdot 1567 + 194 \\
1567 &= 8\cdot 194 + 15\\
194 &= 12\cdot 15 + 14\\
15 &= 1\cdot 14 + 1\\
14 &= 14\cdot 1
\end{align}
so $(1761,1567) = 1$. Then the lcm is $\frac{1761\cdot 1567}{1} = 2759487$.
\item Find the gcd first:
\begin{align}
507885 &= 8\cdot 60808 + 21421 \\
60808 &= 2\cdot 21421 + 17966 \\
21421 &= 1\cdot 17966 + 3455 \\
17966 &= 5\cdot 3455 + 691 \\
3455 &= 5\cdot 691
\end{align}
so $(507885,60808) = 691$. Then the lcm is $\frac{507885\cdot 60808}{691} = 44693880$.
\end{enumerate}
\item If $k \vert a$ and $k \vert b$ then $a = mk$ and $b = nk$ for some $m,n\in\bbz$. Then $as+bt = mks + nkt = k(ms+nt)$.
\item If $n$ is composite then it is either a power of a prime $p^n$ or its prime factorization $\prod_i p_i^{n_i}$ contains multiple primes. In the first case, $a = p$ and $b = p^{n-1}$. In the second case, $a = p_1^{n_1}$ and $b = \prod_{i>1}p_i^{n_i}$.
\item $ax + by = a\left(x_0 + \frac{b}{d}t\right) + b\left(y_0 - \frac{a}{d}t\right) = ax_0 + by_0 + a\frac{b}{d}t - b\frac{a}{d}t = N$
\item \url{https://oeis.org/A000010} lol
\item Call the set $A$. If $\abs{A}=1$, trivially that element $m$ is the minimal element. For $\abs{A}=n>1$ elements, choose one element $a$ (finite choice) and make the set $A^* = \{b \vert b\in A, b\neq a \}$. Then $A^*$ has a minimal element, call it $m_{n-1}$. Since $A$ is a set, $a \neq m_{n-1}$. Then one of them is smaller, and is the unique minimal element of $A$.
\item If $a^2 = pb^2$ then $p = \frac{a^2}{b^2} = \left(\frac{a}{b}\right)^2$. Any prime factor on the right-hand side shows up an even number of times; the only prime factor on the left-hand side shows up once (an odd number of times).
\item We want to find the exponent of $p$ in the prime factorization of $n!$, call this $f(n,p)$. For now say $p=7$, since $7$ is a small prime, but still big enough where it feels like a real prime (sorry 5). Obviously $f = 0$ for $n=1,2,3,4,5,6$, then it jumps up to 1 when we reach $n=7$. It stays at 1 until we reach $n=14$, where it jumps to 2. This suggests that
\begin{equation}
f(n,7) \sim \left\lfloor \frac{n}{7} \right\rfloor\ .
\end{equation}
This form works until we hit $n=49$, which contributes two factors of 7. In fact, every multiple of 49 contributes twice, but so far we've only counted it once. That means we have to increment $f$ by another 1 every 49 factors:
\begin{equation}
f(n,7) \sim \left\lfloor \frac{n}{7} \right\rfloor + \left\lfloor \frac{n}{49} \right\rfloor\ .
\end{equation}
The same is true for $n=343=7^3$: it's a multiple of 7, and of 49, so we've counted it twice, but we should count it three times. Then again for $7^4$, etc. In total we have (the actual answer)
\begin{equation}
f(n,7) = \left\lfloor \frac{n}{7} \right\rfloor + \left\lfloor \frac{n}{49} \right\rfloor + \left\lfloor \frac{n}{343} \right\rfloor + \ldots = \sum_{k=1}^{\infty} \left\lfloor \frac{n}{7^k} \right\rfloor\ .
\end{equation}
In practice, infinitely many of the terms being summed are zero, e.g. $n=1000$ doesn't have any contribution from multiples of $7^{1000}$. The contributions stop when
\begin{equation}
n < 7^k \implies \log n < k \log 7 \implies \frac{\log n}{\log 7} < k\ ,
\end{equation}
so an equivalent form of the sum is
\begin{equation}
f(n,7) = \sum_{k=1}^{\left\lceil \log_7 n \right\rceil} \left\lfloor \frac{n}{7^k} \right\rfloor\ .
\end{equation}
The generalization to all primes $p$ is obvious now:
\begin{equation}
f(n,p) = \sum_{k=1}^{\left\lceil \log_p n \right\rceil} \left\lfloor \frac{n}{p^k} \right\rfloor\ .
\end{equation}
\item 			
\begin{lstlisting}[language=Python]
def get_gcd(a,b):
    if b == 0:
        return 0, (0,0)
        
    if b>a:
        #have to swap the tuple order first
        ans = get_gcd(b,a)
        return ans[0], ans[1][::-1]
	
    #keep track of coefficients to get linear combination
    coeffsA = (1,0) #a = 1*a + 0*b
    coeffsB = (0,1) #b = 0*a + 1*b

    while True:
        #a = nb + r
        n = a//b
        r = a - n*b
        if r == 0:
            return b, coeffsB

        #else keep going
        tmp = coeffsB
        coeffsB = ( coeffsA[0]-n*coeffsB[0], coeffsA[1]-n*coeffsB[1] )
        coeffsA = tmp

        a = b
        b = r

print(get_gcd(20,13)) #(1, (2, -3))
print(get_gcd(69,372)) #(3, (27, -5))
print(get_gcd(792,275)) #(11, (8, -23))
print(get_gcd(11391,5673)) #(3, (-126, 253))
print(get_gcd(1761,1567)) #(1, (-105, 118))
print(get_gcd(507885,60808)) #(691, (-17, 142))
\end{lstlisting}
\item {\color{red} ?}
\item Write the prime factorization of $n$ as $\prod_i p_i^{n_i}$. Then $\phi(n) = \prod_i (p_i-1)p_i^{n_i-1}$. If $d \vert n$ then all the prime factors of $d$ are prime factors of $n$; the prime factorization of $d$ is $p_{i_1}^{m_1} \ldots p_{i_j}^{m_j}$ where $i_1 \ldots i_j$ are some labels of prime factors of $n$ and all $m_i$ are at most the corresponding $n_i$. Then $\phi(d) = \prod_k (p_{i_k}-1)p_k^{m_k-1}$. Comparing to $\phi(n)$, we see $(p_{i_k}-1)p_k^{m_k-1} \vert (p_k-1)p_k^{n_k-1}$ since $n_k \geq m_k$. This is true for each prime factor of $d$.
\end{enumerate}





\section*{\underline{0.3: $\bbz/n\bbz$ : The Integers Modulo $n$}}
\subsection*{\underline{Exercises}}
\begin{enumerate}
\item \begin{enumerate}
\item[$\bar{0}$]: $\{ \ldots, -36, -18, 0, 18, 36, \ldots \}$ 
\item[$\bar{1}$]: $\{ \ldots, -35, -17, 1, 19, 37, \ldots \}$ 
\item[$\bar{n}$]: $\{ \ldots, -36+n, -18+n, n, 18+n, 36+n, \ldots \}$ 
\end{enumerate}
\item For $a \in \bbz$, there exists unique $q,r\in \bbz$ with $r\in [0,n)$ such that $a = qn+r$. Then $a$ and $r$ are in the same equivalence class modulo $n$.
\item Write $a = \sum_i a_i10^i$. Then 
\begin{align}
a\md 9 &= \sum_i \left(a_i10^i\right)\md 9 \\
&= \sum_i \left(a_i \md 9\right) \left(10^i \md 9\right) \\
&\equiv \sum_i \left(a_i\md 9\right) \left(10\md 9\right)^i \\
&= \sum_i \left(a_i\md 9\right) \left(1\md 9\right)^i \\
&= \sum_i \left(a_i\md 9\right) \left(1\md 9\right) \\
&= \sum_i a_i\md 9
\end{align}
\item $37 \equiv 8\mod 29$, so we want $8^{100}\mod 29$. Notice that $8^2 = 64 = 58+8 \equiv 8\mod 29$ so you can keep multiplying by 8 to find $8 \equiv 8^2 \equiv 8^3 \equiv \ldots \equiv 8^{100}$.
\item We want $9^{1500}\mod 100$. Note that (trial and error) $9^10 \equiv 1\mod 100$. Then $9^{1500} = (9^{10})^{150} \equiv 1^{150}\mod 100 = 1\mod 100$.
\item $0^2 = 0$; $1^2 = 1$; $2^2 = 4 \equiv 0$; $3^2 = 9 \equiv 1$
\item Both $a^2$ and $b^2$ are 0 or 1 mod 4, so their sum is 0, 1 or 2 mod 4 (not 3).
\item The left-hand side of $a^2 + b^2 = 3c^2$ is, from above, in one of $\bar{0}, \bar{1}, \bar{2}$. Since $c^2$ is in $\bar{0}$ or $\bar{1}$, the right-hand side is in $\bar{0}$ or $\bar{3}$. For equality the only option is that $c^2$ is in $\bar{0}$ while $a^2$ and $b^2$ are either both in $\bar{0}$ or $\bar{2}$. In either case all are even; being squares, this means that $a^2, b^2, c^2$ are all divisible by 4 and in $\bar{0}$, so we can divide both sides by 4. This can be repeated ad infinitum, even though $a^2, b^2, c^2 > 0$ and there is a minimal positive integer 1.
\item $1^2 = 1$; $3^3 = 9 \equiv 1$; $5^2 = 25 \equiv 1$; $7^2 = 49 \equiv 1$; any higher odd integer is equivalent to one of these.
\item $(\bbz / n \bbz)^x = \{ \bar{a} \in \bbz / n\bbz \ \vert\  (a,n) = 1 \}$, and $\phi(n)$ is the number of $a$ on $[1,n]$ (equivalent to $[0,n-1]$ modulo $n$) such that $(a,n) = 1$.
\item If $(a,n) = 1$ and $(b,n) = 1$, then, taking the prime factorizations of all three, we see that $a$ and $n$ have no prime factors in common (likewise for $b$ and $n$). Then $ab$ has no prime factors in common with $n$, so $(ab,n) = 1$. Take everything modulo $n$ to complete the proof.
\item {\color{red} ?}
\item If $(a,n) = 1$ then there exist $x,y\in\bbz$ such that $ax+ny=1$. Take both sides modulo $n$ to get \\ $(a\mod n)(x\mod n) \equiv 1\mod n$. Then $c = x$, or any equivalent.
\item Exercise 12 shows that if $(a,n) \neq 1$, then there is no $c$ such that $ac \equiv 1\mod n$. Exercise 13 shows that if $(a,n) = 1$, then there is such a $c$. We can take $c$ to be on $[0,n-1)$ without loss of generality. Then $\{ \bar{a} \ \vert\  (a,n) = 1 \} = \{ \bar{a} \ \vert\  \exists c\text{ such that }\bar{a}\cdot\bar{c} \equiv 1\mod n \}$. For $n=12$, the first set is $\{ 1,5,7,11 \}$. To see which elements aren't invertible we have to make a times table: 
\begin{center}
\begin{tabular}{c||c|c|c|c|c|c|c|c|c|c|c|c|}
& \textbf{0} & \textbf{1} & \textbf{2} & \textbf{3} & \textbf{4} & \textbf{5} & \textbf{6} & \textbf{7} & \textbf{8} & \textbf{9} & \textbf{10} & \textbf{11} \\
\hhline{|=|=|=|=|=|=|=|=|=|=|=|=|=|}
\textbf{0} & 0 & 0 & 0 & 0 & 0 & 0 & 0 & 0 & 0 & 0 & 0 & 0 \\ \hline
\textbf{1} & 0 & {\color{red} 1} & 2 & 3 & 4 & 5 & 6 & 7 & 8 & 9 & 10 & 11 \\ \hline
\textbf{2} & 0 & 2 & 4 & 6 & 8 & 10 & 0 & 2 & 4 & 6 & 8 & 10 \\ \hline
\textbf{3} & 0 & 3 & 6 & 9 & 0 & 3 & 6 & 9 & 0 & 3 & 6 & 9 \\ \hline
\textbf{4} & 0 & 4 & 8 & 0 & 4 & 8 & 0 & 4 & 8 & 0 & 4 & 8 \\ \hline
\textbf{5} & 0 & 5 & 10 & 3 & 8 & {\color{red} 1} & 6 & 11 & 4 & 9 & 2 & 7 \\ \hline
\textbf{6} & 0 & 6 & 0 & 6 & 0 & 6 & 0 & 6 & 0 & 6 & 0 & 6 \\ \hline
\textbf{7} & 0 & 7 & 2 & 9 & 4 & 11 & 6 & {\color{red} 1} & 8 & 3 & 10 & 5 \\ \hline
\textbf{8} & 0 & 8 & 4 & 0 & 8 & 4 & 0 & 8 & 4 & 0 & 8 & 4 \\ \hline
\textbf{9} & 0 & 9 & 6 & 3 & 0 & 9 & 6 & 3 & 0 & 9 & 6 & 3 \\ \hline
\textbf{10} & 0 & 10 & 8 & 6 & 4 & 2 & 0 & 10 & 8 & 6 & 4 & 2 \\ \hline
\textbf{11} & 0 & 11 & 10 & 9 & 8 & 7 & 6 & 5 & 4 & 3 & 2 & {\color{red} 1}\\ \hline
\end{tabular}
\end{center}
\item I will be an honest person and not do this on the computer (I did 0.3.16 first):
\begin{enumerate}
\item We did this one in $0.2.1.a$, now solve for the remainder in each line to find the gcd as a linear combination of $a$ and $n$. \\
The first line gives $7 = 20 - 13$. \\
The second line gives $6 = 13 - 7 = 2\cdot 13 - 20$. \\
The third line gives $1 = 7 - 6 = -3\cdot 13 + 2\cdot 20$. \\
Now take both sides modulo 20 to get $1 \equiv -3\cdot 13 \equiv 17\cdot 13$. The inverse of 13 is 17.
\item From scratch, sadly
\begin{align}
89 &= 1\cdot 69 + 20 &&\implies 20 = 1\cdot 89 - 1\cdot 69 \\
69 &= 3\cdot 20 + 9 &&\implies 9 = 69 - 3\cdot 20 = -3\cdot 89 + 4\cdot 69 \\
20 &= 2\cdot 9 + 2 &&\implies 2 = 20 - 2\cdot 9 = 7\cdot 89 - 9\cdot 69 \\
9 &= 4\cdot 2 + 1 &&\implies 1 = 9 - 4\cdot 2 = -31\cdot 89 + 40\cdot 69 \\
2 &= 2\cdot 1
\end{align}
The inverse of 69 is 40.
\item Why don't they reuse numbers man
\begin{align}
3797 &= 2\cdot 1891 + 15 &&\implies 15 = 1\cdot 3797 - 2\cdot 1891 \\
1891 &= 126\cdot 15 + 1 &&\implies 1 = 1891 - 126\cdot 15 = -126\cdot 3797 + 253\cdot 1891 \\
15 &= 15\cdot 1
\end{align}
The inverse of 1891 is 253.
\item I refuse to write these numbers more than once so, $n = 77695236973$, $a = 6003722857$.
\begin{align}
n &= 12\cdot a + 5650562689 \text{\ (why)} &&\implies 5650562689 = n - 12a \\
a &= 1\cdot 5650562689 + 353160168 &&\implies 353160168 = a - 5650562689 = -n + 13a \\
5650562689 &= 16\cdot 353160168 + 1 &&\implies 1 = 5650562689 - 16\cdot 353160168 = 17n - 220a \\
353160168 &= 353160168 \cdot 1
\end{align}
The inverse of $a$ is $-220 = n-220 = 77695236753$.
\end{enumerate}
\item Reuse the gcd code from 0.2.9 to get inverses: if $(a,n) = 1$, then $1 = ax + by$ for some $x,y\in\bbz$, then taking both sides modulo $n$ gives $x$ as the inverse of $a$.
\begin{lstlisting}[language=Python]
def get_gcd(a,b):
    if b == 0:
        return 0, (0,0)

    if b>a:
        #have to swap the tuple order first
        ans = get_gcd(b,a)
        return ans[0], ans[1][::-1]

    coeffsA = (1,0) #a = 1*a + 0*b
    coeffsB = (0,1) #b = 0*a + 1*b

    while True:
        #a = nb + r
        n = a//b
        r = a - n*b
        if r == 0:
            return b, coeffsB

        #else keep going
        tmp = coeffsB
        coeffsB = ( coeffsA[0]-n*coeffsB[0], coeffsA[1]-n*coeffsB[1] )
        coeffsA = tmp

        a = b
        b = r

def get_inv(a,n):
    gcd, coeffs = get_gcd(a,n)
    if gcd == 1:
        #then coeffs[0]*a + coeffs[1]*n = 1
        #take mod n, coeffs[0]*a = 1
        return modn(coeffs[0],n)
    else:
        return None

def modn(a,n):
    q = a//n
    return a - n*q

def addmodn(a,b,n):
    return modn(a+b,n)
def multmodn(a,b,n):
    return modn(a*b,n)

print(modn(-11,12)) #1
print(addmodn(13,5,12)) #6
print(multmodn(7,7,12)) #1
print()
#verify the solution for 0.3.14:
for i in range(24):
    print(i, get_inv(i,12)) #None other than 1,5,7,11, which are their own inverses
print()
#verify the solution for 0.3.15:
print(get_inv(13,20)) #17
print(get_inv(69,40)) #29
print(get_inv(1891,3797)) #253
print(get_inv(6003722857,77695236973)) #77695236753
\end{lstlisting}
\end{enumerate}



\end{document}