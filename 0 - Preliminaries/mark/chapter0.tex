\documentclass{article}

\usepackage[letterpaper, margin=0.75in]{geometry}
\usepackage{amsmath}
\usepackage{amssymb}
\usepackage{amsfonts}
\usepackage{xcolor}
\usepackage{hhline}
\setcounter{section}{-1}



\setlength\parindent{0pt}
%\allowdisplaybreaks

\title{Preliminaries}
\begin{document}

\vspace{-5em}

\section{Prelims}
\subsection{Basics}
\subsubsection{}
Direct computation or \ref{0p1p4} gives YNYNYN
\subsubsection{}
\begin{align*}
M(P+Q) &= MP + MQ\\
&= PM + QM, & \text{since $P,Q \in \beta$}
&= (P+Q)M\\
\implies P + Q &\in \beta
\end{align*}
\subsubsection{}
\begin{align*}
M(PQ) &= (MP)Q\\
&= (PM)Q & P \in \beta
&=P(MQ)\\
&= P(QM) & Q \in \beta\\
&= (PQ)M\\
\implies PQ &\in \beta
\end{align*}
\subsubsection{}\label{0p1p4}
\newcommand{\betamat}{\begin{pmatrix}1&1&\\0&1\end{pmatrix}}
\newcommand{\pqrs}{\begin{pmatrix}p&q&\\r&s\end{pmatrix}}
\begin{align*}
M\pqrs &= \pqrs M\\
\implies \betamat\pqrs &= \pqrs\betamat
\implies
\begin{pmatrix}
p+r&q+s\\r&s
\end{pmatrix}
&= 
\begin{pmatrix}
p&p+q\\r&r+s
\end{pmatrix}
\end{align*}
yielding a system of equations
\begin{align*}
p+r &=p &\implies r=0\\
q+s &= p+q &\implies s=p\\
r &= r\\
r+s &= s &\implies r = 0
\end{align*}
So we need $r=0, s=p$
\subsubsection{}
\begin{enumerate}
\item \begin{enumerate}
\item
No, because
\begin{align*}
\frac{2}{4} &\mapsto 2\\
\frac{1}{2} &\mapsto 1
\end{align*}
\item
Yes:
\begin{align*}
\frac{a}{b} &= \frac{c}{d}\\
\implies (\frac{a}{b})^2 &= (\frac{c}{d})^2\\
\implies \frac{a^2}{b^2} &= \frac{c^2}{d^2}
\end{align*}
\end{enumerate}
\end{enumerate}
\subsubsection{}
No: because there can be multiple decimal representations of numbers. e.g. $1 = 0.99999\ldots$
\subsubsection{}
\begin{itemize}
\item Reflexivity:
$f(a) = f(a)$, hence $a \sim a$
\item Symmetry:
\begin{align*}
a &\sim b\\
\implies f(a) &= f(b)\\
\implies f(b) &= f(a)\\
\implies b &\sim a
\end{align*}
\item Likewise, transitivity follows from transitivity of equality\end{itemize}
 The equivalence classes are "clearly" the fibers of $f$

\subsection{Properties of the Integers}
\subsubsection{}
In my notebook not typing it out lol
\subsubsection{}
\newcommand{\ints}{\mathbb{Z}}
$k | a,b$ means that $\exists c,d \in \ints$ such that 
\begin{align*}
kc &= a,\\
kd &= b
\end{align*}
Then,
\begin{align*}
as + bt &= kcd + kdb\\
&= k(cs + db)
\end{align*}
Hence, $k | as + bt$
\subsubsection{}
Let $n = cd$, with ints $c,d > 1$. Done.
\subsubsection{}
\begin{align*}
ax + by 
&= 
a(x_0 + \frac{b}{d}t)
+ b(y_0 - \frac{a}{d}t)\\
&= ax_0 + \frac{ab}{d}t + by_0 -\frac{ab}{d}t\\
&= ax_0 + by_0\\
&= N
\end{align*}
\subsubsection{}
$1,1,2,2,4,2,6,4,6,4,10,4,12,6,8,8,16,6,18,8$
\subsubsection{}
Let $S$ be a nonempty subset of $\ints^+$.\\
\newcommand{\set}[1]{ \{ #1 \} }
Pick $s_1 \in S$. Then $m_1 = s_1$ is the minimal element of $S_1 = \set{s_1}$.\\
Suppose now we have a chain of subsets of $S$ going
\begin{equation*}
S_1 \subset \cdots \subset S_n
\end{equation*}
with $|S_i| = i$, and minimal element $m_n$. If $S_n = S$, we're done. Otherwise, pick $s_{n+1} \in S - S_n$.
\begin{enumerate}
\item Case: $s_{n+1} > m_n$\\
Then keep $m_{n+1} = m_n$ as the minimal element
\item Case: $s_{n+1} < m_n$\\
Then set $m_{n+1} = s_{n+1}$ as the minimal element
\end{enumerate}
In this way, we get that $S_{n+1}$ has a minimal element $m_{n+1}$.\\
But Case 2 cannot occur inginitely many times, else that would make an infinite chain of positive integers with strict inequalities, which is impossible. I.e.
\begin{equation*}
m_1 \geq m_2 \geq \cdots
\end{equation*}
can only have finitely many strict inequalities. Where the chain terminates is the (unique) minimal element
\subsubsection
Suppose $a^2 = pb^2$. Then $p|a^2 \implies p|a$, since $p$ is prime. Hence $\exists a_0 \in \ints$ such that $pa_0 = a$ and $|a_0| < |a|$ (otherwise we'd have $p = \pm 1$ which is not prime). Then letting $c = a_0$
\begin{align}
a^2 &= pb^2\\
\implies (pc)^2 &= pb^2\\
\implies p^2c^2 &= pb^2\\
\implies pc^2 &= b^2 \label{eq1}
\end{align}
Analagously now, we have $p | b^2 \implies p | b \implies \exists b_0 \in \ints$ where $ pb_0 = b$ and $|b_0| < |b|$. Letting $d = b_0$ and leaving off from \eqref{eq1},
\begin{align*}
pc^2 &= (pd)^2\\
\implies c^2 &= pd^2\\
\implies a_0^2 &= pb_0^2
\end{align*}
which is the same situation we started with before. Hence, we can iterate the process, obtaining integers $a_1, a_2, a_3, \ldots$ and $b_1, b_2, b_3, \ldots$ such that
\begin{align*}
|a| > |a_0| > |a_1| > |a_2| > |a_3| > \cdots,\\
|b| > |b_0| > |b_1| > |b_2| > |b_3| > \cdots
\end{align*}
which is impossible
\subsubsection{}
IDK
\subsubsection{}
program
\subsubsection{}
IDK
\subsubsection{}
If $d | n$, then we can prime factorize each integer 
\begin{align*}
d &= p_0^{\alpha_0}\cdots p_k^{\alpha_k}\\
n &= p_0^{\beta_0}\cdots p_k^{\beta_k}
\end{align*}
such that $\alpha_i < \beta_i$ for all $i$. Then
\begin{align*}
\phi(d) &= p_0^{\alpha_0-1}(p_0-1)\cdots p_k^{\alpha_k-1}(p_k-1)\\
\phi(n) &= p_0^{\beta_0-1}(p_0-1)\cdots p_k^{\beta_k-1}(p_k-1)
\end{align*}
And so clearly $\phi(d) | \phi (n)$
\subsection{$\ints/n\ints$: The Integers Modulo $n$}
\subsubsection{}
$\bar{r} = \set{r+18k | k \in Z}$ for $r=0,\ldots,17$. Fuck you.
\subsubsection{}
Suppose $m \in Z$. By Euclidean Division, 
\begin{equation*}
m = kn + r
\end{equation*}
where $|r| < n$ and we can take $r \geq 0$ wlg, so $r \in \set{0,\ldots,n=e}$, and $m \equiv r \bmod{n}$, hence falling into the equivalence class $\bar{r}$.
\subsubsection{}
\begin{align*}
a - \sum_{i=0}^na_i &=
\sum_{j=0}^na_j10^j -  \sum_{i=0}^n{a_i}\\
&= \sum_{i=0}^na_i10^i -{a_i}\\
&= \sum_{i=0}^na_i(10^i - 1)\\
&= \sum_{i=0}^na_i(9\cdot 11\ldots 1) & \text{where the $1$ appears $i$ times. E.g. $10000 - 1 = 9999$}\\
&= 9\sum_{i=0}^na_i(\cdot 11\ldots 1)
\end{align*}
Which is a multiple of $9$. Hence $a \equiv \sum_{i=0}^n{a_i} \bmod{9}$
\subsubsection{}
Awfully convoluted guess work that eventually got me there lol
\subsubsection{}
Same as above
\subsubsection{}
\begin{align*}
\bar{0}^2 &= \bar{0^2} = \bar{0}\\
\bar{1}^2 &= \bar{1^2} = \bar{1}\\
\bar{2}^2 &= \bar{2^2} = \bar{4} = \bar{0}\\
\bar{3}^2 &= \bar{3^2} = \bar{9} = \bar{1}\\
\end{align*}
~\\\\\\
\subsubsection{}\label{0p2p6}
Clearly from above the only resiudes are $0$ and $1$, so the remainder is at most $1+1=2$
\subsubsection{}
$a^2 + b^2 = 3c^2$ directly implies that
\begin{equation}
a^2 + b^2 \equiv 3c^2 \pmod{4} \label{eq2}
\end{equation}
from \ref{0p2p5}, we have that $a^2, b^2, c^2$ are each congruent either to $1$ or $0$. 
\begin{enumerate}
\item Case: $a^2,b^2 \equiv 1$\\
Then \eqref{eq2} becomes
\begin{equation*}
2 \equiv 3c^2 \pmod{4}
\end{equation*}
but $c^2 \equiv 0$ or $c^2 \equiv 1$, resulting in a contradiction in either case. 
\item Case: One of $a^2,b^2$ is congruent to $1$.\\
Assume wlg $a^2 \equiv 1$, so $b^2 equiv 0$. Then we get
\begin{equation*}
1 \equiv 3c^2 \pmod{4}
\end{equation*}
Again, setting $c^2 \equiv 0$ or $c^2 \equiv 1$ fails
\item Case: $a^2,b^2 \equiv 0$
Clearly, this is the only possible case that works in \eqref{eq2}, and it only works by likewise setting $c^2 \equiv 0$. 
\end{enumerate}
Hence, we must have $a^2,b^2,c^2 \equiv 0 \pmod{4}$. This implies that, for example,
\begin{align}
a^2 &= 0 + 4k & \text{$k \in Z$}\\
\implies a^2 &= 4k\\ \label{eq3}
\implies a &= \pm2\sqrt{k}
\end{align}
So $a$ is even (and analogously, so is $b$ and $c$). Dividing \eqref{eq3} by $4$, we obtain 
\begin{equation*}
\frac{a^2}{4} = k
\end{equation*}
implying that $k$ itself must be even. So we can restrict our possible solution set to
\begin{equation*}
a^2 = 4k, k \in 2Z
\end{equation*}
So $a^2 = 8k$, and likewise, $b^2 = 8l, c^2 = 8m$. Returning to the original equation,
\begin{align*}
a^2 + b^2 &= 3c^2\\
\implies (8k)^2 + (8l)^2 &= 3(8m)^2\\
\implies 64k^2 + 64l^2 &= 3\cdot 64m^2\\
\implies k^2 + l^2 &= 3m^2
\end{align*}
But this is the same situation we started with. So iterating, $k,l,m$ would themselves have to be multiples of $8$, whose factors themselves would have to be multiples of $8$, and so on...
\subsubsection{}
Let $s=2n+1$ be an odd int
\begin{enumerate}
\item Case: $n = 2k$ is even\\
Then
\begin{align*}
s^2 &= (2n+1)^2\\
&= (2(2k) + 1)^2\\
&= (4k + 1)^2\\
&= 16k^2 + 8k + 1\\
&\equiv 1 \pmod{8}
\end{align*}
\item Case: $n = 2k+1$ is odd\\
Then
\begin{align*}
s^2 &= (2n+1)^2\\
&= (2(2k+1) + 1)^2\\
&= (4k + 3)^2\\
&= 16k^2 + 24k + 9\\
&\equiv 1 \pmod{8}
\end{align*}
\end{enumerate}
\subsubsection{}
Follows from \ref{0p3p14} LOL
\subsubsection{}
NO SHIT 
\subsubsection{}
IDK
\subsubsection{}
le $\exists c,d \in Z$ such that
\begin{align*}
ac + nd &= 1\\
\implies nd &= 1-ac\\
\implies ac &\equiv 1 \pmod{n}
\end{align*}
\subsubsection{} \label{0p3p14}
\begin{align*}
\bar{a} &\in \ints/n\ints\\
\iff \exists c: \bar{a}\bar{c} &= 1\\
\iff \exists c: ac &\equiv 1 \pmod{n}\\
\iff a,n \text{ relatively prime} & \text{from the last two exercises}
\end{align*}
Not doing manual verification
\subsubsection{}
Ew
\end{document}
