\documentclass{article}

\usepackage[letterpaper, margin=0.75in]{geometry}
\usepackage{amsmath}
\usepackage{amssymb}
\usepackage{amsfonts}
\usepackage{xcolor}
\usepackage{hhline}
\usepackage[shortlabels]{enumitem}
\usepackage{amsthm}
\usepackage{bm}
%\setcounter{section}{-1}

\newcommand{\ints}{\mathbb{Z}}
\newcommand{\reals}{\mathbb{R}}
\newcommand{\rats}{\mathbb{Q}}
\newcommand{\comps}{\mathbb{C}}
\newcommand{\set}[1]{ \{ #1 \} }
\newcommand{\mult}{\star}
\newcommand{\abs}[1]{| #1 |}
\newcommand{\inv}[1]{ {#1}^{-1} }
\newcommand{\id}{ \bm{1} }
\renewcommand{\bar}{\overline}
\DeclareMathOperator*{\argmin}{\arg\!\min}

\newtheorem{theorem}{Theorem}[section]
\newtheorem{corollary}{Corollary}[theorem]
\newtheorem{lemma}[theorem]{Lemma} 

\setlength\parindent{0pt}
%\allowdisplaybreaks

\title{TITLE}
\begin{document}

\vspace{-5em}

\section{Introduction to Groups}
\subsection{Basic Axioms and Examples}
\subsubsection{}
\begin{enumerate}[(a)]
\item No: $a-(b-c) = a-b+c \neq (a-b)-c$
\item Yes. Note: 
\begin{align}
(a \mult b) \mult c &= (a+b+ab)\mult c\\
&= (a+b+ab) + c + (a+b+ab)c\\
&= a+b+ab+c+ac+bc+abc \label{eq1p1}
\end{align}
And
\begin{align}
a \mult (b \mult c) &= a\mult (b+c+bc)\\
&= a + (b+c+bc) + a(b+c+bc)\\
&= a+b+c+bc+ab+ac+abc \label{eq1p2}
\end{align}
Equations \eqref{eq1p1} and \eqref{eq1p2}  are clearly equal from commutativity of addition in $\reals$.
\item No. Note:
\begin{align}
	a\mult(b\mult c) &= (\frac{a+b}{5})\mult c\\
&= \frac{ \frac{a+b}{5} + c}{5}\\
&= \frac{a+b}{25} + \frac{c}{5}\\
&= \frac{a+b+5c}{25} \label{eq1p3}
\end{align}
But
\begin{align}
	a\mult(b\mult c) &= a\mult (\frac{b+c}{5})\\
&= \frac{a+\frac{b+c}{5}}{5}\\
&= \frac{a}{5} + \frac{b+c}{25}\\
&= \frac{5a+b+c}{25} \label{eq1p4}
\end{align}
Equations \eqref{eq1p3} and \eqref{eq1p4} are clearly not equal in general (e.g. set $a=0,b=0,c=1$)
\item Yes, this is associative. It is just unreduced addition in $\rats$
\item No:
\begin{align*}
(a \mult b) \mult c &= (\frac{a}{b})\mult c\\
&= \frac{ \frac{a}{b} }{c}\\ 
&= \frac{a}{bc}
\end{align*}
But
\begin{align*}
a \mult (b\mult c) &= a \mult \frac{b}{c}\\
&= \frac{a}{ \frac{b}{c} } \\
&= \frac{ac}{b}
\end{align*}
Set $a=1,b=1,c=2$ to see they are not equal
\end{enumerate}
\subsubsection{}
\begin{enumerate}[(a)]
\item No. As mentioned in the text, subtraction on $\ints$ is not commutative
\item Yes:
\begin{align*}
a\mult b &= a+b+ab\\
&= b + a +ba\\
&= b\mult a
\end{align*}
\item Yes:
\begin{align*}
a\mult b &= \frac{a+b}{5}\\
&= \frac{b+a}{5}\\
&= b\mult a
\end{align*}
\item Yes. Again, this is just unreduced addition in $\rats$. 
\item No. $a \mult b = \frac{a}{b}$ but $b \mult a = \frac{b}{a}$.
\end{enumerate}
\subsubsection{}
\begin{align*}
(\bar k + \bar l) + \bar m 
&= \bar{k+l} + \bar m\\
&= \bar{(k+l)+m}\\
&= \bar{k+(l+m)}\\
&= \bar k + \bar{l+m}\\
&= \bar k + (\bar l + \bar m)
\end{align*}
\subsubsection{}
\begin{align*}
(\bar k \cdot \bar l) \bar m 
&= \bar{kl} \cdot \bar m\\
&= \bar{(kl)m}\\
&= \bar{k(lm)}\\
&= \bar k \cdot \bar{lm}\\
&= \bar k \cdot (\bar l  \bar m)
\end{align*}
\subsubsection{}
The element $\bar 0$ doesn't have an inverse.
\subsubsection{}
In each of these, I will denote the set in question by $G$
\begin{enumerate}[(a)]
\item Yes. The three group properties are all inherited from $\rats$ (the identity and the inverses are clearly in $G$), so we just have to verify that $G$ is closed under addition. Given $\frac{a}{b}$, $\frac{c}{d} \in G$, we have
\begin{equation*}
\frac{a}{b} + \frac{c}{d} = \frac{ad+bc}{bd}
\end{equation*}
Since $b,d$ are odd, $bd$ is also odd. And since $bd$ is odd, no even number divides it. Hence, reducing the fraction maintins oddness of the denominator, and the result is still in $G$

\item No. This set isn't closed under addition: $\frac{1}{2} + \frac{3}{2} = \frac{4}{2} = \frac{2}{1}$

\item No: $\frac{2}{3} + \frac{2}{3} = \frac{4}{3}$ and $\abs{\frac{4}{3}} > 1$

\item No: $\frac{5}{3} - \frac{4}{3} = \frac{1}{3}$ and $\abs{\frac{1}{3}} < 1$

\item Yes. This is the set 
\begin{equation*}
G = \ints \cup \set{\frac{r}{2} | r \in \ints}
\end{equation*}
This set is clearly closed under addition. Associativity is inherited from $\rats$. The identity $\frac{0}{2}$ is in $G$. The inverse $\frac{-r}{2}$ of $\frac{r}{2}$ is in $G$. 

\item No. The set isn't closed under addition. $\frac{3}{2} + \frac{2}{3} = \frac{9}{6} + \frac{4}{6} = \frac{13}{6}$ which is not under $G$.
\end{enumerate}
\subsubsection{}
\begin{itemize}
\item The set $G$ is clearly closed under the operation, since we always cut off the ones place and use nonnegative numbers.
\item The operation is associative. Adding two real numbers and cutting off the ones place, then adding another one and cutting off the ones place, is the same as adding them all and THEN cutting off the ones place
\item The identity is $0$
\item The inverse of $x$ is $1-x$
\end{itemize}
\subsubsection{}
\begin{enumerate}[(a)]
\item If $z,c \in \comps$, then $\exists n,m \in \ints^+$ s.t. $z^n,c^m=1$.
And 
\begin{align*}
(zc)^{nm} &= z^{nm}c^{nm}\\
&= (z^n)^m(c^m)^n\\
&= 1^m 1^n\\
&= 1
\end{align*}
Hence, $zc \in \comps$, so $G$ is closed under complex multiplication. Associativity is inherited from associativity of complex multiplication. The identity $1^1$ i in $G$. And given $z \in G$ with $z^n =1$, $n \in \ints^+$, we have $(\frac{1}{z})^n = \frac{1^n}{z^n} = \frac{1}{1} = 1$, so the inverse of $z$ is in $G$.
\item No identity element. $0 \notin G$
\end{enumerate}
\subsubsection{}
\begin{enumerate}[(a)]

\item Note $(a + b\sqrt{2}) + (c+d\sqrt{2}) = (a+c) + (b+d)\sqrt{2} \in G$. So $G$ is closed under the operation. Associativity follows from associativity of addition in $\reals$. The identity is $0 = 0 + 0\sqrt{2}$. And the inverse of $a+b\sqrt{2}$ is $-a -b\sqrt{2}$.

\item Set $G' = G - \set{0}$. Note that
\begin{align*}
(a+b\sqrt{2})(c+d\sqrt 2) &= ac + (ad+bc)\sqrt 2 + 2bd\\
&= (ac + 2bd) + (ad+bc)\sqrt 2\\
&\in G
\end{align*}
And since in $\reals$, $xy = 0 \implies x = 0 \mbox{ or } y=0$, the result is in fact in $G-\set{0} = G'$. Hence, $G'$ is closed under multiplication. Associativity is inherited from the reals, the identity is $1 + 0\sqrt 2 = 1$, and the following demonstrates that the inverse is in $G'$:
\begin{align*}
\frac{1}{a+b\sqrt 2} &= \frac{a-b\sqrt 2}{(a+b\sqrt 2)(a-b\sqrt 2)}\\
&= \frac{a-b\sqrt 2}{a^2-2b^2}\\
&= \frac{a}{a^2-2b^2} + \frac{-b}{a^2-2b^2}\sqrt{2}
\end{align*}
\end{enumerate}
\subsubsection{}
NO SHIT. Transposing the matrix is equivalent to reversing the order of operations, and the upper triangle remains the same since the matrix is symmetric.
\subsubsection{}\label{ex11}
Process: $|\bar r| = \frac{l}{r}$ where $l = \mbox{lcm}(r,12)$. $l$ itself can be calculated by $gl = 12r$, where $g=\mbox{gcd}(r,12)$. Given that, we have,
\begin{align*}
\abs{\bar 0 } &= 1\\
\abs{\bar 1 } &= 12\\
\abs{\bar 2 } &= 6\\
\abs{\bar 3 } &= 4\\
\abs{\bar 4 } &= 3\\
\abs{\bar 5 } &= 12\\
\abs{\bar 6 } &= 2\\
\abs{\bar 7 } &= 12\\
\abs{\bar 8 } &= 3\\
\abs{\bar 9 } &= 4\\
\abs{\bar{10} } &= 6\\
\abs{\bar{11} } &= 12\\
\end{align*}
\subsubsection{}
Noting that $5^2 = 12\cdot 2 + 1$, $7^2 = 12\cdot 4+1$, $13 = 12+1$, we have
\begin{align*}
\abs{\bar 1 } &= 1\\
\abs{\bar -1 } &= 2\\
\abs{\bar 5 } &= 2\\
\abs{\bar 7 } &= 2\\
\abs{\bar{-7} } &= 2\\
\abs{\bar{13}} &= 1
\end{align*}
\subsubsection{}\label{ex13}
Same process as in \ref{ex11}, yielding
\begin{align*}
\abs{\bar 1 } &= 36\\
\abs{\bar 2 } &= 18\\
\abs{\bar 6 } &= 6\\
\abs{\bar 9 } &= 4\\
\abs{\bar{10}} &= 18\\
\abs{\bar{12} } &= 3\\
\abs{\bar{ -1} } &= 36\\
\abs{\bar{ -10 }} &= 18\\
\abs{\bar{-18 }} &= 2\\
\end{align*}
\subsubsection{}
Note that $5^3 = 36\cdot 4 + 1$, $13^3 = 36\cdot 61 + 1$, $17^2 = 36\cdot 8 +1 $, $(-13)^6 = (13^3)^2$. Then
\begin{align*}
\abs{\bar 1 } &= 1\\
\abs{\bar -1 } &= 2\\
\abs{\bar 5 } &= 3\\
\abs{\bar{ 13} } &= 3\\
\abs{\bar{ -13} } &= 6\\
\abs{\bar {17} } &= 2\\
\end{align*}
\subsubsection{}
Using generalized associativity,
\begin{align*}
(a_1a_2\cdots a_{n-1}a_n)(\inv{a_n}\inv{a_{n-1}}\cdots \inv{a_1})
&= a_1a_2\cdots a_{n-1}(a_n\inv{a_n})\inv{a_{n-1}}\cdots \inv{a_1}\\
&= a_1a_2\cdots a_{n-1}(1)\inv{a_{n-1}}\cdots \inv{a_1}\\
&= a_1a_2\cdots a_{n-1}\inv{a_{n-1}}\cdots \inv{a_1}
\end{align*}
Et cetera. You could make this an "inductive" proof but it's not necessary.
\subsubsection{}
NOOOOOOOOOOO SHIIIIIIIIIIIIIIIIIIIIIIIT
\subsubsection{}
\begin{align*}
\abs{x} &= n\\
\implies x^n &= 1\\
\implies x^n\inv{x} &= 1\cdot\inv{x}\\
\implies x^{n-1}x\inv{x} &= \inv{x}\\
\implies x^{n-1}\cdot 1 &= \inv{x}\\
\implies x^{n-1} &= \inv{x}
\end{align*}
\subsubsection{}
The proof is in the very formulation of the exercise llololololol
\subsubsection{}
(a) and (b) are trivial. For (c), 
\begin{itemize}
\item Case: $a \geq 0, b \leq 0$.\\
We treat $-b$ as a nonnegative number, and $x^ax^b = x^a\inv{(x^{-b})} = x^a(\inv{x})^{-b} = x^{a-(-b)} = x^{a+b}$.\\
Also $(x^a)^b = \inv{( (x^a)^{-b} )} = \inv{( x^{-ab}  )} = x^{ab}$. \\
The case $a \leq 0, b \geq 0$ is analgous to this one.
\item Case: $a,b \leq 0$\\
We treat both $-a,-b$ as nonnegative numbers, and $x^ax^b = \inv{(x^{-a})}\inv{(x^{-b})} = \inv{( x^{-b}x^{-a} )} = \inv{( x^{-b + -a})} = \inv{( x^{-(a+b)} )} = x^{a+b}$.\\
Also $(x^a)^b = \inv{( (\inv{x})^{-a})^{-b} )} = \inv{( (\inv{x})^{ab} )} = (\inv{x})^{-ab} = x^{ab}$
\end{itemize}
\subsubsection{}\label{ex20}
I'm going to prove a Lemma that will be useful for these exercises
\begin{lemma} \label{lemma1}
Let $|x|=n$. If $m \in \ints$, then $x^m = x^r$ for $0 \leq r < n$, $r \equiv m \pmod{n}$.
\end{lemma}
\begin{proof}
This follows directly from the division algorithm. We can write $m = dn+r$ where $r$ satisfies the inequalities and congruence above. Then
\begin{align*}
x^m &= x^{dn+r}\\
&= x^{dn}x^r\\
&= (x^n)^dx^r\\
&= 1^d x^r\\
&= x^r
\end{align*}
\end{proof}
From this we immediately obtain two corollaries:
\begin{corollary} \label{cor1p1}
$x^m = 1 \implies m$ is a multiple of $|x|$. 
\end{corollary}
\begin{corollary} \label{cor1p2}
If $|x| = n$, then $1,x,x^2,\ldots,x^{n-1}$ are distinct
\end{corollary}
Nowe we finish the exercise:\\
Let $|x| = n$. Then $(\inv{x})^n = x^{-n} - \inv{(x^n)} = \inv{1} = 1$. Hence, $|\inv{x}| \leq n$. \\
Now suppose $|\inv{x}| = m$. Then $0 < m \leq n$. But
\begin{align*}
x^m &= (\inv{(\inv{x})})^m\\
&= (\inv{x})^{-m}\\
&= \inv{((\inv{x})^m)}\\
&= \inv{1}\\
&= 1
\end{align*}
Hence, by Corollary \ref{cor1p1}, $m$ is a multiple of $n$, but $0 < m \leq n$ means that $m = n$.
\subsubsection{}
Let $n=2l+1$. Then
\begin{align*}
x^n &= 1\\
\implies x^{2l+1} &= 1\\
\implies x^{2l}x &= 1\\
\implies x^{2l} &= \inv{x}\\
\implies (x^2)^l &= \inv{x}\\
\implies (x^2)^lx^2 &= \inv{x}x^2\\
\implies (x^2)^{l+1} &= x
\end{align*}
Why did we need $G$ to be finite? Lol
\subsubsection{}
\newcommand{\gxg}{\inv{g}xg}
Let $|x| = n$. Then $(\inv{g}xg)^n = g^{-n}x^ng^n = g^{-n}g^n = 1$, so $|\gxg| \leq n$. Suppose $|\gxg| = m$, with $0 < m \leq n$. Then 
\begin{align*}
(\gxg)^m &= 1\\
\implies g^{-m}x^mg^m &= 1\\
\implies x^mg^m &= g^m\\
\implies x^m &= 1
\end{align*}
So $m$ must be a multiple of $|x|=n$, but $0 < m \leq n$ so $m=n$.\\
Now let $|ab| = n$. Then
\begin{align*}
(ab)^n &= 1\\
\iff a^nb^n &= 1\\
\iff a^n &= b^{-n}\\
\iff 1 &= a^{-n}b^{-n}\\
\iff 1 &= \inv{(a^n)}\inv{(b^n)}\\
\iff 1 &= \inv{b^na^n}\\
\iff 1 &= \inv{(ba)^n}
\end{align*}
So $|\inv{(ba)}| \leq n$. But if we follow the proof above backwards with arbitrary $n$, we see that since $|ab| = n$, $|\inv{(ba)}| = n$. But by \ref{ex20}, we get that in fact $|ba| = n$.
\subsubsection{}
\begin{align*}
x^n &= 1\\
\iff x^{st} &= 1\\
\iff (x^s)^t &= 1
\end{align*}
So $|x^s| \leq t$. Follow the proof backwards to establish that $|x^s| = t$.
\subsubsection{}\label{ex1p24} 
Clearly $(ab)^1 = a^1b^1$. 
\\
Now suppose for $n>0$, $(ab)^n = a^nb^n$. Then
\begin{align*}
(ab)^{n+1} &= (ab)^nab\\
&= a^nb^nab\\
&= a^nab^nb\\
&= a^{n+1}b^{n+1}
\end{align*}
so $(ab)^n = a^nb^n$ for $n \geq 1$. The case $n=0$ is obvious. And also if $n \geq 0$, we have
\begin{align*}
(ab)^{-n} &= \inv{((ab)^n)}\\
&= \inv{(a^nb^n)}\\
&= \inv{(b^n)}\inv{(a^n)}\\
&= b^{-n}a^{-n}\\
&= a^{-n}b^{-n}
\end{align*}
So $(ab)^n = a^nb^n$ for negative $n$ as well.
\subsubsection{}
Given $x,y\in G$,
\begin{align*}
xy &= xy1\\
&=xy(yx)^2\\
&=xyyxyx\\
&=xy^2xyx\\
&=x1xyx\\
&=xxyx\\
&=x^2yx\\
&=1yx\\
&=yx
\end{align*}
\subsubsection{}
Trivial. Associativity and identity are inherited from $G$. Closure and inverses are given in the definition.
\subsubsection{}
Let $H = \set{x^n | n \in \ints}$. Then $x^nx^m = x^{n+m} \in H$, satisfying closure. And given $x^n \in H$, $x^{-n}$ is also clearly in $H$.
\subsubsection{}
\begin{enumerate}[(a)]
\item 
\begin{align*}
(a,b)( (c,d), (e,f) ) &= (a,b)(ce,df)\\
&= (a(ce), b(df))\\
&= ((ac)e, (bd)f)\\
&= (ac,bd)(e,f)\\
&= ((a,b),(c,d))(e,f)
\end{align*}
\item Le $(a,b)(1,1) = (a\cdot 1, b\cdot 1) = (a,b) = (1\cdot a, 1\cdot b) = (1,1)(a,b)$
\item  Le $(a,b)(\inv{a}, \inv{b}) = (a\inv{a}, b\inv{b}) = (1,1) = (\inv{a}a,\inv{b}b) = (\inv{a},\inv{b})(a,b)$
\end{enumerate}
\subsubsection{}
\begin{itemize} 
\item $\implies$:
\\
Suppose $A,B$ are abelian. Then given $(a,b),(c,d) \in A \times B$, we have $(a,b)(c,d) = (ac,bd) = (ca,db) = (c,d)(a,b)$. Hence $A \times B$ is abelian.
\item $\impliedby$:
\\
Suppose $A \times B$ is abelian. Then given $a,c \in A$,
\begin{align*}
(a,1)(c,1) &= (c,1)(a,1)\\
\implies (ac, 1) &= (ca, 1)\\
\implies ac &= ca
\end{align*}
Hence, $A$ is abelian. An analagous proof shows that $B$ is also abelian.
\end{itemize}
\subsubsection{}
$(a,1)(1,b) = (a1,1b) = (a,b) = (1a,b1) = (1,b)(a,1)$.\\
Then
\begin{align*}
(a,b)^n &= (1,1)\\
\iff ((a,1)(1,b))^n &= (1,1)\\
\iff ((a,1)^n(1,b^n) &= (1,1) & \mbox{(Since they commute)}\\
\iff (a^n,1)(1,b^n) &= (1,1)\\
\iff (a^n, b^n) &= (1,1)\\
\iff a^n = 1 &, b^n =1
\end{align*}
Suppose $n$ satisfies both these equations. Then $n$ must be a multiple of both $|a|$ and $|b|$, by Corollary \ref{cor1p1}. But $n$ is an order, so it in fact must be the LEAST common multiple of $|a|$ and $|b|$.
\subsubsection{}
$g \in t(G) \implies \inv{g} \in t(G)$, where these elements are distinct. Hence, the elements of $t(G)$ appear in pairs, and $|t(G)|$ is even.\\
Suppose $x \in G-t(G)$. Then $x = \inv{x} \implies x^2 = 1$. So $|x| \leq 2$. But $|x|=1 \implies x=1$, so if $x \neq 1$ and $x \in G-t(G)$, then $|x| = 2$. But since $|t(G)|$ is even and $1 \in G-t(G)$, then $G-t(G)$ must have at least one other element for $|G|$ to be even, and this element has order $2$.
\subsubsection{}
This is Corollary \ref{cor1p2}
\subsubsection{}\label{ex1p33}
Assume $i > 0$ wlg. Note
\begin{align}
x^i &= x^{-i} \label{eq33p1} \\
\implies x^{2i} &= 1 \label{eq33p2}
\end{align}
Since the order of $x$ is $n$, we must have by Corollary \ref{cor1p1} that $2i$ is a multiple of $n$. There is some positive $d$ such that
\begin{equation*}
nd = 2i \label{eq33p3}
\end{equation*}
\begin{enumerate}[(a)]
\item If $d=1$, then $n$ is not odd, a contradiction. If $d \geq 2$, then \label{ex1p33a}
\begin{align*}
2i &= nd\\
&\geq 2n\\
\implies i \geq n
\end{align*}
contradicting the assumption that $i=1,2,\ldots,n-1$. Hence, \eqref{eq33p1} is impossible.
\item Let $n=2k$. Then \label{ex1p33b}
\begin{align*}
nd &= 2i\\
\implies 2kd &= 2i\\
\implies i &= kd
\end{align*}
\begin{itemize}
\item Case: $d = 2l+1$ is odd. Then
\begin{align*}
nd &= 2i\\
\implies nd -n &= 2i - n\\
\implies n(d -1) &= 2i - 2k\\
\implies n(2l+1 -1) &= 2(i - k)\\
\implies 2nl &= 2(i - k)\\
\implies nl &= i - k\\
\implies i \equiv k \pmod{n}
\end{align*}
\item Case: $d = 2l$ is even. Then
\begin{align*}
nd &= 2i\\
\implies n2l &= 2i\\
\implies nl &= i\\
\implies i \equiv 0 \pmod{n}
\end{align*}
\end{itemize}
Hence, this exercise is WRONG, because we can't have $k=0$. For a trivial counterexample, take $i=n$ (or any multiple of $n$). Then the exercise would tell us that $n \equiv k \pmod{n}$. But $n=2k$, so $0<k<n$, so this is ridiculous.
\end{enumerate}
\subsubsection{}
Let $n,m \in \ints$, with $n \neq m$. Then 
\begin{align*}
x^n &= x^m\\
\implies x^{n-m} &=1
\end{align*}
\begin{itemize}
\item Case: $n-m > 0$\\
Then $|x| \leq n-m < \infty$, a contradiction
\item Case: $n-m < 0$\\
Then $(x^{n-m})^{-1} = 1 \implies x^{m-n} = 1$, and $|x| \leq m-n \leq \infty$, a contradiction.
\end{itemize}
\subsubsection{}
This is Corollary \ref{cor1p2}.
\subsubsection{}
We cannot have $|x| = 1$ for $x \neq 1$. So let's consider the case $|a| = 3$. We can't have $a^2 = 1$, since this contradicts that. We can't have $a^2 = a$, or else cancellation gives $a=1$. So assume wlg $a^2 = b$. Then $b^2 = a^2a^2 = a^3a = a$. So $|b| \neq 2$, i.e. $|b| = 3$. If $b^2=a$. Then $a^2 = b \implies a^3 = ab \implies 1 = ab$, and right multiplication instead gives $1=ba$. I.e. $b=\inv{a}$. Then
\begin{align*}
c &= 1\cdot c\\
&= abc\\
\implies ac &= a^2bc\\
\implies ac &= bbc\\
\implies ac &= b^2c\\
\implies a &= ac\\
\implies c &= 1 
\end{align*}
But this reduces the order of the group. So this is a contradiction, and we cannot have $|a| = 3$. Setting $|b|=3$, $|c|=3$ of course also yields contradicions\\
In which case, we must have $|a|=2$, so $a^2=1$. We can't have $ab=a$ or $ab=b$, else we'd get $b=1$ or $a=1$. We can't have $ab=1$ or $ba=1$, else we'd get $b=\inv{a}$. But $a^2=1 \implies \inv{a}=a \implies b = a$, reducing the order of the group. So we must have $ab=c$ and $ba=c$. Similarly, we are forced to set $|b|=2, |c|=2$, which forces $bc=cb=a$ and $ca=ac=b$, respectively. 
\subsection{Dihedral Groups}
\subsubsection{}\label{ex2p1}
Note that for any $k$,
\begin{align*}
(sr^k)^2 &= sr^ksr^k\\
&= ssr^{-k}r^k\\
&= s^2\\
&= 1
\end{align*}
So $|sr^{k}| = 2$. On the other hand, the elements $r^k$ form a cyclic subgroup of order $n$ which can be identified with the group $\ints/n\ints$. Hence, to compute the order of each of the elements, we follow the same method as in exercises \ref{ex11}, \ref{ex13}. Hence...
\begin{enumerate}[(a)]
\item $n=3$
\begin{align*}
|1| &= 1\\
|r| &= 3\\
|r^2| &= 3\\
|sr^k| &= 2
\end{align*}
\item $n=4$
\begin{align*}
|1| &= 1\\
|r| &= 4\\
|r^2| &= 2\\
|r^3| &= 4\\
|sr^k| &= 2
\end{align*}
\item $n=5$
\begin{align*}
|1| &= 1\\
|r| &= 5\\
|r^2| &= 5\\
|r^3| &= 5\\
|r^4| &= 5\\
|sr^k| &= 2
\end{align*}
\end{enumerate}
\subsubsection{}
$x = sr^k$. So $rx = rsr^k = s\inv{r}r^k = sr^{k-1} = sr^k\inv{r} = x\inv{r}$
\subsubsection{}
$|sr^k| = 2$ was shown in \ref{ex2p1}. And note that\\
$r^k = 1r^k = 1^kr^k = (1r)^k = (s^2r)^k = (s(sr))^k$,\\
and $sr^k = s(s(sr))^k$
\subsubsection{}
Using the method of \ref{ex11} again, $\mbox{lcm}(k,2k) = 2k$, so $|z|=|r^k|=\frac{2k}{k} = 2$. $z$ obviously commutes with the $r^i$. And
\begin{align*}
z(sr^i) &= r^ksr^i\\
&= sr^{-k}r^i\\
&= (sr^i)r^{-k}\\
&= (sr^i)r^{n-k}\\
&= (sr^i)r^{2k-k}\\
&= (sr^i)r^k\\
&= (sr^i)z
\end{align*}
Note:
\begin{align*}
(sr^i)(sr^j) &= sr^isr^j\\
&= ssr^{-i}r^{j}\\
&= s^2r^{j-i}\\
&= r^{j-i}
\end{align*}
Whereas $(sr^j)(sr^i) = r^{i-j} =r^{-(j-i)}$, which according to \ref{ex1p33b} is equal to $r^{i-j}$ iff $i-j =k$, which is not the case for arbitrary $i,j$. Hence, the elements $sr^i$ do not commute in general.\\
On the other hand, the $r^i$ commute with each other. And 
\begin{align*}
(r^i)(sr^j) &= r^isr^j\\
&= sr^{-i}r^j\\
&= sr^{j-i}
\end{align*}
Whereas $(sr^j)r^i = sr^{j+i}$. And 
\begin{align*}
sr^{j+i} &= sr^{j-i}\\
\implies r^{j+i} &= r^{j-i}
\end{align*}
In particular, the above equality has to apply to $j=n$. The only possible setting of $0 \leq i < n$ that fulfills this (other than $i=0$, which gives the identity) is $i=k$
\subsubsection{}
Follow the previous exercise and note that we can't set $i=n/2$
\subsubsection{}
Note: $t(yx) = (xy)(yx) = x(yy)x = xx = 1$, so $\inv{t} = yx$,\\
and $tx = xyx = x\inv{t}$.
\subsubsection{}
\begin{itemize}
\item The old relations come from the new. Note that $\inv{b} = \inv{r}\inv{s}$, and $b^2 = 1 \implies b = \inv{b}$. So we have\\
$s^2 = a^2 = 1$\\
$rs = 1rs = s^2rs = ssrs = aba = a\inv{b}a = s\inv{r}\inv{s} s = s\inv{r}$\\
$r^n = (1r)^n = (s^2r)^n = (s(sr))^n = (ab)^n = 1$
\item The new relations come from the old. We have\\
$a^2 = s^2 = 1$\\
$b^2 = (sr)^2 = srsr = ss\inv{r}r = 1\cdot 1 = 1$\\
$(ab)^n =(s(sr))^n = (s^2r)^n = (1r)^n = r^n = 1$
\end{itemize}
\subsubsection{}
$n$. lol
\subsubsection{}
For these next 5 exercises, we note that the order is the amount of rotations a face can undergo, times the number of faces the face can be sent to. I.e. the formula is
\begin{equation*}
\mbox{Num vertices on a face} \times \mbox{Num faces}
\end{equation*}
So for a tetrahedron, it's $3\cdot 4=12$
\subsubsection{}
$4\cdot6 = 24$
\subsubsection{}
$3\cdot 8 = 24$
\subsubsection{}
$5\cdot 12 = 60$
\subsubsection{}
$3\cdot 20 = 60$
\subsubsection{}
$\ints = <1>$
\subsubsection{}
$\ints/n\ints = < \bar{1} | \bar{n} = \bar{1} >$
\subsubsection{}
$x_1^2 = y_1^2 = 1$ directly corresponds to $r^n = s^2 = 1$. And
\begin{align*}
(x_1y_1)^2 &= 1\\
\iff x_1y_1x_1y_1 &= 1\\
\iff x_1y_1 &= \inv{y_1}\inv{x_1}\\
\iff x_1y_1 &= y_1\inv{x_1}
\end{align*}
\subsubsection{}
$X_{2n} = <x,y | x^n = y^2 = 1, xy=yx^2>$
Recall from the text that $x^3 = 1$
\begin{enumerate}[(a)]
\item $x^3 = 1 \implies x^2 x = 1 \implies \inv{x} = x^2$. So $xy=yx^2 \iff xy = y\inv{x}$ corresponds to $rs = s\inv{r}$. Also, $x^n =y^2=1$ corresponds to $ r^n =s^2=1$, and thus in this case $X_{2n} = D_6$, so the order of course is $6$.
\item Use \ref{cor1p2} and note that both $3$ and $n$ have to share a divisor which is the order of $x$. But $(3,n) = 1$ implies the order is $1$.
\end{enumerate}
\subsubsection{}
$Y = <u,v| u^4 = v^3 = 1, uv = v^2u^2>$
\begin{enumerate}[(a)]
\item $v^3 = 1 \implies \inv{v} = v^2$ \label{2p18a}
\item \label{2p18b}
\begin{align*}
v^2u^3v &= (v^2u^2)(uv)\\
&= (uv)(v^2u^2)\\
&= uv^3u^2\\
&= u1u^2\\
&= u^3
\end{align*}
So
\begin{align*}
v^2u^3v &= u^3\\
\implies \inv{v}u^3v &= u^3\\
\implies u^3v &= vu^3
\end{align*}
\item \label{2p18c}
$u^9 = u^4u^4u = 1\cdot1\cdot u = u$.\\
$vu = vu^9 = u^9v = u^8uv = 1uv = uv$
\item \label{2p18d}
\begin{align*}
uv &= v^2u^2\\
\implies uv &= (uv)^2 &  \mbox{Commutativity of $u,v$}\\
\implies 1 &= uv
\end{align*}
\item \label{2p18e}
\begin{align*}
u^4v^3 &= 1\\
\implies uu^3v^3 &= 1\\
\implies u(uv)^3 &= 1 & \mbox{Commutativity}\\
\implies u1 &= 1\\
\implies u &= 1
\end{align*}
And thus also $uv = 1 \implies v = 1$
\end{enumerate}
\subsection{Symmetric Groups}
In these exercises, I will use exercises 10, 13-15 throughout without explicit reference
\subsubsection{}
\begin{itemize}
\item $\sigma = (1\ 3\ 5)(2\ 4)$
\item $\tau = (1\ 5)(2\ 3)$
\item $\sigma^2 = (1\ 5\ 3)$
\item $\sigma\tau = (2\ 5\ 3\ 4)$
\item $\tau\sigma = (1\ 2\ 4\ 3)$
\item $\tau^2\sigma = \id\sigma = \sigma$
\end{itemize}
\subsubsection{}
No thanks
\subsubsection{}
Only doing it for exercise 1, using exercise \ref{ex3p15}.
\begin{itemize}
\item $|\sigma| = 6$
\item $|\tau| = 2$
\item $|\sigma^2| = 3$
\item $|\sigma\tau| = 4$
\item $|\tau\sigma| = 4$
\item $|\tau^2\sigma| = 6$
\end{itemize}
\subsubsection{}\label{ex3p4}
\begin{enumerate}[(a)]
\item $S_3 = \set{\id, (1\ 2), (2\ 3), (1\ 3), (1\ 2\ 3), (1\ 3\ 2)}$\\
The orders are, respectively, $1,2,2,2,3,3$
\item $S_4 = \set{\id, (1\ 2), (2\ 3), (1\ 3), (1\ 4), (2\ 4), (3\ 4), 
(1\ 2)(3\ 4), (1\ 3)(2\ 4), (1\ 4)(2\ 3),
(1\ 2\ 3), (1\ 3\ 2), (1\ 3\ 4), (1\ 4\ 3), (2\ 3\ 4), (2\ 4\ 3), (1\ 2\ 4), (1\ 4\ 2),
(1\ 2\ 3\ 4), (1\ 3\ 2\ 4), (1\ 4\ 2\ 3), (1\ 2\ 4\ 3), (1\ 3\ 4\ 2), (1\ 4\ 3\ 2)}$\\
The order of $1$ is $1$, the order of the elements with $2$-cycles are $2$, the order of the $3$ and $4$ cycles are $3$ and $4$ respectively.
\end{enumerate}
\subsubsection{}\label{ex3p5}
Using \ref{ex3p15}, $|\sigma| = \mbox{lcm}(2,3,5) = 30$
\subsubsection{}\label{ex3p6}
Done in \ref{ex3p4}
\subsubsection{}\label{ex3p7}
Done in \ref{ex3p4}
\subsubsection{}\label{ex3p8}
$\infty! = \infty$ jk. Consider the permutation $\sigma$ that shifts each element to the right by one: $1 \mapsto 2, 2\mapsto 3$, etc. Then $\sigma^n$ is distinct for all $n \geq 0$
\subsubsection{}\label{ex3p9}
We use \ref{ex3p11}.
\begin{enumerate}[(a)]
\item $1, 5, 7, 11$
\item $3, 5, 7$
\item $3, 5, 9, 11, 13$
\end{enumerate}
\subsubsection{}\label{ex3p10}
Of course, $\sigma^0(a_k) = \id(a_k) = a_k = a_{k+0}$\\
Now suppose $\sigma^i(a_k) = a_{k+i}$.\\
Then $\sigma^{i+1}(a_k) = \sigma\sigma^i{a_k} = \sigma(a_{k+i}) = a_{k+i+1}$, where we implicitly replace the subscripts with their residues.\\
Hence, $\sigma^i(a_k) = a_{k+i}$ holds in general.\\
For $1 \leq i < m$,\\
$\sigma^i(a_k) = a_{k+i} \neq a_k = \id(a_k)$, so $\sigma^i \neq \id$.\\
But $\sigma^m(a_k) = a_{k+m} = a_k = \id(a_k)$, hence $\sigma^m = \id$ and $|\sigma| = m$
\subsubsection{}\label{ex3p11}
For convenience, relabel the elements $0, \ldots, m-1$, so
\begin{equation*}
\sigma = (0\ 1\ 2\ \ldots\ m-1)
\end{equation*}
and by the previous exercise \ref{ex3p10}, 
\begin{equation*}
\sigma^i(n) = n+i \pmod{m}
\end{equation*}
Then
\begin{align}
& \sigma^i \mbox{an $m$-cycle} \\
\iff m &= \argmin_k (\sigma^i)^k(0) = 0\\
\iff m &= \argmin_k \sigma^{ik}(0) = 0\\
\iff m &= \argmin_k 0 + ik \mod{m} = 0\\
\iff m &= \argmin_k 0 + ik = 0 \pmod{m}\\
\iff m &= \argmin_k m | ik\\ \label{athing1}
\iff m &= \mbox{lcm}(i,m)/i\\ \label{athing2}
\iff mi &= \mbox{lcm}(i,m)\\
\iff \mbox{gcd}(i,m) &= 1 \label{athing3}
\end{align}
\subsubsection{}\label{ex3p12}
\begin{enumerate}[(a)]
\item Set $\sigma = (1\ 3\ 5\ 7\ 9\ 2\ 4\ 6\ 8\ 10)$.\\
Using \ref{ex3p10}, I placed each elem $5$ spaces away.
\item idk
\end{enumerate}
\subsubsection{}\label{ex3p13}
The "if" is obvious, so we handle the "only if"\\
Consider $\sigma \in S_n$ with order $2$.\\
given $s \in \set{1,\ldots,n}$.\\
\begin{itemize}
\item Case: $\sigma(s) = s$\\
Then $s$ is in a $1$-cycle in $\sigma$
\item Case: $\sigma(s) = t (t \neq s)$\\
Then since $|\sigma| =2$, 
\begin{align*}
\sigma^2(s) &= s\\
\implies \sigma(\sigma(s)) &= s\\
\implies \sigma(t) &= s
\end{align*}
So $(s\ t)$ makes a $2$-cycle in $\sigma$
\end{itemize}
Hence, $\sigma$ has only $1$ and $2$-cycles
\subsubsection{}\label{ex3p14}
The "if" is again obvious, so we handle the "only if"\\
Consider $\sigma \in S_n$ with order $p > 2$.\\
given $s \in \set{1,\ldots,n}$.\\
\begin{itemize}
\item Case: $\sigma(s) = s$\\
Then $s$ is in a $1$-cycle in $\sigma$
\item Case: $\sigma(s) = t (t \neq s)$\\
Then we know at least that $\sigma^p(s) = s$.\\
Suppose $s$ is in an $m$-cycle with $1 < m < p$.\\
Then $\sigma^m(s) = s$.\\
Let $p = mq  + r$ by Euclidean division. Note $0 < r < m$ (if $r=0$, then $m|p$, contradicting primeness of $p$). Then
\begin{align*}
s &= \sigma^{p}(s)\\
&= \sigma^{mq+r}(s)\\
&= \sigma^r\sigma^{mq}(s)\\
&= \sigma^r(\sigma^m)^q(s)\\
&= \sigma^r(\id)^q(s)\\
&= \sigma^r\id(s)\\
&= \sigma^r(s)\\
\implies \sigma^r(s) &= s
\end{align*}
So $s$ is actually in an $r$-cycle, a contradiction.
\end{itemize}
A counterexample is any product of cycles of different length
\subsubsection{}\label{ex3p15}
Exercise \ref{ex3p10} immediately implies that a $m$-cycle has an order of $m$, while a combination of Corollary \ref{cor1p1} and exercise \ref{ex1p24} show that it's the lcm when we have a product of cycles
\subsubsection{}\label{ex3p16}
"Count the number of ways of forming an $m$-cycle and divide by the number of representations of a particular $m$-cycle" is the solution lol.
\subsubsection{}\label{ex3p17}
Analgous to above, we count the number of ways of ordering $4$ elements, then divide by $2$ since we don't care about the order of the cycles, then divide by $2$ again since we don't care about the order of the elements in the first cycle, then $2$ again for the second cycle.
\subsubsection{}\label{ex3p18}
How many ways are there to add up to $5$?
\begin{itemize}
\item lcm$(1,1,1,1,1) = 1$
\item lcm$(2,1,1,1) = 2$
\item lcm$(2,2,1) = 2$
\item lcm$(2,3) = 6$
\item lcm$(3,1,1) = 3$
\item lcm$(4,1) = 4$
\end{itemize}
\subsubsection{}\label{ex3p19}
How many ways are there to add up to $7$?
\begin{itemize}
\item lcm$(1,1,1,1,1,1,1) = 1$
\item lcm$(2,1,1,1,1,1) = 2$
\item lcm$(2,2,1,1,1) = 2$
\item lcm$(2,2,2,1) = 2$
\item lcm$(3,2,2) = 6$
\item lcm$(3,1,1,1,1) = 3$
\item lcm$(3,2,1,1) = 6$
\item lcm$(4,2,1) = 4$
\item lcm$(4,1,1,1) = 4$
\item lcm$(4,3) = 12$
\end{itemize}
\subsubsection{}\label{ex3p20}
$S_3 = <(1\ 2\ 3), (1\ 2)>$.\\ Verify:\\
$(1\ 2\ 3)^2 = (1\ 3\ 2)$\\
$(1\ 2\ 3)(1\ 3\ 2) = \id$\\
$(1\ 2\ 3)(1\ 2) = (1\ 3)$\\
$(1\ 2\ 3)(1\ 3) = (2\ 3)$
\end{document}
