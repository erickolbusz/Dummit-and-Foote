\documentclass[]{article}

\usepackage[letterpaper, margin=0.75in]{geometry}
\usepackage{amsmath}
\usepackage{amssymb}
\usepackage{amsfonts}
\usepackage{xcolor}
\usepackage{hhline}
\usepackage{hyperref}
\usepackage{tikz}

\newcommand{\abs}[1]{\left\vert #1 \right\vert}
\newcommand{\md}{\,\text{mod}\,}
\newcommand{\bbz}{\mathbb{Z}}
\newcommand{\bbq}{\mathbb{Q}}
\newcommand{\bbr}{\mathbb{R}}
\newcommand{\bbc}{\mathbb{C}}
\newcommand{\bbn}{\mathbb{N}}
\newcommand{\bbf}{\mathbb{F}}

\usepackage{pgffor}
\newcommand*{\cycle}[1]{( \foreach \entry [count=\i] in {#1} {\ifnum\i>1\ \fi\entry})}

\setlength\parindent{0pt}
%\allowdisplaybreaks

\usepackage{graphicx}
\usepackage{listings}

\graphicspath{ {./py/img/} }

\definecolor{codegreen}{rgb}{0,0.6,0}
\definecolor{codegray}{rgb}{0.5,0.5,0.5}
\definecolor{codepurple}{rgb}{0.58,0,0.82}
\definecolor{backcolour}{rgb}{0.95,0.95,0.92}

\lstdefinestyle{mystyle}{
	backgroundcolor=\color{backcolour},   
	commentstyle=\color{codegreen},
	keywordstyle=\color{orange},
	numberstyle=\tiny\color{codegray},
	stringstyle=\color{codepurple},
	basicstyle=\ttfamily\footnotesize,
	breakatwhitespace=false,         
	breaklines=true,                 
	captionpos=b,                    
	keepspaces=true,                 
	numbers=left,                    
	numbersep=5pt,                  
	showspaces=false,                
	showstringspaces=false,
	showtabs=false,                  
	tabsize=2
}

\lstset{style=mystyle}



\title{Introduction to Groups}
\author{}
\date{}
\begin{document}

\maketitle
\vspace{-5em}

\tableofcontents

\addcontentsline{toc}{section}{1.1: Basic Axioms and Examples}
\section*{\underline{1.1: Basic Axioms and Examples}}
\addcontentsline{toc}{subsection}{Exercises}
\subsection*{\underline{Exercises}}
\begin{enumerate}

\item \begin{enumerate}
\item no: $a-(b-c) = a-b+c \neq (a-b)-c$
\item yes: \begin{align}
(a\star b)\star c &= (a + b + ab)\star c = a + b + ab + c + (a+b+ab)c = a + b + c + ab + ac + bc + abc \\
a \star (b\star c) &= a\star (b + c + bc) = a + b + c + bc + a(b + c + bc) = a + b + c + bc + ab + ac + abc
\end{align}
\item no: \begin{align}
(a\star b)\star c &= \frac{a+b}{5}\star c = \frac{\frac{a+b}{5} + c}{5} = \frac{a+b+5c}{25} \\
a \star (b\star c) &= a\star \frac{b+c}{5} = \frac{a + \frac{b+c}{5}}{5} = \frac{5a+b+c}{25}
\end{align}
\item yes: \begin{align}
((a,b)\star (c,d))\star (e,f) &= (ad+bc,bd)\star (e,f) = ((ad+bc)\cdot f + bd\cdot e, bd\cdot f) = (adf+bcf+bde, bdf) \\
(a,b)\star((c,d)\star (e,f)) &= (a,b)\star (cf+de,df) = (a\cdot df + b\cdot (cf+de), b\cdot df) = (adf + bcf + bde,bdf)
\end{align}
\item no: \begin{align}
(a\star b)\star c &= \frac{a}{b}\star c = \frac{a}{bc} \\
a\star (b\star c) &= a\star \frac{b}{c} = \frac{a}{\frac{b}{c}} = \frac{ac}{b}
\end{align}
\end{enumerate}


\item \begin{enumerate}
\item no: $a\star b = a - b \neq b - a = b\star a$
\item yes: $a\star b = a + b + ab = b + a + ba = b\star a$
\item yes: $a\star b = \frac{a+b}{5} = \frac{b+a}{5} = b\star a$
\item yes: $(a,b)\star (c,d) = (ad+bc,bd) = (cb+da,db) = (c,d)\star (a,b)$
\item no: $a\star b = \frac{a}{b} \neq \frac{b}{a} = b\star a$
\end{enumerate}


\item I usually don't distinguish between $a$ and $\bar{a}$ but here I will. This is basically just spamming modulo $n$ (since applying it once is the same as applying it e.g. ten times), and then using the associativity of addition.
\begin{align}
(\bar{a} + \bar{b}) + \bar{c} &= \left( (a+b)\md n + c \right) \md n \\
&= \left( (a+b)\md n + c\md n \right) \md n \\
&= \left( a\md n + (b+c)\md n \right) \md n \\
&= \left( a + (b+c)\md n \right) \md n \\
&= (\bar{a} + \bar{b}) + \bar{c}
\end{align}


\item This is identical to the above with all $+$'s replaced with $\cdot$'s.


\item We just showed it was associative, we know that $\bar{1}$ is the identity, so we need to show that not every element has an inverse. For $n>1$, clearly $\bar{0}$ has no inverse since $\bar{0}\cdot\bar{a} = \bar{a}\cdot\bar{0} = \bar{0} \neq \bar{1}$ for all $\bar{a}$. For $n=1$, $\bar{0} = \bar{1}$ is the only element.


\item Addition on the reals is obviously associative, and all of these examples contain the additive identity $0$, so we just need to check closure and inverses.\begin{enumerate}
\item \underline{Closure}: {\color{red} idk} \\
\underline{Inverse}: For any $\frac{a}{2n+1}$ in this set, $\frac{(-a)}{2n+1}$ is also in the set; the two add to zero.
\item no closure: $\frac{1}{2} + \frac{1}{2} = 1 = \frac{1}{1}$ 
\item no closure: $\frac{1}{2} + \frac{1}{2} = 1$ again
\item no closure: $-\frac{3}{2} + 1 = -\frac{1}{2}$ (can't reuse the same example a third time sadly)
\item \underline{Closure}: A (reduced) rational number with a denominator of 1 can be written with a denominator of 2: $\frac{a}{1} = \frac{2a}{2}, a\in\bbz$. A (reduced) rational number with a denominator of 2 must have an odd numerator, since if it didn't then we could divide both top and bottom by 2; so these fractions are of the form $\frac{2b+1}{2}, b\in\bbz$. Now just following the rules of adding even and odd numbers (in the numerators) we see that this set is closed under addition: adding two reduced rational numbers with denominator 1, or adding two numbers with a denominator 2, yields a sum with denominator 1; adding a denominator 1 with a denominator 2 gives an denominator 2. \\
\underline{Inverse}: The inverse of $\frac{a}{1}$ is $\frac{(-a)}{1}$; likewise for denominator 2.
\item no closure: $\frac{1}{2} + \frac{1}{3} = \frac{5}{6}$.
\end{enumerate}


\item For $x,y\in G$, $0 \leq x,y < 1$ so $0 \leq x+y < 2$. If $0\leq x+y < 1$ then $\lfloor x+y \rfloor = 0$ and $x\star y = x+y$. If $1\leq x+y < 2$ then $\lfloor x+y \rfloor = 1$ and $x\star y = x+y-1$. These two cases cover all possibilities.
\begin{enumerate}
\item \underline{Closure}: From the above, if $0\leq x+y < 1$ then $x\star y = x+y$ so $0\leq x\star y < 1$ as required. If $1\leq x+y < 2$, then $x\star y = x+y-1$ or $x\star y + 1 = x+y$ so $1\leq (x\star y + 1) < 2$ which means $0\leq x\star y < 1$ as required. \\
\underline{Associativity}: Follow from associativity of addition over $\bbr$.\\ 
\underline{Identity}: The additive identity of addition (zero) is in $G$.\\
\underline{Inverse}: For $x \in G$, the inverse is $x^{-1} = 1-x$ since $x+x^{-1} = x + (1-x) = 1 \equiv 0$. The exception here is that zero is its own inverse; these two rules cover all elements of $G$.\\
\underline{Commutativity}: $x\star y = x + y - \lfloor x+y \rfloor = y + x - \lfloor y+x \rfloor = y\star x$.
\end{enumerate}


\item \begin{enumerate}
\item \underline{Closure}: If $z_1^n = z_2^n = 1$, then $(z_1 z_2)^n = 1$. \\
\underline{Associativity}: Follows from associativity of multiplication over $\bbc$. \\ 
\underline{Identity}: $1^n = 1$ so $1 \in G$. \\
\underline{Inverse}: We want to show that the obvious candidate $z^{-1} = \frac{1}{z}$ is the inverse of $z$ where $z^n = 1$. Clearly $z \cdot z^{-1} = 1$, so we just need to check that $z^{-1} \in G$. This follows from $\left(z^{-1}\right)^ n = \left(\frac{1}{z}\right)^n = \frac{1}{z^n} = 1$.
\item Writing each $z \in G$ in polar form, we see that $\abs{z} = 1$. Clearly $1 \in G$ for all $n$; but $1+1 = 2$ has absolute value 2 and hence is not in $G$, so the operation of addition is not closed.
\end{enumerate}


\item \begin{enumerate}
\item \underline{Closure}: The addition of two generic elements is $(a + b\sqrt{2}) + (c + d\sqrt{2}) = (a+c) + (b+d)\sqrt{2}$. There is no weird edge case where maybe something cancels because $\sqrt{2} \notin \bbq$ so the two terms are guaranteed to stay separate. \\
\underline{Associativity}: Follows from associativity of addition for $\bbq$. \\
\underline{Identity}: $a + b\sqrt{2}$ with $a,b=0$ gives the additive identity. \\
\underline{Inverse}: For $a+b\sqrt{2} \in G$, $(-a) + (-b)\sqrt{2} \in G$; the two add to the identity of zero.
\item \underline{Closure}: The multiplcation of two generic elements is $(a + b\sqrt{2})(c + d\sqrt{2}) = (ac+2bd) + (ad+bc)\sqrt{2}$. Since $a,b,c,d\in\bbq$, $ac+2bd\in\bbq$ and $ad+bc\in\bbq$ so the product is still in $G$.\\
\underline{Associativity}: Follows from associativity of multiplication for $\bbr$. \\
\underline{Identity}: $a + b\sqrt{2}$ with $a=1,b=0$ gives the multiplicative identity. \\
\underline{Inverse}: For $a+b\sqrt{2}$, we can define the number $\frac{1}{a+b\sqrt{2}}$ since $0 \notin G$. Now we massage:
\begin{equation}
\frac{1}{a+b\sqrt{2}} = \frac{a- b\sqrt{2}}{a^2-2b^2} = \frac{a}{a^2-2b^2} + \frac{(-b)}{a^2-2b^2}\cdot\sqrt{2}\ .
\end{equation}
Since $a,b\in\bbq$, both $\frac{a}{a^2-2b^2}$ and $\frac{(-b)}{a^2-2b^2}$ are also in $\bbq$, so the inverse is in $G$.
\end{enumerate}


\item Label the elements of $G$ as $i_1, i_2, \ldots, i_{\abs{G}}$, and denote the matrix of the multiplication as $M$ (so the product $i_j\cdot i_k$ is in $M_{jk}$). If $G$ is abelian then $i_j i_k$ = $i_k i_j$ for all $j,k$, which means $M_{jk} = M_{kj}$ for all $j,k$. Likewise if $M_{jk} = M_{kj}$ for all $j,k$, then $i_j i_k$ = $i_k i_j$ for all $j,k$.


\item This question asks to find the smallest $k$ such that $ka \equiv 0\md{12}$. Then $ka = \text{lcm}(a,12)$. From the relation $\text{lcm}(a,b) \cdot (a,b) = ab$, we see $k = \frac{\text{lcm}(a,12)}{a} = \frac{12}{(a,12)}$.\begin{enumerate}
\item[$\bar{0}$]: order 1 (identity)
\item[$\bar{1}$]: order 12
\item[$\bar{2}$]: order 6
\item[$\bar{3}$]: order 4
\item[$\bar{4}$]: order 3
\item[$\bar{5}$]: order 12
\item[$\bar{6}$]: order 2
\item[$\bar{7}$]: order 12
\item[$\bar{8}$]: order 3
\item[$\bar{9}$]: order 4
\item[$\bar{10}$]: order 6
\item[$\bar{11}$]: order 12
\end{enumerate}


\item \begin{enumerate}
\item[$\bar{1}$]: order 1
\item[$-\bar{1}$]: $-1\cdot -1 = 1$; order 2
\item[$\bar{5}$]: $5\cdot 5 = 25 \equiv 1$; order 2
\item[$\bar{7}$]: $7\cdot 7 = 49 \equiv 1$; order 2
\item[$-\bar{7}$]: $-7 \equiv 5$; order 2
\item[$\bar{13}$]: $13 \equiv 1$; order 1
\end{enumerate}


\item Again, the order is $\frac{36}{(a,36)}$, unless of course $a=0$.\begin{enumerate}
\item[$\bar{1}$]: order 36
\item[$\bar{2}$]: order 18
\item[$\bar{6}$]: order 6
\item[$\bar{9}$]: order 4
\item[$\bar{10}$]: order 18
\item[$\bar{12}$]: order 3
\item[$-\bar{1}$]: $-1 \equiv 35$ so $(35,36) = 1$; order 36
\item[$-\bar{10}$]: $-10 \equiv 26$ so $(26,36) = 2$; order 18
\item[$-\bar{18}$]: $-18 \equiv 18$ so $(18,36) = 18$; order 2
\end{enumerate}


\item \begin{enumerate}
\item[$\bar{1}$]: order 1
\item[$-\bar{1}$]: $-1\cdot -1 = 1$; order 2
\item[$\bar{5}$]: $5^2 = 25 \rightarrow 25\cdot 5 = 125 \equiv 17 \rightarrow 17\cdot 5 = 85 \equiv 13 \rightarrow 13\cdot 5 = 65 \equiv 29 \rightarrow 29\cdot 5 = 145 \equiv 1$; order 6
\item[$\bar{13}$]: $13^2 = 169 \equiv 25 \rightarrow 25\cdot 13 = 325 \equiv 1$; order 3
\item[$-\bar{13}$]: from above, $13^3 \equiv 1$, so $(-13)^3 \equiv -1$. Then $(-13)^6 \equiv -1\cdot -1 = 1$; order 6
\item[$\bar{17}$]: $17^2 = 289 \equiv 1$; order 2 (thank you)
\end{enumerate}


\item For $n=1$ the equality is trivial. For $n=2$ we want the inverse of $(a_1a_2)$. Call it $x$. Then
\begin{align}
(a_1a_2)x &= 1 \\
a_2 x &= a_1^{-1} \\
x &= a_2^{-1} a_1^{-1}\ .
\end{align}
Now we want the inverse of $(a_1\ldots a_n)$, and we know the inverse of $(a_1\ldots a_{n-1})$ is $a_{n-1}^{-1}\ldots a_1^{-1}$. Call the total inverse $x$ again.
\begin{align}
(a_1\ldots a_{n-1}a_n)x &= 1 \\
(a_1\ldots a_{n-1})a_nx &= 1 \\
a_nx &= (a_1\ldots a_{n-1})^{-1} \\
a_nx &= a_{n-1}^{-1}\ldots a_1^{-1} \\
x &= a_n^{-1}\cdot a_{n-1}^{-1}\ldots a_1^{-1}
\end{align}


\item If $\abs{x} = 1$ then $x^1 = x = 1$, so $x^2 = 1\cdot 1 = 1$. If $\abs{x} = 2$ then by definition $x^2 = 1$. For the other direction, if $x^2=1$, then $\abs{x}$ is at most 2 since $\abs{x}$ is by definition the smallest power $n$ such that $x^n=1$. If $\abs{x} = 2$ then (don't hold your breath) $x^2 = 1$, if $\abs{x}<2$ the only option is $\abs{x} = 1$ so $x^2 = x\cdot x \equiv 1\cdot 1 = 1$.


\item If $n=1$ then $x^1 = x = 1$ so trivially any power of $x$ is the identity. For $n>1$, expand $x^n = 1$ to get $x \cdot x \cdot \ldots \cdot x = 1$ where there are a total of $n$ factors of $x$. Group all but the first factor together to get $x \cdot x^{n-1} = 1$. By the uniqueness of the inverse, $x^{n-1} = x^{-1}$.


\item Start with $xy = yx$. Left multiply by $y^{-1}$ to get $y^{-1}xy = x$. Left multiply by $x^{-1}$ to get $x^{-1}y^{-1}xy = 1$. The other direction of implications follows from this operation being reversible since e.g. $y = \left( y^{-1} \right)^{-1}$.


\item \begin{enumerate}
\item The first formula is just counting: \begin{equation}
x^a x^b = \underbrace{x\ldots x}_{a \text{\ times}} \underbrace{x\ldots x}_{b \text{\ times}} = \underbrace{x\ldots x x \ldots x}_{a+b \text{\ times}} = x^{a+b}\ .\label{eq:1.19a1}
\end{equation} Note that this tells us that $x^ax^a = x^{2a}$. Inductively, we get that the product of $b$ copies of $x^a$ is $x^{ab}$. More clearly,
if $b=1$ then $(x^a)^b = x^a = x^{ab}$. For $b\geq 2$ we use induction: 
\begin{equation}
(x^a)^b = \underbrace{x^a \ldots x^a}_{b\text{\ times}} = x^a \underbrace{x^a \ldots x^a}_{b-1\text{\ times}} = x^a \cdot x^{\overbrace{a + \ldots + a}^{b-1 \text{\ times}}} = x^a\cdot x^{a(b-1)} = x^{a+a(b-1)} = x^{ab}\ . \label{eq:1.19a2}
\end{equation}
\item The equation $(x^a)^{-1} = x^{-a}$ seems weirdly tautological so let's rephrase it as $(x^a)^{-1} = (x^{-1})^a$: multiplying $a$ copies of the inverse of $x$ to $x^a$ gives the identity. Prove this inductively as well, starting with $a=1$. Obviously multiplying 1 copy of $x^{-1}$ to $x^1 = x$ gives the identity. Now for general $a>1$,
\begin{equation}
\underbrace{x^{-1}\ldots x^{-1}}_{a\text{\ times}} x^{a} = x^{-1}\underbrace{x^{-1}\ldots x^{-1}}_{a-1\text{\ times}} x^{a-1}x^1 = x^{-1}\left(\underbrace{x^{-1}\ldots x^{-1}}_{a-1\text{\ times}} x^{a-1}\right)x^1 = x^{-1}(1)x^1 = x^{-1}x^1 = 1\ . \label{eq:1.19b}
\end{equation}
\item {\color{red} aaaaaaaaaaaaaaa}
\end{enumerate}


\item First we show that $\abs{x^{-1}} \leq \abs{x}$. If $\abs{x} = n$, then by \eqref{eq:1.19b}, $1 = 1^{-1} = (x^n)^{-1} = (x^{-1})^n$, so the order of $x^{-1}$ is at most $n$. Now repeat the same process with $x$ and $x^{-1}$ switched to get that $\abs{x} \leq \abs{x^{-1}}$. Therefore the two must be equal.


\item If the order of $x$ is odd then $1 = x^{2k-1}$ for some $k\geq 1$. Multiplying both sides by $x$, we see that \begin{equation}
x = x\cdot x^{2k-1} = x^{2k} = (x^2)^k
\end{equation} where the second equality is by \eqref{eq:1.19a1} and the third equality is by \eqref{eq:1.19a2}.


\item We want to find $\abs{g^{-1}xg}$. Start multiplying it by itself to see the pattern:
\begin{align}
\left(g^{-1}xg\right)^1 &= g^{-1}xg \\
\left(g^{-1}xg\right)^2 &= g^{-1}xgg^{-1}xg = g^{-1}x1xg = g^{-1}x^2g 
\end{align}
so it looks like $\left(g^{-1}xg\right)^n = g^{-1}x^ng$. Prove this inductively:
\begin{align}
\left(g^{-1}xg\right)^n &= g^{-1}xg\cdot \left(g^{-1}xg\right)^{n-1} \\
&= g^{-1}xg\cdot g^{-1}x^{n-1}g \\
&= g^{-1}x(1)x^{n-1}g \\
&= g^{-1}x^{n}g 
\end{align}
Now we first show that $\abs{x} = n \implies \abs{g^{-1}xg} = n$. Using $x^n = 1$ and manipulating,
\begin{equation}
\left(g^{-1}xg\right)^n = g^{-1}x^{n}g = g^{-1}(1)g = g^{-1}g = 1\ .
\end{equation}
Now we do the implication the other way: $\abs{g^{-1}xg} = n \implies \abs{x} = n$. This is just more manipulation:
\begin{align}
1 &= \left(g^{-1}xg\right)^n \\
&= g^{-1}x^{n}g \\
g &= g g^{-1}x^n g \\
g &= x^n g \\
gg^{-1} &= x^n gg^{-1} \\
1 &= x^n
\end{align}


\item If $1 = x^n = x^{st}$, from \eqref{eq:1.19a2} we have $1 = (x^s)^t$ so $\abs{x^s} = t$.


\item For $n=0$ this is brainless: $(ab)^0 = 1 = 1\cdot 1 = a^0b^0$. Just as brainless for $n=1$: $(ab)^1 = ab = a^1b^1$. Now prove inductively for $n>1$, assuming $(ab)^{n-1} = a^{n-1}b^{n-1}$:
\begin{equation}
(ab)^n = (ab)(ab)^{n-1} = (ab)a^{n-1}b^{n-1}\ . \label{eq:1.24a}
\end{equation}
Now we want to show that $ba^{n-1} = a^{n-1}b$ which requires induction again. For $n=2$ we have $ba = ab$, which is true by the definition of commutativity. Then for $n>2$,
\begin{equation}
ba^{n-1} = ba^{n-2}\cdot a = a^{n-2}b\cdot a = a^{n-2}\cdot ba = a^{n-2} \cdot ab = a^{n-1}b \label{eq:1.24b}
\end{equation}
where, if we're being pedantic, we used \eqref{eq:1.19a1}. Continuing from \eqref{eq:1.24a},
\begin{equation}
(ab)^n = (ab)a^{n-1}b^{n-1} = a(ba^{n-1})b^{n-1} = a(a^{n-1}b)b^{n-1} = a^nb^n \label{eq:1.24c}
\end{equation}
again using \eqref{eq:1.19a1}.
Now we want to prove this for $n < 0$. More explicitly, since
\begin{equation}
(ab)^{-n} = \left((ab)^n\right)^{-1} = \left(a^nb^n\right)^{-1}
\end{equation}
(we used \eqref{eq:1.19b} in the first equality), we want to show that $a^{-n}b^{-n}$ is the inverse of $a^n b^n$. Multiplying the two looks like \begin{equation}
1 =^{?} a^{-n}b^{-n} \cdot a^n b^n\ .
\end{equation}
If we can show that $b^{-n}$ and $a^n$ commute then we're done since we get $1 = a^{-n}a^{n}\cdot b^{-n}b^n = 1\cdot 1$. We have to show this inductively (wow!), first do it for $n=1$:
\begin{equation}
ab = ba \implies b^{-1}ab = b^{-1}ba = a \implies b^{-1}abb^{-1} = ab^{-1} \implies b^{-1}a = ab^{-1}\ .
\end{equation}
Now inductively we assume that $b^{-(n-1)}a^{(n-1)} = a^{(n-1)}b^{-(n-1)}$ for $n>1$. Then
\begin{equation}
b^{-n}a^{n} = b^{-1}\left(b^{-(n-1)}a^{(n-1)} \right) a = b^{-1}\left( a^{(n-1)}b^{-(n-1)} \right) a\ .
\end{equation}
We need to show that we can commute $b^{-1}$ with powers of $a$, and vice versa. Luckily we already did this in \eqref{eq:1.24b} and \eqref{eq:1.24c}; e.g. we can just rename $b^{-1} \to b$ since the only fact that was used was that these two elements commute. Finally we have
\begin{equation}
b^{-n}a^{n} = b^{-1}a^{(n-1)}b^{-(n-1)}a = a^{(n-1)} b^{-1}b^{-(n-1)}a = a^{(n-1)} b^{-n}a = a^{(n-1)} a b^{-n} = a^nb^{-n}
\end{equation}
(using \eqref{eq:1.19a1} twice) as required.

\item If $x^2 = 1$ for all $x\in G$, then picking any two elements $x$ and $y$, their product squares to the identity: $1 = (xy)^2 = xyxy$. Then
\begin{equation}
xy = x(1)y = x(xyxy)y = (xx)yx(yy) = (1)yx(1) = yx\ .
\end{equation}


\item \underline{Closure}: We are told that for all $h,k \in H$, $hk \in H$. \\
\underline{Associativity}: This is inherited from the associativity of $G$ since all elements in $H$ are also in $G$. \\
\underline{Inverse}: We are given that for all $h \in H$, $h^{-1} \in H$. \\
\underline{Identity}: If $h \in H$, then $h^{-1} \in H$, and $hh^{-1} = 1 \in H$.


\item In the language of the previous exercise, let $H(x) = \{ x^n \ \vert\ n \in \bbz \}$. We want to show that for all $h,k \in H(x)$, we have $hk \in H(x)$ and $h^{-1} \in H(x)$. From the form of $H$ we have $h = x^m$ and $k = x^n$ for some $m,n\in \bbz$. Then $hk = x^mx^n = x^{m+n} \in H(x)$ by \eqref{eq:1.19a1} and $h^{-1} = \left(x^m\right)^{-1} = x^{-m} \in H(x)$ by \eqref{eq:1.19b}.


\item Given the groups $(A,\star)$ and $(B,\diamond)$, for the group $A\times B$ we have\begin{enumerate}
\item \underline{Associativity}: follows from algebra bashing using the associativity of $A$ and $B$ in the third equality\begin{align}
(a_1,b_1) \left[ (a_2,b_2) (a_3,b_3) \right] &= (a_1,b_1)\left[(a_2\star a_3, b_2\diamond b_3)\right] \\
&= (a_1\star(a_2\star a_3),b_1\diamond (b_2\diamond b_3)) \\
&= ((a_1\star a_2)\star a_3,(b_1\diamond b_2)\diamond b_3) \\
&= \left[a_1\star a_2,b_1\diamond b_2\right](a_3,b_3) \\
&= \left[(a_1,b_1)(a_2,b_2)\right](a_3,b_3)
\end{align}
\item \underline{Identity}: We want to show that $ae = ea = a$ for all $a \in A\times B$, with $e = (1_A,1_B)$:
\begin{align}
(a,b)(1_A,1_B) &= (a\star 1_A, b\diamond 1_B) = (a,b) \\
(1_A,1_B)(a,b) &= (1_A\star a, 1_B\diamond b) = (a,b) 
\end{align}
\item \underline{Inverse}: We want to show that $(a,b)^{-1} = (a^{-1},b^{-1})$:
\begin{align}
(a^{-1},b^{-1})(a,b) &= (a^{-1}\star a, b^{-1}\diamond b) = (1_A, 1_B) = 1
\end{align}
Since $a^{-1} \in A$ and $b^{-1} \in B$, $(a^{-1},b^{-1})$ is in $A\times B$ and this is well-defined.
\end{enumerate}


\item Use the same notation from the previous exercise. For arbitrary $(a,b)$ and $(c,d) \in A\times B$, 
\begin{equation}
(a,b)(c,d) = (a\star c,b\diamond d)\ .
\end{equation}
If $A\times B$ is abelian, then we also have
\begin{equation}
(a,b)(c,d) = (c,d)(a,b) = (c\star a, d\diamond b)
\end{equation}
which tells us that $a\star c = c\star a$ (for arbitrary $a,c\in A$) meaning $A$ is abelian. The same can be said for $B$. This works in the other direction: if $A$ and $B$ are both abelian then 
\begin{equation}
(a,b)(c,d) = (a\star c, b\diamond d) = (c\star a, d\diamond b) = (c,d)(a,b)\ .
\end{equation}


\item Proving $(a,1_B)$ and $(1_A,b)$ commute is trivial:
\begin{equation}
(a,1_B)(1_A,b) = (a\star 1_A, 1_B\diamond b) = (a,b) = (1_A\star a, b\diamond 1_B) = (1_A,b)(a,1_B)\ .
\end{equation}
Then, using the result from exercise 24 to split apart the product group (please don't make me prove the last equality),
\begin{equation}
(a,b)^n = \left( (a,1_B)(1_A,b) \right)^n = (a,1_B)^n (1_A,b)^n = (a^n,1_B)(1_A,b^n)\ .
\end{equation}
If we want $\abs{(a,b)} = n$ then we must have both $a^n = 1_A$ and $b^n = 1_B$. The smallest positive $n$ that satisfies this condition is the least common multiple of $\abs{a}$ and $\abs{b}$.


\item Following the hint, let $t(G)$ be the set of all elements in $G$ that are not their own inverse. From exercise 32 we know that, since $G$ is a finite group, the order of all $x \in G$ is finite. Choose an arbitrary element $x$ and call its order $n$, then 
\begin{equation}
1 = x^n = x\cdot x^{n-1} \implies x^{-1} = x^{n-1}\ .
\end{equation}
If $n > 2$, then $x \neq x^{n-1}$, so we have found two elements that are not their own inverse: $x$ and $x^{-1}$. Add these to $t(G)$. Continuing like this and finding all such pairs in $G$ (double-counting is fine; we just care that they come in pairs), we end up with an even number of elements in $t(G)$. Clearly, $1 \notin t(G)$ since the identity is its own inverse. That means the set $t(G) \cup {1}$ has an odd number of elements in it, and contains all $x \in G$ such that $\abs{x} = 1$ or $\abs{x} > 2$. Then $G - (t(G) \cup {1})$, assuming it is nonempty, contains all elements of order 2. If $\abs{G}$ is even, then this set must contain an odd (i.e. nonzero) number of elements, hence there is at least one element of order 2.


\item Prove the contrapositive. Suppose that $1, x, x^2, \ldots, x^{n-1}$ are not all distinct, i.e. there exist $a,b \in\bbz$ with $0 \leq a < b \leq n-1$ such that $x^a = x^b$. Then, multiplying both sides by $x^{-a}$ and using \eqref{eq:1.19a1} we see
\begin{equation}
1 = x^{-a}x^a = x^{-a}x^b = x^{b-a}
\end{equation}
so $\abs{x} = b-a < n$. Then it cannot be the case that $\abs{x} = n$. \\
Since there are $\abs{G}$ distinct elements in $G$ (duh), by the sequence of $\abs{G}+1$ elements $1, x, x^2, \ldots, x^{\abs{G}}$ cannot contain $\abs{G}+1$ distinct elements. Following the above argument, we see that $\abs{x} < \abs{G}+1$ meaning $\abs{x} \leq \abs{G}$.


\item If $x^n = 1$ for some $n$ then, for some given power $x^a$,
\begin{equation}
(x^i)^{-1} = x^{-i} = 1\cdot x^{-i} = x^n x^{-i} = x^{n-i}
\end{equation}
using \eqref{eq:1.19b} and \eqref{eq:1.19a1}, so if we want the two equal we require $i = n-i$ (more technically, both sides are modulo $n$), so $n = 2i$.
\begin{enumerate}
\item If $n$ is odd then there is no $i$ such that $n=2i$, so the above is impossible.
\item If $n$ is even, then there is only one solution for $i$ (modulo $n$), and we have $n = 2i$.
\end{enumerate}


\item Prove the contrapositive. Suppose that the elements $x^n, n\in\bbz$ are not all distinct. That means there exist $a,b \in\bbz$ such that $x^a = x^b$. Then, multiplying both sides by $x^{-a}$ and using \eqref{eq:1.19a1} we see
\begin{equation}
1 = x^{-a}x^a = x^{-a}x^b = x^{b-a}
\end{equation}
so we see $\abs{x} = b-a$, so $x$ does not have infinite order.


\item Suppose $\abs{x} = n$ for some finite (duh) integer $n>0$, so $x^n = 1 = 1^{-1} = x^{-n}$. Consider an arbitrary power $x^a$. By the division algorithm, we can write $a = qn + r$ for some $q,r\in\bbz$ and $0 \leq r < n$. Then
\begin{equation}
x^a = x^{qn+r} = x^{qn}x^r = (x^n)^qx^r = 1^qx^r = 1x^r = x^r
\end{equation}
where the second equality is by \eqref{eq:1.19a1} and the third equality is by \eqref{eq:1.19a2}. We see that $x^a = x^{a\md n}$.


\item {\color{red} not sure how to do this using the hint about cancellation rules, the only way I could do this was extremely ugly and trial and error; it'd be easier if I could use the fact that no element has order 3 but we're not there yet}

\end{enumerate}

























\addcontentsline{toc}{section}{1.2: Dihedral Groups}
\section*{\underline{1.2: Dihedral Groups}}

\addcontentsline{toc}{subsection}{Page 25 -- $D_{2n}$ Relations}
\subsection*{\underline{Page 25 -- $D_{2n}$ Relations}}
Given an regular $n$-gon with its vertices labeled 0 through $n-1$, denote the set of all vertices $S = \{0,1,\ldots, n-1\}$. This makes modular arithmetic much easier, and is equivalent to the textbook formulation; just add 1 to every label. Then any element $g \in D_{2n}$ sends each vertex to some (possibly the same) vertex, so we have a corresponding function $\sigma_g : S \to S$. We want to describe $r$ and $s$ in terms of their corresponding functions in order to understand arbitrary compositions of them. We (the textbook authors) have already established what $\sigma_r$ is:
\begin{equation}
\sigma_r(i) = \begin{cases}
i+1\ , & 0\leq i < n-1 \\
0\ , & i = n-1
\end{cases}
\end{equation}
which can be written in the succinct form (this is why I labeled them 0 to $n-1$)
\begin{equation}
\sigma_r(i) = (i+1)\md n\ .
\end{equation}
For $\sigma_s$ we go vertex-by-vertex. Since the axis of reflection goes through vertex $0$, clearly this is a fixed point with $\sigma_s(0) = 0$. The two adjacent vertices, labeled by $1$ and $n-1$, swap under the action of $s$. The two vertices after, $2$ and $n-2$ also switch. This goes all the way around the polygon, so the total description is
\begin{equation}
\sigma_s(i) = \begin{cases}
n-i\ , & 0< i \leq n-1\\
0\ , & i = 0
\end{cases}
\end{equation}
and, since $n-0 = n \equiv 0$ when considered modulo $n$, we can combine the two cases into
\begin{equation}
\sigma_s(i) = (n-i)\md n = -i\md n\ .
\end{equation}

\begin{enumerate}

\item Given $\sigma_r$, the action of $\sigma_{r^2}$ is easy:
\begin{equation}
\sigma_{r^2}(i) = \sigma_r (\sigma_r(i)) = \sigma_r( (i+1)\md n ) = \left( \left( (i+1)\md n \right) + 1\right) \md n  = ((i+1)+1)\md n = (i+2)\md n\ .
\end{equation}
Inductively,
\begin{align}
\sigma_{r^k}(i) = \sigma_r(\sigma_{r^{k-1}}(i)) = \sigma_r \left( (i+k-1)\md n \right) =  \left( \left((i+k-1)\md n\right) +1 \right)\md n = (i+k)\md n\ . \label{eq:2.0a}
\end{align}
The easiest way to show that $r^k$ and $r^l$ ($0 \leq k,l < n$ and $k \neq l$) are different transformations is to show that there exists an $i$ such that $\sigma_{r^k}(i) \neq \sigma_{r^l}(i)$. This is easily verified by choosing $i=0$: $\sigma_{r^k}(0) = k$ and $\sigma_{r^l}(0) = l$, so these cannot be the same function and hence $r^k$ and $r^l$ are not the same transformation. Showing that $r^n = 1$ is just as straightforward:
\begin{equation}
\sigma_{r^n}(i) = (i+n)\md n = i\md n = i \label{eq:2.0b}
\end{equation}
where the last equality is because $0 \leq i < n$. Since $\sigma_{r^n}$ is the identity map on all of $S$, $r^n$ is the identity element of $D_{2n}$.


\item We know what $\sigma_s$ is so we just compose it twice:
\begin{equation}
\sigma_{s^2}(i) = \sigma_s(\sigma_s(i)) = \sigma_s (-i\md n) = -(-i\md n)\md n = -(-i)\md n = i \md n = i
\end{equation}
where, again, the last equation is because $0 \leq i < n$. Just as in the above example, this shows that $s^2 = 1$. Obviously $s \neq 1$ so we see $\abs{s} = 2$.


\item This is easily shown by looking at the image of $0$ and $1$. The action of $s$ has $\sigma_s(0) = 0$ and $\sigma_s(1) = n-1$. If we want a power of $r$ that recreates $0 \mapsto 0$ then we want $\sigma_{r^k}(0) = (0+k)\md n = k\md n = 0$ so we have $k$ is a multiple of $n$, write it $k = an$. But then $r^{k} = r^{an} = (r^n)^a = 1^a = 1$ so we must have $\sigma_{r^k}(1) = 1$. The only way this can agree with $\sigma_s$ is if $n=2$, but then we don't even have a polygon to begin with.


\item We want to show that $sr^k \neq sr^l$ (again for $0\leq k,l < n$ and $k\neq l$). Again, this just requires showing that there is one element $i \in S$ such that $\sigma_{sr^k}(i) \neq \sigma_{sr^l}(i)$. Choose $i=0$ again; the left-hand side is
\begin{equation}
\sigma_{sr^k}(0) = \sigma_s(\sigma_{r^k}(0)) = \sigma_s(k\md n) = -(k\md n) \md n = -k\md n
\end{equation}
and likewise the right-hand side is $-l\md n$. Since $k\neq l$, $n-k \neq n-l$ so $-k$ and $-l$ are different equivalence classes modulo $n$.


\item We can show this for all $i\in S$, but first we need to establish what $\sigma_{r^{-1}}$ is. Since $r^n = 1$, we immediately have $r^{-1} = r^{n-1}$ which means
\begin{equation}
\sigma_{r^{-1}}(i) = \sigma_{r^{n-1}}(i) = (i+n-1)\md n = (i-1)\md n
\end{equation}
as expected. Now to show $rs = sr^{-1}$ by showing $\sigma_r \circ \sigma_s = \sigma_s \circ \sigma_{r^{-1}}$. The left-hand side is
\begin{equation}
\sigma_r(\sigma_s(i)) = \sigma_r(-i \md n) = ((-i\md n)+ 1)\md n = (-i+1) \md n\ .
\end{equation}
The right-hand side is
\begin{equation}
\sigma_s(\sigma_{r^{-1}}(i)) = \sigma_s( (i-1)\md n) = -((i-1)\md n)\md n = -(i-1)\md n = (-i+1)\md n\ .
\end{equation}


\item We have already shown equality for $k=1$. Now inductively commute all but one power of $r$, then do the last by itself:
\begin{equation}
r^ks = r(r^{k-1}s) = r(sr^{-(k-1)}) = (rs)r^{-(k-1)} = (sr^{-1})r^{-(k-1)} = s(r^{-1}r^{-(k-1)}) = sr^{-k}\ .
\end{equation}

\end{enumerate}




\addcontentsline{toc}{subsection}{Page 27 -- ``it is easy to see''}
\subsection*{\underline{Page 27 -- ``it is easy to see''}}
{\color{red} not sure, following the discussion for $X_{2n}$ I just get $r = r$ lol}




\addcontentsline{toc}{subsection}{Exercises}
\subsection*{\underline{Exercises}}
\begin{enumerate}

\item The elements of $D_{2n}$ are $\{1, r, r^2, \ldots, r^{n-1}, s, sr, sr^2, \ldots, sr^{n-1}\}$. A lot of these have the same answers: consider $sr^k$ for some $k\geq 1$. Squaring it gives
\begin{equation}
(sr^k)^2 = (sr^k)(sr^k) = s(r^ks)r^k = s(sr^{-k})r^k = s^2r^{-k}r^k = s^2 = 1 \label{eq:2.1}
\end{equation}
so every element of this form has order 2. We also know $s^2 = 1$, and we made no reference to $k$ being positive, so this is in fact true for all integer $k$. 
\begin{enumerate}
\item for $D_6$, $n=3$:
\begin{enumerate}
\item[$1$]: 1
\item[$r$]: 3 
\item[$r^2$]: $r^6 = (r^2)^3 = 1$ so order is 3; note this is $\frac{\text{lcm}(2,3)}{2}$ which is equivalent to $\frac{3}{(2,3)}$ (c.f. exercise 1.1.11)
\item[$s$]: 2
\item[$sr$]: 2
\item[$sr^2$]: 2
\end{enumerate}

\item for $D_8$, $n=4$:
\begin{enumerate}
\item[$1$]: 1
\item[$r$]: 4
\item[$r^2$]: $\frac{4}{(2,4)} = 2$
\item[$r^3$]: $\frac{4}{(3,4)} = 4$
\item[$s$]: 2
\item[$sr$]: 2
\item[$sr^2$]: 2
\item[$sr^3$]: 2
\end{enumerate}

\item for $D_{10}$, $n=5$:
\begin{enumerate}
\item[$1$]: 1
\item[$r$]: 5
\item[$r^2$]: $\frac{5}{(2,5)} = 5$
\item[$r^3$]: $\frac{5}{(3,5)} = 5$
\item[$r^4$]: $\frac{5}{(4,5)} = 5$
\item[$s$]: 2
\item[$sr$]: 2
\item[$sr^2$]: 2
\item[$sr^3$]: 2
\item[$sr^4$]: 2
\end{enumerate}

\end{enumerate}


\item If $x$ is not a power of $r$, then we can write it in the form $sr^k$. Now do the algebra bash:
\begin{equation}
rx = r(sr^k) = (rs)r^k = (sr^{-1})r^k = sr^{k-1} = (sr^k)r^{-1} = xr^{-1}\ . \label{eq:2.2}
\end{equation}


\item See \eqref{eq:2.1}. $D_{2n}$ is generated by $\{r, s\}$, but we can write $r = s\cdot sr$, so we can also generate the group with $\{sr, s\}$. We already knew $s^2 = 1$; we have shown $(sr)^2 = 1$.


\item If $n = 2k$, then $1 = r^n = r^{2k} = (r^k)^2$ so $r^k$ has order 2 (and is its own inverse). Any element of $D_{2n}$ is either $r^l$ for some $l$, or $sr^l$. Commutation is obvious in the first case since $r^kr^l = r^{k+l} = r^{l+k} = r^lr^k$ (I guess I showed it anyway). In the second case, we can use \eqref{eq:2.2} to inductively show that $r^lx = xr^{-l}$:
\begin{equation}
r^lx = r(r^{l-1}x) = r(xr^{1-l}) = (rx)r^{1-l} = xr^{-1}r^{1-l} = xr^{-l}\ .
\end{equation}
We see that $r^l$ commutes with any and all of these $x$ elements exactly when $r^{l} = r^{-l} = r^{n-l}$. For $0 \leq l < n$, the only solutions are the identity (trivial), and $n = 2l$. We see that $r^k$ is the only element (other than the identity) that commutes with everything.


\item This follows from my solution of the previous exercise; there is no solution to $n = 2l$ if $n$ is odd, so there is no second element that commutes with everything.


\item If $x$ and $y$ are order two then $x = x^{-1}$ and $y = y^{-1}$. Then $t^{-1} = (xy)^{-1} = y^{-1}x^{-1} = yx$ so 
\begin{equation}
tx = (xy)x = x(yx) = xt^{-1} \ .
\end{equation}


\item Two of the relations follow trivially since $a = s$ and $ab = ssr = r$:
\begin{align}
s^2 = 1 &\iff a^2 = 1 \\
(ab)^n = 1 &\iff r^n = 1
\end{align}
The last relation is also pretty easy:
\begin{align}
b^2 = 1 \iff b = b^{-1} \iff sr = (sr)^{-1} \iff sr = r^{-1}s^{-1} = r^{-1}s \iff r = sr^{-1}s \iff rs = sr^{-1}
\end{align}


\item The order is $n$... this seems so easy I think I have to be interpreting it incorrectly... we showed in \eqref{eq:2.0a} and \eqref{eq:2.0b} that $1, r, r^2, \ldots r^{n-1}$ are all distinct, and $r^n = 1$.


\item These problems seem intimidating but have a straightforward solution. If we orient the solid such that one face is on ``the bottom'', and specify the rotation of this face, we have fixed the entire solid. The rotational symmetry group of the face is $\bbz /3 \bbz$ so it has three elements. There are four possible faces we can choose to be on the bottom, so the total number of automorphisms of the vertices of a tetrahedron is $4\cdot 3 = 12$.


\item Likewise, the cube has six faces, each of which has 4 vertices, so $6\cdot 4 = 24$.


\item Eight faces, three vertices on a face, $8\cdot 3 = 24$.


\item Twelve faces, five vertices on a face, $12\cdot 5 = 60$.


\item Twenty faces, three vertices on a face, $20\cdot 3 = 60$.


\item Wasn't this given as an example? $\bbz = \langle\, 1\, \rangle$.


\item We can reach every element of $\bbz / n\bbz$ by repeatedly adding $1$ to itself. The only condition is that $n = 0$ so 
\begin{equation}
\bbz / n\bbz = \langle\, 1 \ \vert\ n(1) = 0 \, \rangle\ .
\end{equation}


\item Take exercise 7 and set $n=2$.


\item We already know that $\abs{X_{2n}} \leq 6$, and $x^3 = 1$ for any $n$. That means $x^2 = x^{-1}$ and the relation $xy = yx^2$ can be rewritten $xy = yx^{-1}$, which is the same as the relation for $D_{2n}$ with $y$ taking the role of $s$ and $x$ taking the role of $r$. The other relations are $y^2 = 1$, which is identical to $s^2 = 1$, and $x^n = 1$.
\begin{enumerate}
\item If $n = 3k$, then $1 = x^n = x^{3k} = (x^3)^k = 1^k$ and the condition that $x^n = 1$ is the same as the condition that $x^3 = 1$. We have $x^3 = y^2 = 1$ and $xy = yx^{-1}$, which is the presentation for $D_6$. The elements are $1, x, x^2, y, yx, yx^2$.
\item If $(3,n) = 1$ then there exist integers $a, b$ such that $3a + nb = 1$. Then
\begin{equation}
x = x^1 = x^{3a+nb} = x^{3a}x^{nb} = (x^3)^a (x^n)^b = 1^a 1^b = 1
\end{equation}
and the only elements of the group are $1$ and $y$.
\end{enumerate}


\item The presentation is $Y = \langle\, u,v \ \vert\ u^4=v^3=1, uv=v^2u^2\, \rangle$.
\begin{enumerate}
\item This is so easy that I used it without comment in the previous exercise. Take $v^3 = 1$ and apply $v^{-1}$ to both sides to get $v^2 = v^{-1}$.
\item Lots of algebra bashing:
\begin{equation}
vu^3 = vu1u^2 = vuv^3u^2 = v(uv)(v^2u^2) = v(v^2u^2)(uv) = v^3u^3v = u^3v\ .
\end{equation}
\item Since $u^4 = 1$, $u = (u^4)^2u = u^8u = u^9$. Then
\begin{equation}
vu = vu^9 = vu^3u^3u^3 = u^3vu^3u^3 = u^3u^3vu^3 = u^3u^3u^3v = u^9v = uv\ .
\end{equation}
\item Now that we know we can arbitrarily commute $u$ and $v$, the last relation becomes
\begin{equation}
uv = v^2u^2 = uvuv = (uv)^2 \implies 1 = uv\ .
\end{equation}
\item Use $u^4 = v^3 = 1$:
\begin{equation}
1 = 1\cdot 1 = u^4v^3 = uvuvuvu = (uv)^3u = 1^3u = u\ ,
\end{equation}
then since $uv = 1$ we see that we also have $v = 1$. Since $Y$ is generated by $u=1$ and $v=1$, and arbitrary products of $1$ still equal to $1$, we find that the only element of $Y$ is the identity.
\end{enumerate}


\end{enumerate}
























\addcontentsline{toc}{section}{1.3: Symmetric Groups}
\section*{\underline{1.3: Symmetric Groups}}
\addcontentsline{toc}{subsection}{Exercises}
\subsection*{\underline{Exercises}}
\begin{enumerate}

\item This was not fun.
\begin{enumerate}
\item $\sigma = \cycle{1,3,5}\cycle{2,4}$
\item $\tau = \cycle{1,5}\cycle{2,3}$
\item $\sigma^2 = \cycle{1,3,5}\cycle{2,4} \circ \cycle{1,3,5}\cycle{2,4} = \cycle{1,5,3}$
\item $\sigma\tau = \cycle{1,3,5}\cycle{2,4} \circ \cycle{1,5}\cycle{2,3} = \cycle{2,5,3,4}$
\item $\tau\sigma = \cycle{1,5}\cycle{2,3} \circ \cycle{1,3,5}\cycle{2,4} = \cycle{1,2,4,3}$
\item $\tau^2\sigma = \cycle{1,5}\cycle{2,3} \circ \cycle{1,5}\cycle{2,3} \circ \cycle{1,3,5}\cycle{2,4} = \cycle{1,3,5}\cycle{2,4}$
\end{enumerate}


\item This was so not fun that I wrote a program to do it. It also computes orders because why not.
\begin{lstlisting}[language=Python]
def f_to_perm(l):
    #given a mapping f as a list l, returns the cycle decomp
    #l[i] = j iff f(i) = j
    #choose a convention of l[0] = -1
    
    cycles = []
    i = 1
    done = [] #numbers we already considered
    while i<len(l):
        if len(done) == len(l)-1: #ignoring the l[0]
            break
        if i not in done:
            cycle = [i]
            current = l[i]
            done.append(i)
            while current != i:
                cycle.append(current)
                done.append(current)
                current = l[current]
            if len(cycle) > 1:
                cycles.append(cycle)
        i += 1
    return cycles

def perm_to_f(cycles,n):
    #given cycle decomp, returns a list l of the mapping
    #numbers are [1, n]
    
    l = [-1 for _ in range(n+1)] #l[0] = -1

    for c in cycles:
        for i in range(len(c)):
            l[c[i]] = c[(i+1)%len(c)]

    #cycles don't include fixed points
    for i in range(1,len(l)):
        if l[i] == -1:
            l[i] = i

    return l
   
def compose(fl,gl):
    #takes list fl of f, gl of g
    #returns list l of f \circ g
    assert len(fl) == len(gl) #don't need robustness so just check
    l = [-1]
    for i in range(1,len(fl)):
        l.append(fl[gl[i]])
    return l

def get_order(l,MAXN=100):
    power = 1
    power_l = l
    while power<MAXN:
        if len(f_to_perm(power_l)) == 0: #identity map
            return power
        power += 1
        power_l = compose(l,power_l)
    return -1 #recursion depth reached

def pprint(cycles):
    #cycle to latex
    s = '$'
    for c in cycles:
        s += "\\cycle{" + (",").join(str(i) for i in c) + "}"
    s += "$"
    print(s)
    
s = [-1,13,2,15,14,10,6,12,3,4,1,7,9,5,11,8]
t = [-1,14,9,10,2,12,6,5,11,15,3,8,7,4,1,13]

ss = compose(s,s)
st = compose(s,t)
ts = compose(t,s)
tts = compose(t,ts)

cs = f_to_perm(s)
ct = f_to_perm(t)
css = f_to_perm(ss)
cst = f_to_perm(st)
cts = f_to_perm(ts)
ctts = f_to_perm(tts)

print("  S:", f_to_perm(s), get_order(s))
pprint(cs)
print("\n  T:", f_to_perm(t), get_order(t))
pprint(ct)
print("\n SS:", f_to_perm(ss), get_order(ss))
pprint(css)
print("\n ST:", f_to_perm(st), get_order(st))
pprint(cst)
print("\n TS:", f_to_perm(ts), get_order(ts))
pprint(cts)
print("\nTTS:", f_to_perm(tts), get_order(tts))
pprint(ctts)
\end{lstlisting}

\begin{enumerate}
\item $\sigma = \cycle{1,13,5,10}\cycle{3,15,8}\cycle{4,14,11,7,12,9}$
\item $\tau = \cycle{1,14}\cycle{2,9,15,13,4}\cycle{3,10}\cycle{5,12,7}\cycle{8,11}$
\item $\sigma^2 = \cycle{1,5}\cycle{3,8,15}\cycle{4,11,12}\cycle{7,9,14}\cycle{10,13}$
\item $\sigma\tau = \cycle{1,11,3}\cycle{2,4}\cycle{5,9,8,7,10,15}\cycle{13,14}$
\item $\tau\sigma = \cycle{1,4}\cycle{2,9}\cycle{3,13,12,15,11,5}\cycle{8,10,14}$
\item $\tau^2\sigma = \cycle{1,2,15,8,3,4,14,11,12,13,7,5,10}$
\end{enumerate}


\item A single cycle of length $n$ has order $n$ on its own. This is evident by following the image  of the first element. For the purposes of illustration, choose $\sigma = \cycle{1,2,\ldots,n}$, without loss of generality since the numerical labels are arbitrary. For the first power, $\sigma$ sends 1 to 2. Then $\sigma^2$ sends 1 to 3, etc. so $\sigma^k$ sends 1 to $k+1$; this is basically just induction of the successor function on the natural numbers. When we reach the right end of the cycle we loop back to the beginning so $\sigma^k$ actually sends 1 to $(k+1)\md n$. The first power $k>0$ such that we loop back and return to $1 = (k+1)\md n$ is $k=n$. 

For multiple disjoint cycles, we want each cycle to exponentiate to the identity map on its elements. Given two cycles with orders $n_1$ and $n_2$, an exponent that is a multiple of $n_1$ will send the first cycle to the identity; likewise a multiple of $n_2$ will do the same for the second cycle. The first time both of these are true is $\text{lcm}(n_1,n_2)$. This extends for any number of disjoint cycles by induction. Therefore, the order of a permutation is the least common multiple of the lengths of the disjoint cycles in its cycle decomposition.

For the first exercise,
\begin{enumerate}
\item $\abs{\sigma} = \text{lcm}(3,2) = 6$
\item $\abs{\tau} = \text{lcm}(2,2) = 2$
\item $\abs{\sigma^2} = 3$
\item $\abs{\sigma\tau} = 4$
\item $\abs{\tau\sigma} = 4$
\item $\abs{\tau^2\sigma} = \text{lcm}(3,2) = 6$
\end{enumerate}
and for the second exercise,
\begin{enumerate}
\item $\abs{\sigma} = \text{lcm}(4,3,6) = 12$
\item $\abs{\tau} = \text{lcm}(2,5,2,3,2) = 30$
\item $\abs{\sigma^2} = \text{lcm}(2,3,3,3,2) = 6$
\item $\abs{\sigma\tau} = \text{lcm}(3,2,6,2) = 6$
\item $\abs{\tau\sigma} = \text{lcm}(2,2,6,3) = 6$
\item $\abs{\tau^2\sigma} = 13$
\end{enumerate}


\item 
\begin{enumerate}
\item The order of the group is $3! = 6$ so we can just list everything, using that the order of a cycle is just its length.
\begin{enumerate}
\item $\abs{1} = 1$
\item $\abs{\cycle{1,2}} = 2$
\item $\abs{\cycle{1,3}} = 2$
\item $\abs{\cycle{2,3}} = 2$
\item $\abs{\cycle{1,2,3}} = 3$
\item $\abs{\cycle{1,3,2}} = 3$
\end{enumerate}

\item For $S_4$ it's more annoying to list everything out.
\begin{enumerate}
\item $\abs{1} = 1$
\item There are $\binom{4}{2} = 6$ ways to pick two elements to make a two-cycle. The order of the two elements doesn't matter so there are 6 two-cycles, each with order 2:
\begin{equation}
\abs{\cycle{1,2}} = \abs{\cycle{1,3}} = \abs{\cycle{1,4}} = \abs{\cycle{2,3}} = \abs{\cycle{2,4}} = \abs{\cycle{3,4}} = 2
\end{equation}
\item There are $\binom{4}{3} = 4$ ways to pick three elements for a three-cycle. Then there are two distinct ways to order them since $\cycle{1,2,3} \neq \cycle{1,3,2}$. More generally, this factor of 2 comes from choosing a `first' element and then finding every permutation of the $3-1=2$ remaining elements $(2! = 2)$. These 8 three-cycles have order 3:
\begin{equation}
\abs{\cycle{1,2,3}} = \abs{\cycle{1,3,2}} = \abs{\cycle{1,2,4}} = \abs{\cycle{1,4,2}} = \abs{\cycle{1,3,4}} = \abs{\cycle{1,4,3}} = \abs{\cycle{2,3,4}} = \abs{\cycle{2,4,3}} = 3
\end{equation}
\item There's obviously only one set of elements to use for a four-cycle. Choosing $1$ to be the first element (we can always do this by cycling the permutation), there are $3! = 6$ ways to order the remaining elements, so there are 6 four-cycles with order 4:
\begin{equation}
\abs{\cycle{1,2,3,4}} = \abs{\cycle{1,2,4,3}} = \abs{\cycle{1,3,2,4}} = \abs{\cycle{1,3,4,2}} = \abs{\cycle{1,4,2,3}} = \abs{\cycle{1,4,3,2}} = 4
\end{equation}
\item We can also form the product of two disjoint two-cycles in the form $\cycle{a,b}\cycle{c,d}$. Since these cycles commute, and $\cycle{a,b} = \cycle{b,a}$, we can set $a=1$ without loss of generality. Then sine $\cycle{c,d} = \cycle{d,c}$ all that remains is to choose the element $b$ that's in the same cycle as the element 1. There are 3 choices, so there are 3 two-by-two cycles, of course with order 2:
\begin{equation}
\abs{\cycle{1,2}\cycle{3,4}} = \abs{\cycle{1,3}\cycle{2,4}} = \abs{\cycle{1,4}\cycle{2,3}} = 2
\end{equation}
\end{enumerate}
\end{enumerate}


\item Following exercise 3, $\text{lcm}(5,2,3,2) = 30$.


\item See exercise 4.


\item See exercise 4.


\item It suffices to show that there are an infinite number of 2-cycles in $S_\Omega$. Consider the subset of 2-cycles of the form $\cycle{1,n}$ for some $n\neq 1$. There are infinitely many possible values for $n$, so there are infinitely many 2-cycles in $S_\Omega$.


\item Go back to the code from exercise 2:
\begin{lstlisting}[language=Python]
def exercise9(cycle):
    s = perm_to_f(cycle,len(cycle[0]))
    sk = s
    for k in range(1, get_order(s)+1):
        print(k, get_order(sk))
        sk = compose(sk,s)
\end{lstlisting}
and now just run this for the three given cycles
\begin{enumerate}
\item \begin{lstlisting}[language=Python]
exercise9([[1,2,3,4,5,6,7,8,9,10,11,12]])
\end{lstlisting}
gives that $\abs{\sigma^k} = 12$ for $k = 1,5,7,11$, modulo 12 of course since $\sigma^{12} = 1$.

\item \begin{lstlisting}[language=Python]
exercise9([[1,2,3,4,5,6,7,8]])
\end{lstlisting}
gives that $\abs{\tau^k} = 8$ for $k = 1,3,5,7$ modulo 8.

\item \begin{lstlisting}[language=Python]
exercise9([[1,2,3,4,5,6,7,8,9,10,11,12,13,14]])
\end{lstlisting}
gives that $\abs{\omega^k} = 14$ for $k = 1,3,5,9,11,13$ modulo 14.
\end{enumerate}


\item See exercise 3 and note that the labeling of the elements is arbitrary, as is the choice to follow the $1$ element.


\item 


\item
\begin{enumerate}
\item We want a single $m$-cycle $\sigma$ such that, for some $k$, $\sigma^k$ breaks up into 2-cycles, meaning that $\sigma^{2k}$ is the identity map. The easiest thing to try is $2k=10$, so we have a 10-cycle where $\sigma^5(1) = 2$, $\sigma^5(2) = 1$, etc. The explicit form is
\begin{equation}
\sigma = \cycle{1,3,5,7,9,2,4,6,8,10}\ .
\end{equation}
\item This is not possible. Since the cycling the elements of the elements in the permutation is arbitrary, if we can take some $m$-cycle and exponentiate it to $\cycle{1,2}\cycle{3,4,5}$, then we can cycle the elements and then do the exponentiation again to get e.g. $\cycle{3,4}\cycle{5,6,7}$ (if $m<7$ then the last two values are modulo $m$) which is clearly not equal.
\end{enumerate}


\item Let $\sigma$ be the element of $S_n$ we care about, and let $c$ be its cycle decomposition. From exercise 3, we know that $\abs{\sigma}$ is the lcm of the lengths of the cycles in $c$. Suppose $c$ has a cycle whose length is at least 3. Then, since the lcm of a set is at least its maximum, we would have $\abs{\sigma} \geq 3$. Therefore, if it is the case that $\abs{\sigma} = 2$, then we cannot have any cycles of length three or more, so all cycles are 1-cycles or 2-cycles. We do not explicity write 1-cycles, so every cycle explicitly written in $c$ will have length 2. In the other direction, if $c$ is a product of disjoint 2-cycles, then again from exercise 3 we know that $\abs{\sigma} = \text{lcm}(2,2,2,\ldots) = 2$.


\item This is just a generalization of the previous example, and the proof proceeds similarly. From above we know that if $\abs{\sigma} = p$, then all cycles in $c$ have length at most $p$. If there is a cycle in $c$ whose length is $a \neq 1,p$, then $(\abs{\sigma}, a) \neq 1$, so $\abs{\sigma}$ cannot be prime. The contrapositive is that, if $\abs{\sigma} = p$, then the only cycles in $c$ are 1-cycles and $p$-cycles. The other direction is the same as before: $\abs{\sigma} = \text{lcm}(p,p,p,\ldots) = p$. \\
For the other part of the question, the product of a 2-cycle and a disjoint 3-cycle has order 6, but cannot be written as a 6-cycle (c.f. exercise 12.b).


\item See exercise 3. It'd be miserable to do most of this section without proving this first.


\item I can't believe there isn't a better notation for permutations: let
\begin{equation}
x^{\underline{n}} := x(x-1)\ldots(x-n+1)
\end{equation}
denote the falling factorial. Then there are $n^{\underline{m}}$ ways to pick $m$ distinct elements from the set $\{1,2,\ldots, n\}$ where the order matters. Once we've chosen the $m$ elements we want in our $m$-cycle, we can rotate the elements and still have the same cycle, which means that there are $m$ permutations that represent the same cycle. Therefore the number of cycles if $n^{\underline{m}}/m$.


\item To make two 2-cycles $\cycle{a,b}\cycle{c,d}$ we need four elements $a,b,c,d$; there are $n^{\underline{4}}$ (ordered) ways to pick them. Then we can switch $a \leftrightarrow b$, $c \leftrightarrow d$, and also swap the two cycles because they're disjoint. That means there are $2\cdot 2 \cdot 2 = 8$ ways that represent the same element of $S_n$, so the number we want is $n^{\underline{4}}/8$.


\item We want to partition the set $\{1, 2, 3, 4, 5\}$ every way possible and find the lcm of the cardinalities of the subsets. The easy ones are:
\begin{enumerate}
\item $1,1,1,1,1$ (identity) $\rightarrow n = 1$
\item $2,1,1,1$ (one 2-cycle) $\rightarrow n = 2$
\item $3,1,1$ (one 3-cycle) $\rightarrow n = 3$
\item $4,1$ (one 4-cycle) $\rightarrow n = 4$
\item $5$ (one 5-cycle) $\rightarrow n = 5$
\end{enumerate}
the other possibilities are
\begin{enumerate}
\item $2,2,1$ (two 2-cycle) $\rightarrow n = 2$
\item $3,2$ (a 2-cycle and 3-cycle) $\rightarrow n = 6$
\end{enumerate}
so $n = 1,2,3,4,5,6$.


\item The easy ones, from above, are $n = 1,2,3,4,5,6,7$. The more interesting ones are
\begin{enumerate}
\item $5,2 \rightarrow n = 10$
\item $4,3 \rightarrow n = 12$
\item $4,2,1 \rightarrow n = 4$
\item $3,3,1 \rightarrow n = 3$
\item $3,2,2 \rightarrow n = 6$
\item $3,2,1,1 \rightarrow n = 6$
\item $2,2,1,1,1 \rightarrow n = 2$
\item $2,2,2,1 \rightarrow n = 2$
\end{enumerate}
so $n = 1,2,3,4,5,6,7,10,12$.


\item We can reach any 3-cycle using $s = \cycle{1,2,3}$, and any 2-cycle with $r = \cycle{1,2}$, or by composing $r$ with $s$ since $s$ just cycles which two elements $r$ swaps. Experimentally, I found $rs^2 = sr = \cycle{1,3}$ so that's a relation between the two. Obviously $s^3 = r^2 = 1$, so I think a presentation is

\begin{equation}
S_3 = \langle\, s,r\ \vert\ s^3=r^2=1, rs^2 = sr\, \rangle\ .
\end{equation}

\end{enumerate}
























\addcontentsline{toc}{section}{1.4: Matrix Groups}
\section*{\underline{1.4: Matrix Groups}}
\addcontentsline{toc}{subsection}{Exercises}
\subsection*{\underline{Exercises}}

\begin{enumerate}


\item We have a matrix $\begin{bmatrix}a&b\\c&d\end{bmatrix}$ where $a,b,c,d$ are all $0$ or $1$, and we want $ad \neq bc$. Enumerate the possibilities:
\begin{itemize}
\item $ad = 1$, meaning $a=d=1$. Then $bc = 0$, which means either $b=0$, $c=0$, or both $b=c=0$. There are therefore three different invertible matrices of this type.
\item $ad = 0$, meaning $a=0$, $d=0$, or both $a=d=0$. There are again three different choices; all three require $bc=1$ meaning $b=c=1$.
\end{itemize}
the two options together make $3+3=6$ total elements.


\item ``Manual labor has become obsolete by the invention of the computer'' -- Nikola Tesla

\begin{lstlisting}[language=Python]
import numpy as np

def get_order_z2(m, timeout=100):
    identity = np.identity(m.shape[0])
    current = m
    for n in range(1,timeout):
        if np.array_equal(current,identity):
            return n
        current = (current @ m)%2
    return -1

#ad = 1
g1 = np.array([[1,0],[0,1]])
g2 = np.array([[1,0],[1,1]])
g3 = np.array([[1,1],[0,1]])

#bc = 1
g4 = np.array([[0,1],[1,0]])
g5 = np.array([[0,1],[1,1]])
g6 = np.array([[1,1],[1,0]])

print(get_order_z2(g1))
print(get_order_z2(g2))
print(get_order_z2(g3))
print(get_order_z2(g4))
print(get_order_z2(g5))
print(get_order_z2(g6))
\end{lstlisting}
tells us that
\begin{enumerate}
\item $\begin{bmatrix}1&0\\0&1\end{bmatrix}$ has order 1 (wow!)
\item $\begin{bmatrix}1&0\\1&1\end{bmatrix}$ has order 2
\item $\begin{bmatrix}1&1\\0&1\end{bmatrix}$ has order 2
\item $\begin{bmatrix}0&1\\1&0\end{bmatrix}$ has order 2
\item $\begin{bmatrix}0&1\\1&1\end{bmatrix}$ has order 3
\item $\begin{bmatrix}1&1\\1&0\end{bmatrix}$ has order 3
\end{enumerate}


\item The first two elements I picked didn't commute:
\begin{equation}
\begin{bmatrix}1&0\\1&1\end{bmatrix}\begin{bmatrix}1&1\\0&1\end{bmatrix} = \begin{bmatrix}1&1\\1&0\end{bmatrix} \neq \begin{bmatrix}0&1\\1&1\end{bmatrix} = \begin{bmatrix}1&1\\0&1\end{bmatrix}\begin{bmatrix}1&0\\1&1\end{bmatrix}\ . \label{eq:4.3}
\end{equation}


\item If $n$ is not a prime, then $(\bbz/n\bbz)^\times \ncong \bbz/n\bbz$, i.e. there exists at least one element $g$ that does not have a multiplicative inverse modulo $n$. Therefore $F - \{0\}$ cannot be a group.


\item Let $M$ be an arbitrary element of $GL_n(F)$. If $F$ has finite cardinality, then, ignoring the determinant constraint, each $M_{ij}$ can be any of $\abs{F}$ elements, so there are $\abs{F}^{n^2}$ possibilities for $M$ and hence only a finite number of matrices in the group. The determinant constraint means that some of these $M$s are not in the group (e.g. the zero matrix), making $\abs{GL_n(F)} < \abs{F}^{n^2}$. In the other direction, there is a subset $S$ (subgroup, but whatever) of $GL_n(F)$ that consists of matrices of the form $eI_n$, where $e\in F$ and $I_n$ is the $n$-by-$n$ identity matrix. If $\abs{GL_n(F)}$ is finite, then $\abs{S} < \abs{GL_n(F)}$ is finite, and we clearly have a bijection from $S$ to $F$ so $\abs{F}$ is also finite.


\item See exercise 5.


\item {\color{red} can't get the right form despite 5000 attempts}


\item In \eqref{eq:4.3} we gave explicit 2-by-2 matrices in any field $F$ (since the entries are just the multiplicative and additive identities) that do not commute; hence $GL_2(F)$ is nonabelian for any F. We can reuse this for any $n>2$ by just extending these matrices with the identity on the diagonal, e.g. for $GL_4(F)$:
\begin{equation}
\begin{bmatrix}\begin{bmatrix}1&0\\1&1\end{bmatrix}&0\\0&I_2\end{bmatrix} \begin{bmatrix}\begin{bmatrix}1&1\\0&1\end{bmatrix}&0\\0&I_2\end{bmatrix} = 
\begin{bmatrix}\begin{bmatrix}1&1\\1&0\end{bmatrix}&0\\0&I_2\end{bmatrix} \neq 
\begin{bmatrix}\begin{bmatrix}0&1\\1&1\end{bmatrix}&0\\0&I_2\end{bmatrix} =
\begin{bmatrix}\begin{bmatrix}1&1\\0&1\end{bmatrix}&0\\0&I_2\end{bmatrix}
\begin{bmatrix}\begin{bmatrix}1&0\\1&1\end{bmatrix}&0\\0&I_2\end{bmatrix}
\end{equation}
is an explicit example of two noncommuting 4-by-4 matrices over $F$, where $I_2$ is the 2-by-2 identity matrix.

\item Algebra bash: multiply the second two first...
\begin{align}
\begin{bmatrix}a&b\\c&d\end{bmatrix}\left(\begin{bmatrix}e&f\\g&h\end{bmatrix}\begin{bmatrix}i&j\\k&l\end{bmatrix}\right) &= \begin{bmatrix}a&b\\c&d\end{bmatrix}\begin{bmatrix}ei+fk&ej+fl\\gi+hk&gj+hl\end{bmatrix} \nonumber \\
&= \begin{bmatrix}a(ei+fk)+b(gi+hk)&a(ej+fl)+b(gj+hl)\\c(ei+fk)+d(gi+hk)&c(ej+fl)+d(gj+hl)\end{bmatrix} \nonumber \\
&= \begin{bmatrix}aei+afk+bgi+bhk&aej+afl+bgj+bhl\\cei+cfk+dgi+dhk&cej+cfl+dgj+dhl\end{bmatrix}
\end{align}
...and now multiply the first two first
\begin{align}
\left(\begin{bmatrix}a&b\\c&d\end{bmatrix}\begin{bmatrix}e&f\\g&h\end{bmatrix}\right)\begin{bmatrix}i&j\\k&l\end{bmatrix} &= \begin{bmatrix}ae+bg&af+bh\\ce+dg&cf+dh\end{bmatrix}\begin{bmatrix}i&j\\k&l\end{bmatrix} \nonumber \\
&= \begin{bmatrix}(ae+bg)i+(af+bh)k&(ae+bg)j+(af+bh)l\\(ce+dg)i+(cf+dh)k&(ce+dg)j+(cf+dh)l\end{bmatrix} \nonumber \\
&= \begin{bmatrix}aei+bgi+afk+bhk&aej+bgj+afl+bhl\\cei+dgi+cfk+dhk&cej+dgj+cfl+dhl\end{bmatrix}\ .
\end{align}
The two are equal by commutativity of addition of the reals.


\item I am allergic to using indices.
\begin{enumerate}
\item Since none of $a,c,d,f$ are zero, $ad$ and $cf$ are both nonzero:
\begin{equation}
\begin{bmatrix}a&b\\0&c\end{bmatrix}\begin{bmatrix}d&e\\0&f\end{bmatrix} = \begin{bmatrix}ad&ae+bf\\0&cf\end{bmatrix} \label{eq:4.10a}
\end{equation}

\item All $g \in G$ are invertible since $\det g = ac \neq 0$. Use the general form of the inverse of a 2-by-2 matrix:
\begin{equation}
\left( \begin{bmatrix}a&b\\0&c\end{bmatrix} \right)^{-1} = \frac{1}{ac}\begin{bmatrix}c&-b\\-0&a\end{bmatrix} = \begin{bmatrix}\frac{1}{a}&-\frac{b}{ac}\\0&\frac{1}{c}\end{bmatrix} \label{eq:4.10b}
\end{equation}

\item $G$ is closed under matrix multiplication (from part a), associative (inherited from $GL_2(\bbr)$), contains the identity ($a=c=1$ and $b=0$), and contains an inverse for every element (from part b).
\item If $a=c$ and $d=f$, then $ad = cf$ in \eqref{eq:4.10a} so the product has equal entries on the diagonal. We also see that $\frac{1}{a} = \frac{1}{c}$ in \eqref{eq:4.10b} so the inverse of one of these matrices satisfies the same condition. By the same argument as in part c, this is a subgroup of $GL_2(\bbr)$.
\end{enumerate}


\item
\begin{enumerate}
\item We don't have any restrictions on $a,b,c,d,e,f$ so all that matters is the placement of the $0$s and $1$s:
\begin{equation}
XY = \begin{bmatrix}1&a&b\\0&1&c\\0&0&1\end{bmatrix}\begin{bmatrix}1&d&e\\0&1&f\\0&0&1\end{bmatrix} = \begin{bmatrix}1&d+a&e+af+b\\0&1&f+c\\0&0&1\end{bmatrix} \label{eq:4.11a}
\end{equation}
We can show noncommutativity using only the identity elements $0$ and $1$, which are the same regardless of $F$. Let $a=f=0$, and $b=c=d=e=1$. The upper-right-most entry in $XY$ is $e+af+b = 1+0+1$, while the same entry in $YX$ is $b+dc+e = 1+1+1$. The two are only equal if $0=1$.

\item The inverse is
\begin{equation}
\left( \begin{bmatrix}1&a&b\\0&1&c\\0&0&1\end{bmatrix} \right)^{-1} = \begin{bmatrix}1&-a&ac-b\\0&1&-c\\0&0&1\end{bmatrix}
\end{equation}
which obviously satisfies the conditions of the subgroup.

\item In exercise 9 we proved that matrix multiplication for 2-by-2 matrices of reals was associative. We only used associativity of multiplication and commutativity of addition, so this is true for any field (not just the reals). The 3-by-3 case is the same exact computation just with more terms per entry.

\item More code, just iterating through all ordered triples $(a,b,c)$:
\begin{lstlisting}[language=Python]
for a in [0,1]:
    for b in [0,1]:
        for c in [0,1]:
            X = np.array([[1,a,b],[0,1,c],[0,0,1]])
            print(a,b,c, get_order_z2(X))
\end{lstlisting}
yields
\begin{enumerate}
\item $(a,b,c)=(0,0,0)$ has order 1 (identity)
\item $(0,0,1)$ has order 2
\item $(0,1,0)$ has order 2
\item $(0,1,1)$ has order 2
\item $(1,0,0)$ has order 2
\item $(1,0,1)$ has order 4
\item $(1,1,0)$ has order 2
\item $(1,1,1)$ has order 4
\end{enumerate}

\item Consider $Y=X$ in \eqref{eq:4.11a}:
\begin{equation}
X^2 = \begin{bmatrix}1&a&b\\0&1&c\\0&0&1\end{bmatrix}^2 = \begin{bmatrix}1&2a&ac+2b\\0&1&2c\\0&0&1\end{bmatrix}\ . \label{eq:4.11e1}
\end{equation}
By induction,
\begin{equation}
X^n = \begin{bmatrix}1&a&b\\0&1&c\\0&0&1\end{bmatrix}^n = \begin{bmatrix}1&na&\text{(mess)}\\0&1&nc\\0&0&1\end{bmatrix} \label{eq:4.11e2}
\end{equation}
First consider the case where at least one of $a,c$ are nonzero. If we want $X^n = 1$ for some $n$, then we must have $na = nc = 0$. Obviously $n>0$, so we require two nonzero values to multiply to zero which is impossible. In the other case when $a=c=0$, the top-right entry in \eqref{eq:4.11e1} is just $2b$ so the top-right entry in \eqref{eq:4.11e2} is $nb$ and we have the same argument.


\end{enumerate}

\end{enumerate}



























\addcontentsline{toc}{section}{1.5: The Quaternion Group}
\section*{\underline{1.5: The Quaternion Group}}
\addcontentsline{toc}{subsection}{Exercises}
\subsection*{\underline{Exercises}}
\begin{enumerate}

\item It's not worth it to write a program for this:
\begin{enumerate}
\item 1: order 1 obviously
\item $-1$: order 2, given in group definition
\item $i$: order 4, $i^2 = -1$ so $i^3 = -i$ and $i^4 = 1$
\item $j$: order 4, same as above
\item $k$: order 4, same as above
\item $-i$: order 4, $(-i)^2 = (-1\cdot i)^2 = (i\cdot -1)(-1 \cdot i) = i\cdot 1 \cdot i = i^2 = -1$, so $\left((-i)^2\right)^2 = 1$
\item $-j$: order 4, same as above
\item $-k$: order 4, same as above
\end{enumerate}

\renewcommand{\arraystretch}{1.25}
\item For each table, the entry in row $a$ and column $b$ is the product $ab$, e.g. $\cycle{1,3,2}\cycle{2,3} = \cycle{1,3}$.
\begin{enumerate}
\item $S_3$:
\begin{table}[h!] \centering
\begin{tabular}{c|c|c|c|c|c|c|}
& 1 & $\cycle{1,2}$ & $\cycle{1,3}$ & $\cycle{2,3}$ & $\cycle{1,2,3}$ & $\cycle{1,3,2}$ \\ \hline
1 & 1 & $\cycle{1,2}$ & $\cycle{1,3}$ & $\cycle{2,3}$ & $\cycle{1,2,3}$ & $\cycle{1,3,2}$  \\ \hline
$\cycle{1,2}$ & $\cycle{1,2}$ & 1 & $\cycle{1,3,2}$ & $\cycle{1,2,3}$ & $\cycle{2,3}$ & $\cycle{1,3}$ \\ \hline
$\cycle{1,3}$ & $\cycle{1,3}$ & $\cycle{1,2,3}$ & 1 & $\cycle{1,3,2}$ & $\cycle{1,2}$ & $\cycle{2,3}$ \\ \hline
$\cycle{2,3}$ & $\cycle{2,3}$ & $\cycle{1,3,2}$ & $\cycle{1,2,3}$ & 1 & $\cycle{1,3}$ & $\cycle{1,2}$ \\ \hline
$\cycle{1,2,3}$ & $\cycle{1,2,3}$ & $\cycle{1,3}$ & $\cycle{2,3}$ & $\cycle{1,2}$ & $\cycle{1,3,2}$ & 1 \\ \hline
$\cycle{1,3,2}$ & $\cycle{1,3,2}$ & $\cycle{2,3}$ & $\cycle{1,2}$ & $\cycle{1,3}$ & 1 & $\cycle{1,2,3}$ \\ \hline
\end{tabular}
\end{table}

\item $D_8$:
\begin{table}[h!] \centering
\begin{tabular}{c|c|c|c|c|c|c|c|c|}
 & 1 & $r$ & $r^2$ & $r^3$ & $s$ & $sr$ & $sr^2$ & $sr^3$ \\ \hline
1 & 1 & $r$ & $r^2$ & $r^3$ & $s$ & $sr$ & $sr^2$ & $sr^3$ \\ \hline
$r$ & $r$ & $r^2$ & $r^3$ & 1 & $sr^3$ & $s$ & $sr$ & $sr^2$ \\ \hline
$r^2$ & $r^2$ & $r^3$ & 1 & $r$ & $sr^2$ & $sr^3$ & $s$ & $sr$ \\ \hline
$r^3$ & $r^3$ & 1 & $r$ & $r^2$ & $sr$ & $sr^2$ & $sr^3$ & $s$ \\ \hline
$s$ & $s$ & $sr$ & $sr^2$ & $sr^3$ & 1 & $r$ & $r^2$ & $r^3$ \\ \hline
$sr$ & $sr$ & $sr^2$ & $sr^3$ & $s$ & $r^3$ & 1 & $r$ & $r^2$ \\ \hline
$sr^2$ & $sr^2$ & $sr^3$ & $s$ & $sr$ & $r^2$ & $r^3$ & 1 & $r$ \\ \hline
$sr^3$ & $sr^3$ & $s$ & $sr$ & $sr^2$ & $r$ & $r^2$ & $r^3$ & 1 \\ \hline
\end{tabular}
\end{table}


\item $Q_8$:
\begin{table}[h!] \centering
\begin{tabular}{c|c|c|c|c|c|c|c|c|}
 & 1 & $-1$ & $i$ & $-i$ & $j$ & $-j$ & $k$ & $-k$ \\ \hline
1 & 1 & $-1$ & $i$ & $-i$ & $j$ & $-j$ & $k$ & $-k$ \\ \hline
$-1$ & $-1$ & 1 & $-i$ & $i$ & $-j$ & $j$ & $-k$ & $k$ \\ \hline
$i$ & $i$ & $-i$ & $-1$ & 1 & $k$ & $-k$ & $-j$ & $j$ \\ \hline
$j$ & $j$ & $-j$ & $-k$ & $k$ & $-1$ & 1 & $i$ & $-i$ \\ \hline
$k$ & $k$ & $k$ & $j$ & $-j$ & $-i$ & $i$ & $-1$ & 1 \\ \hline
$-i$ & $-i$ & $i$ & 1 & $-1$ & $-k$ & $k$ & $j$ & $-j$ \\ \hline
$-j$ & $-j$ & $j$ & $k$ & $-k$ & 1 & $-1$ & $-i$ & $i$ \\ \hline
$-k$ & $-k$ & $k$ & $-j$ & $j$ & $i$ & $-i$ & 1 & $-1$ \\ \hline
\end{tabular}
\end{table}


\end{enumerate}

\end{enumerate}




















\end{document}